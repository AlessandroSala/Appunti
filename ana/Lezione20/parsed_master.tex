
\section{Applicazione della trasformata di Fourier alle equazioni differenziali}
Data un'equazione differenziale in $u=u(x)$, applicando la trasformata si ottiene un'equazione algebrica in $\hat{u}$. Risolvendo per $\hat{u}$, si trova quindi $u$.
\subsection{Esempio} 
\[u'(x)-u(x)=e^{-x}H(x)\ \ H(x):=\begin{cases}
	1\ \ \ge 0\\
	0\ \ x<0
\end{cases}\]
Da risolvere per q.o. $x\in \R$.
\\Cerco soluzioni $u\in L^{1}(\R)\cap\text{AC}(\R)$ 
\[\hat{u'}-\hat{u}= \widehat{e^{-x}H(x)}\]
\[i\xi \hat{u}-\hat{u}=\frac{1}{1+i\xi}\]
\[\hat{u}(\xi)=-\frac{1}{1+\xi^2}\implies u(x)=-\frac{1}{2}e^{-|x|}\]
\subsection{Formule di inversione per la trasformata di Fourier in $L^1$ }
Sia $u\in L^{1}(\R^n)$ tale che $\hat{u}\in L^{1}(\R^n)$. Allora:
\[u(-x)=\frac{1}{(2\pi)^n}\hat{\hat{u}}(x)\ \ x\in \R^n\]
\[\check{u}(x):=u(-x)\]
\textbf{Problema:} Esiste uno spazio $X$ tale che 
\[\mathcal F:X\to X\]
e valga in $X$ la formula di inversione?
\\\textbf{Risposta:} Sì, $X=L^{2}(\R^n)$.
\\Sia $u\in L^{2}(\R^n)$ 
\[u(-x)=\frac{1}{(2\pi)^n}\hat{\hat{u}}(x)\ \ x\in \R^n\]
Ma chi è $u\in L^{2}(\R^n)\setminus L^{1}(\R^n)$?
\[|\hat{u}(\xi)|<+\infty\text{ se }u\in L^{1}(\R)\]
\subsection{Trasformata di Fourier in $L^2$ }
\begin{tcolorbox}
\textbf{Spazio delle funzioni a decrescenza rapida}
\\$\mathcal S(\R^n):= \{u\in C^\infty(\R^n)\} $ tali che 
\[ \forall \alpha, \beta\text{ multiindici }\sup_{x\in \R^n}|x^\alpha D^\beta u(x)|<+\infty\]
\end{tcolorbox}
\textbf{Osservazioni} 
\begin{itemize}
	\item $C^\infty_0(\R^n)\subset \mathcal S(\R^n)\subseteq  C^\infty(\R^n)$ 
	\item $u\in \mathcal S(\R^n)\implies x^\gamma u, D^\gamma u\in S(\R^n)$ 
	\item $\mathcal S(\R^n) \subseteq  L^{1}(\R^n)$
	\item Valgono in $S(\R^n)$ le seguenti formule:
\end{itemize}
\begin{enumerate}
	\item $\widehat{D^\alpha u}=i^{|\alpha|}\xi^\alpha \hat{u}$ 
	\item $D^\alpha \hat{u}=(-i)^{|\alpha|}\widehat{x^\alpha u}$
\end{enumerate}
\begin{itemize}
	\item Formula di inversione in $\mathcal S(\R^n)$ 
		\\Sia $u\in \mathcal S(\R^N)$. Allora $\hat{u}\in \mathcal S(\R^n)$ 
		\[\check u=\frac{1}{(2\pi)^n}\hat{\hat{u}}\]
\end{itemize}
\textbf{Dimostrazione: }Se dimostro che $\hat{u}\in \mathcal S(\R^n)$, allora vale la formula di inversione. (come conseguenza di quelle in $L^{1}(\R^n)$, perché $u,\hat{u}\in \mathcal S(\R^n)\subseteq  L^{1}(\R^n)$ ).
\begin{itemize}
	\item $\hat{u}\in C^\infty(\R^n) $ perché $\ \forall \alpha,\ x^\alpha u \in L^{1}(\R^n)$
		\[u\in \mathcal S(\R^n)\implies x^\alpha u \in \mathcal S(\R^n)\subseteq  L^{1}(\R^n)\]
		Dunque per la prop. 2 $\hat{u}$ ha derivate di ogni ordine
	\item $\sup_{x\in \R^n}|\xi^\alpha D^\beta \hat{u}(\xi)|<+\infty\ \forall \alpha,\beta$
		\\Perché $\xi^\alpha D^\beta \hat{u}$ è la trasformata di una funzione di $L^{1}(\R^n)$ 
		\[\xi^\alpha D^\beta \hat{u}\sim \xi^\alpha \widehat{x^\beta u}\sim \widehat{D^\alpha (x^\beta u)}\]
	E $D^\alpha(x^\beta u)$ appartiene a $L^{1}(\R^n)$ perché appartiene a $\mathcal S(\R^n)$ 
	\\Quindi per Riemann-Lebesgue la sua trasformata sta in $L^{\infty}(\R^n)$

\end{itemize}
\subsubsection{Altre proprietà di $\mathcal F$ in $\mathcal S$}
\[ \int_{\R^n}^{} \hat{u}v=\int_{\R^n}^{} \hat{v}u\]
\textbf{Identità di Plancherel} 
\[\int_{\R^n}^{} |u|^2=\frac{1}{(2\pi)^n}\int_{\R^n}^{} |\hat{u}|^2\]
\textbf{Prodotto di convoluzione} 
\[\widehat{u * v}=\hat{u}\cdot \hat{v}\]
\[\widehat{uv}=(2\pi)^{-n}\hat{u}* \hat{v}\]
\divider
\begin{tcolorbox}
	\textbf{Definizione: }Sia $u\in L^{2}(\R^n),$ e sia $\{u_h\} \subseteq  \mathcal S(\R^n)$ tale che $u_h\to u$ in $L^{2}(\R^n)$.\\
	Considero $\hat{u_h}\subseteq  \mathcal S(\R^n)$ e definisco 
	\[\hat{u}:=\lim_{h \to +\infty} \hat{u}_h\]

\end{tcolorbox}
(*) Esiste una tale $u_h$ perché $C_0^\infty(\R^n)$ è denso in $L^{2}(\R^n)$.\\
\textbf{Idea} 
\[\mathcal S\ni u_h\to u\in L^{2}\]
Applico Fourier 
\[\mathcal S \ni  \hat{u}_h\to \hat{ u}\in L^{2}\]
\textbf{Osservazioni} 
\begin{itemize}
	\item $\exists \lim_{h \to +\infty} \hat{u}_h$ (in $L^{2}(\R^n)$ ) perché $\hat{u}_h$ è di Cauchy: infatti
		\[\|\hat{u}_h-\hat{u}_k\|_2=\frac{1}{(2\pi)^n}\|u_h-u_k\|_2\text{ (identità di Plancherel)}\]
		e la successione è di Cauchy
	\item $\lim_{h \to +\infty} \hat{u}_h$ è indipendente dalla scelta di $u_h $: $u_h\xrightarrow{L^2}u$,  $v_h\xrightarrow{L^2}v$ \\Dunque $\lim_{h \to +\infty}\hat{u}_h=\lim_{h \to +\infty} \hat{v}_h $ \\perché $u_h-v_h\xrightarrow{L^2}0\ \substack{\implies \\\text{Plancherel}}\ \hat{u}_h-\hat{v}_h\to 0$
	\item Se $u\in L^{2}(\R^n)\cap L^{1}(\R^n)$, allora le trasformate coincidono
		\\Infatti, prendendo $u_h \subseteq  \mathcal S(\R^n)$ tale che (possibile tramite mollificatori)
		\[\begin{cases}
			u_h\xrightarrow{L^1}u\\
		u_h\xrightarrow{L^2}u
	\end{cases}\]
	si ha:
	\[\begin{cases}
		\hat{u}_h\xrightarrow{L^\infty}\hat{u} \text{ Trasformata in }L^1,\ \text{perché }\mathcal F:L^1\to L^\infty\text{ continuo}\\
		\\ \hat{u}_h\xrightarrow{L^2}\hat{u} \text{ Trasformata in }L^2\text{, per definizione di }\mathcal F\text{ in }L^{2}(\R^n) 
	\end{cases}\]
\item Vale l'id. di Plancherel in $L^{2}(\R^n)$ 
	\[\|u\|_{L^2}^2=(2\pi)^{-n}\|\hat{u}\|^2_{L^2}\ \forall u\in L^{2}(\R^n)\]
	Infatti, presa $u_h \subseteq  \mathcal S(\R^n):u_h\xrightarrow{L^2}u$:
	\[\|u_h\|^2_{L^2}=(2\pi)^{-n}\|\hat{u}_h\|^2_{L^2}\ \forall h\]
	Passando al limite si ottiene l'identità.
\end{itemize}
\subsubsection{Calcolo nella pratica}
Basta osservare che la successione
\[u_h=u\cdot \chi_{(-h,h)}\]
\[u_h\xrightarrow {}u\text{ in }L^{2}(\R)\]
\[\implies \hat{u}_h\to \hat{u}\text{ in }L^{2}(\R)\]
Per Plancherel.\\
Qunidi,
\[\hat{u}(\xi)=\lim_{h \to +\infty} \hat{u}_h(\xi)\]
\[=\lim_{h \to +\infty} \int_{\R}^{} u_h(x)e^{-i\xi x}dx\]
\\La formula di inversione vale in $L^{2}$, basta prendere $u_h \subseteq  \mathcal S(\R^n),u_h\xrightarrow{L^2}u$ \\
Sappiamo:
\[\check u_h=(2\pi)^{-n}\hat{\hat{u}}_h\ \forall h\]
Basta passare al limite per $h\to +\infty$ 
\[\check u=(2\pi)^{-n}\hat{\hat{u}}\]


