
\section{Introduzione Analisi Complessa}
\textbf{Definizione}
\\Una funzione di variabile complessa è una funzione $f:\Omega\subseteq \mathbb{C}\to \mathbb{C}$
\\Numero complesso: $z=x+iy$ dove $x,y\in \mathbb{R}\ i^{2}=-1$
\\$f(z)=u(z)+iv(z)=u(x,y)+iv(x,y)$
\\Dunque ad ogni funzione complessa è possibile associare due funzioni reali in due variabili
\\$f\leftrightarrow u,v:\Omega\subseteq \mathbb{R}^2\to \mathbb{R}$
\\\textbf{Esempi di funzioni elementari}
\[
	f(z)=z_0\in\mathbb{C},\ z_0=x_0+iy_0,\ u=x_0,\  v=y_0
\]
\[
	f(z)=Rez,\  z=x+iy,\  u(x,y)=x
	,\ v(x,y)=0
\]
\[
	f(z)=Imz,\ z=x+iy\implies f(z)=y
\]
\[
	f(z)=|z|,\ f(z)=\sqrt{x^2+y^2} \implies u(x,y)=\sqrt{x^2+y^2} , \ v(x,y)=0
\]
\[
	f(z)=P(z)=a_nz^n+\ldots+a_1z+a_0
\]
In questo ultimo caso è necessario calcolare manualmente le funzioni $u,v$ associate
\[
	f(z)=\frac{P(z)}{Q(z)} \text{ con } P,Q \text{ funzioni polinomiali }
\]
Quest'ultima funzione non è definita $\forall z \in \mathbb{C}$
\[
	f(z)=e^z, \sin z, \cos z, \sinh z, \cosh z
\]
\textbf{Funzione Esponenziale}
\\È l'estensione della funzione esponenziale nel campo dei reali
\[
	z=x+iy\implies e^z:=e^x\cdot e^{iy}:=e^x\cdot(\cos y+i\sin y)
\]
Se $z=x\in R \implies e^z=e^x\implies$ è estensione della funzione reale
\\$u(x,y)=e^x \cos y,\ v(x,y)=e^x\sin y $
\[ e^{z+2\pi i}=e ^{x+iy+2\pi i}=e ^{x + i(y +2 \pi }=e^x(\cos (y+2 \pi)+i \sin (y+2 \pi))=e^z\]
Questo implica che la funzione esponenziale nel campo complesso è periodica di periodo $T=2 \pi i$
Inoltre,
\[e^{z_1+z_2}=e^{z_1}e^{z_2}
\]
\[
	e^z=0\iff|e^z|=0
\]
\[	
	|e^x(\cos y + i \sin y)|=|e^x| |\cos y +i\sin y|=e^x=0\ \not\forall z  
\]
\textbf{Funzioni coseno, seno}
\[\cos z:=\frac{1}{2}(e^{iz}+e^{-iz})\]
\[\sin z:=\frac{1}{2i}(e^{iz}-e^{-iz})\]
Sono funzioni definite $\forall z\in \mathbb C$
\\Come per la funzione esponenziale sono estensioni delle funzioni reali
\\Dunque se $z=x\in R \implies \cos z = \cos x$
\[\cos(z+2\pi)=\cos z\]
$\implies \cos z$ è periodica sia nel campo dei reali sia nel campo dei complessi
\\Inoltre
\[\cos z =0\iff z=\frac{\pi}{2}+k\pi,k\in Z\]
Per le altre funzioni valgono proprietà analoghe essendo definite come estensioni
\\\textbf{Formule alternative}


\section{Limiti}
\textbf{Definizione: } \[f:\Omega\subseteq\C\to \C, z_0\in \text{acc}(\Omega),l\in\C\]
\[\lim_{z \to z_0} f(z)=l\iff\forall V(l)\exists
u(z_0) \text{ tc }\forall z\in (u(z_0) \cap \Omega\setminus\{z_0\}), f(z)\in V(l)\]
\textbf{Definizione di continuità}
\[f:\Omega\subseteq\C\to \C, z_0\in a(\Omega)\cap \Omega,\  f \text{ continua } z_0 \iff\exists \lim_{z \to z_0} f(z)=f(z_0)\]
\textbf{Osservazioni}
\\$z=x+iy, z_0=x_0+iy_0, l=l_1+il_2, f=u+iv$
\[\lim_{z \to z_0} f(z)=l\iff 
\begin{cases}	
	\lim_{(x,y) \to (x_0,y_0)} u(x,y)=l_1\\ 
	\lim_{(x,y) \to (x_0,y_0)} v(x,y)=l_2
\end{cases}
\]
Con le stesse notazioni
\[f \text{ continua in }z_0\iff u,v \text{ continue in } (x_0,y_0)\]
Sono continue (sul loro dominio di definizione) tutte le funzioni elementari introdotte nella lezione scorsa
\\Vale l'algebra dei limti e il teorema del limite della funzione composta ($\implies$ composizione di continue rimane continua)
\\Vale il teorema di unicità del limite
\subsubsection{Infinito nei complessi}
Un intorno di $\infty$ nei complessi è il complementare di un qualsiasi disco
\[z\to \infty \iff z\in u(\infty)\iff |z|>R\iff |z|\to +\infty\]
\[f(z)\to \infty \iff f(z)\in u(\infty) \iff |f(z)|>R \iff |f(z)|\to +\infty\]

\subsection{Derivabilità}
\textbf{Definizione: }
\[f:\Omega \subseteq \C\to \C, z_0\in \text{acc}(\Omega)\cap \Omega\]
\[f \text{ derivabile (in senso complesso) in } z_0 \iff\exists \lim_{z \to z_0} \frac{f(z)-f(z_0)}{z-z_0}(\in\C)\]
e tale limite si dice $f'(z_0)$ 
\\Definizioni alternative:
\[\exists \lim_{h \to 0} \frac{f(z_0+h)-f(z_0)}{h}\ldots\]
\[f(z_0+h)=f(z_0)+\lambda\cdot h + o(h)\ \ \ \frac{o(h)}{h}\to 0\]
Dove $\lambda$ è la derivata prima della funzione nel punto $z_0$
\\Quest'ultima è la definizione di differenziabilità
\\\textbf{Attenzione} Se $u,v$ sono differenziabili ciò non implica la differenziabilità/derivabilità di $f$ (Un esempio è $f(z)=Imz$)
\begin{tcolorbox}
\textbf{Teorema (caratterizzazione della derivaibilità)}
\[f:\Omega \subseteq \C\to \C, z_0\in \Omega \cap \text{acc}(\Omega),\ z_0=x_0+iy_0,\ f=u+iv\]
\[f \text{ derivabile in }z_0\iff u,v \text{ differenziabile in } (x_0,y_0),\]
\[\text{Inoltre } \begin{cases}	
u_x(x_0,y_0)=v_y(x_0,y_0)
\\u_y(x_0,y_0)=-v_x(x_0,y_0)
\end{cases}
\]
(sistema, condizione di cauchy riemann)
Inoltre, in tal caso 
\[f'(z_0)=u_x(x_0,y_0)-iu_y(x_0,y_0)\]
\[f'(z_0)=v_y(x_0,y_0)+iv_x(x_0,y_0)\]

\end{tcolorbox}
\textbf{Dimostrazione}
\[(\implies) \text{Per Hp } \exists f'(z_0)=\alpha +i\beta\in\C\implies f(z_0+h)=f(z_0)+f'(z_0)h+g,\ g=o(h)\]
\[
	u(x_0+h_1,y_0+h_1)+iv(x_0+h_1,y_0+h_2)=u(x_0,y_0)+iv(x_0,y_0)+(\alpha+i\beta)(h_1+ih_2)+g_1+ig_2
\]
\[u(x_0+h_1+h_1,y_0+h_2)=u(x_0,y_0)+(\alpha h_1-\beta h_2)+g_1\]
\[v(x_0+h_1,y_0+h_2)=v(x_0,y_0)+(\beta h_1+\alpha h_2)+g_2\]
Queste due equazioni indicanon che $u,v$ sono differenziabili in $(x_0,y_0)$, con 
\[
	\begin{cases}
\nabla u(x_0,y_0)=(\alpha, -\beta)\\ 
\nabla v(x_0,y_0)=(\beta, \alpha)
	\end{cases}
\]
Questo dimostra inoltre che $f'(z_0)=\alpha+i\beta$
\\$(\impliedby)$ Procedere al contrario













