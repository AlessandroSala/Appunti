
\section{Problemi di Cauchy per l'equazione del calore e delle onde}
Si tratta di problemi di evoluzione, ovvero l'incognita è $u=u(x_1,\ldots,x_{n},t)$ 
\[u_t-\Delta u = f \text{ equazione del calore (parabolica)}\]
\[u_{t t}-\Delta u = f \text{ equazione delle onde (iperbolica)}\]
Ad entrambe è associata l'equazione di Poisson $-\Delta u = f $ (ellittica)
\\In dimensione $n=2$, le equazioni omogenee diventano
\[u_t-u_{x x}=0\]
\[u_{t t}-u_{x x}=0\]
\[u_{x_1x_1}+u_{x_2x_2}=0\]
\begin{align*}
&A=\begin{pmatrix} 0&0\\0&-1 \end{pmatrix} \begin{pmatrix} u_{t t}&u_{tx}\\u_{x t}&u_{x x} \end{pmatrix} &\text{ det}A = 0\\
&A= \begin{pmatrix} 1&0\\0&-1 \end{pmatrix} &\text{ det}A<0\\
&A= \begin{pmatrix} 1 & 0\\0 & 1 \end{pmatrix} &\text{ det}A>0
\end{align*}
Si ha una terminologia simile alla classificazione delle coniche
\subsection{Problema di Cauchy per l'equazione del calore}
\[\begin{cases}
	u_t-u_{x x}=0\ \ x\in \R,\ t>0\\
	u(x,0)=u_0(x)\ \ x\in \R
\end{cases}
 \]
\textbf{Ipotesi:} $u_0\in L^{2}(\R)$ 
\\\textbf{Strategia:} Applicare $\mathcal F_{(x\to \xi)}$ (Fourier rispetto a $x$)
\[\hat{u}(\xi, t)=\mathcal F_{x\to \xi}(u(x,t))= \int_{\R}^{} u(x,t) e^{-i\xi x}dx\]
\begin{enumerate}
	\item Trasformare il problema
		\[\widehat{u_t}- \widehat{u_{x x}}=0\]
		\[\hat{u_t}=(\hat{u})_t\text{ purché }u\text{ sia sufficientemente regolare}\]
		\[\frac{\partial }{\partial t} \bigg( \int_{\R}^{} u(x,t)e^{-i\xi x}dx\bigg)= \int_{\R}^{} \frac{\partial }{\partial t} u(x,t)e^{-i\xi x}dx\]
		\[\widehat{u _{x x}}=(i\xi )^2 \hat{u}=-\xi^2 \hat{u}\]
		L'equazione soddisfatta da $\hat{u}$ è
		\[(\hat{u})_t + \xi^2 \hat{u}=0\]
		Il dato iniziale trasformato è 
		\[\hat{u}(\xi,0)=\int_{\R}^{} u_0 e^{-i\xi x}dx\]
Il problema risolto da $\hat{u}$ è
\[\begin{cases}
	(\hat{u})_t+\xi^2 \hat{u}=0\\
\hat{u}(\xi,0)=\hat{u}_0(\xi)

\end{cases}
\]
Ovvero una ODE, lineare, omogenea, del I ordine.
\item Determinare $\hat{u}$ 
	\\Se poniamo $\varphi(t):=\hat{u}(\xi, t)$ (per $\xi$ fissato), il problema risolto da $\varphi $ :
	\[\begin{cases}
		\varphi'(t)=-\xi^2 \varphi(t)\\
		\varphi(0)=\hat{u}_0(\xi)
	\end{cases}\implies \varphi(t)=C e^{-\xi^2 t}\]
	\[\varphi(0)=C=\hat{u}_0(\xi)\implies \varphi(t)=\hat{u}_0(\xi)e^{-\xi^2t}\]
\item Determinare $u$ a partire da $\hat{u}$ 
	\[u_(x,t)=\mathcal F^{-1}_{x\to \xi}(\hat{u}_0(\xi))*\mathcal F^{-1}_{x\to \xi}(e^{-\xi^2t})\]
	\[u_1(x,t)*_x u_2(x,t)= \int_{\R}^{} u_1(x-y,t)u(y,t)dy\] 
	
\end{enumerate}
\begin{tcolorbox}
	\[u(x,t)=u_0(x)*\frac{1}{\sqrt{4\pi t} }e^{- \frac{x^2}{4t}}\]
\end{tcolorbox}
\textbf{Commenti} 
\begin{itemize}
	\item Avendo preso $u_0 \in L^{2}(\R)$, il prodotto di convoluzione ha senso ($u_0\in L^2,$ gaussiana $\in L^1$)
	\item Se il prodotto scritto  ha senso per $t>0$, si può far vedere che:
		\[\lim_{t \to 0} u(x,t)=u_0(x)\text{ in }L^{2}(\R)\]
		Infatti
		\[ \|u(x,t)-u_0(x)\|^2_{L^2}=(2\pi)^{-1} \|\hat{u}(\xi,t)-\hat{u}_0(\xi)\|^2_{L^2}=(2\pi)^{-1} \int_{\R}^{} |\hat{u}(\xi, t)-\hat{u}_0(\xi)|^2d\xi\]
		\[=(2\pi)^{-1}\int_{\R}^{} |\hat{u}_0(\xi) e^{-\xi^2 t}-\hat{u}_0(\xi)|^2d\xi=\]
		\[(2\pi)^{-1}\int_{\R}^{} |\hat{u}_0(\xi)|^2 |e^{-\xi^2 t}-1|^2d\xi \xrightarrow[t\to 0]{}0\]
		Per Plancherel. Inoltre il limite per $t\to 0$ passa sotto integrale per convergenza dominata.
\end{itemize}
\subsection{Effetto regolarizzante dell'operatore calore}
$\ \forall t>0,x\mapsto u(x,t)$ è di classe $C^\infty$ per le proprietà del prod. di convoluzione.
\\La formula mostra anche quella che si dice \emph{velocità di propagazione infinita} dell'operatore $\partial_t-\partial_{x x}$ 
\\Ovvero: anche se $u_0(x)$ è a supporto compatto 
\[\ \forall t>0,x\mapsto u(x,t)\text{ è diversa da 0 }\ \forall x\in \R\]
\subsection{Problema di Cauchy per l'equazione delle onde}
\[\begin{cases}
	u_{t t}-u_{x x}=0\ \ x\in \R,\ t>0\\
	u(x,0)=u_0(x)\ \ x\in \R
	\\u_t(x,0)=u_1(x)\ \ x\in \R
\end{cases}\]
\textbf{Ipotesi:} $u_0\in C^2(\R)\cap L^{1}(\R),u_1\in C^1(\R)\cap L^{1}(\R)$
\\Spezzando (P) in due problemi distinti
\[(P_1) \begin{cases}
	v_{t t}-v_{x x}=0\\
	v(x,0)=u_0(x)\\
	v_t(x,0)=0
\end{cases}
(P_2) \begin{cases}
	w_{t t}-w_{x x}=0\\
	w(x,0)=0\\
	w_t(x,0)=u_1(x)
\end{cases}\]
Se trovo $v$ e $w$ soluzioni dei due problemi $\implies u=v+w$ soluzioni di (P)
\\Risolvo $(P_1)$ 
\begin{enumerate}
	\item $\widehat{v_{t t}}-\widehat{v_{x x}}=0$, se $v$ sufficientemente regolare
		\[(\hat{v})_{t t}=-\xi^2 \hat{v}\]
		\[\hat{v}(\xi,0)=\hat{u}_0(\xi)\]
		\[\hat{v}_t(\xi,0)=0\]
		Problema di Cauchy per un'ODE (lineare, omogenea del II ordine)
		Completare
		\[\implies \varphi(t)=C_1e^{i\xi t}+C_2e^{-i\xi t}\]
		Imponiamo i dati iniziali 
		\[\varphi(0)=C_1+C_2=\hat{u}_0(\xi)\]
		\[\varphi'(0)|_{t=0}=i\xi C_1e^{(i\xi t)}-i\xi C_2e^{-i\xi t}|_{t=0}=i\xi C_1-i\xi C_2=0 \iff C_1=C_2\]
		Quindi $C_1=C_2=\frac{1}{2}\hat{u}_0(\xi)$ 
		\[\implies \varphi(t)=\hat{v}(\xi,t)=\frac{1}{2}\hat{u}_0(\xi)e^{i\xi t}+\frac{1}{2}\hat{u}_0(\xi)e^{-i\xi t}\]
	\item Ricavo $v$ a partire da $\hat{v}$, ricordando la regola di trasformazione (traslazione):
		\[v(x,t)=\frac{1}{2}[u_0(x+t)+u_0(x-t)]\]
		Somma di due onde, (progressiva e regressiva) che hanno stessa forma di $u_0$ e metà ampiezza
		Procedendo in maniera analoga:
		\[w(x,t)=\frac{1}{2}(U_1(x+t)-U_1(x-t))=\frac{1}{2}\int_{x-t}^{x+t} u_1(y)dy\]
		Conclusione: \textbf{Formula di d'alembert}
		\[u(x,t)=\frac{1}{2}[u_0(x+t)+u_0(x-t)]+\frac{1}{2} \int_{x-t}^{x+t} u_1(y)dy\]


\end{enumerate}
\textbf{Commenti} 
\begin{itemize}
	\item Non c'è effetto regolarizzante, nelle ipotesi sopra $u$ è di classe $C^2(\R)$.
		\\Si può dimostrare che D'Alembert vale più in generale.
	\item La velocità di propagazione è finita


\end{itemize}


