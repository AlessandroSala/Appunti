
\section{Serie di Fourier in $L^2$}
Si considera $L^{2}(I)$ dove $I=(-\frac{T}{2},\frac{T}{2})$, e il prodotto scalare è definito come
\[(f,g)=\int_{I}^{} f(x)g(x)dx\]
\textbf{Sistema ortonormale dei polinomi trigonometrici}
\[p_0=\frac{1}{\sqrt{T} },\ p_k=\cos\frac{\xi_kx}{\sqrt{\frac{T}{2}} },\ q_k= \frac{\sin(\xi_kx}{\sqrt{\frac{T}{2}} },\ k\ge 1\]
dove 
\[\xi_k=\bigg(\frac{2\pi}{T}\bigg)^k\ \ (\text{se }T=2\pi,\ \xi_k=k)\]
formano un sistema ortonormale
\\Dunque, data $f\in L^{2}(I)$ la sua serie di Fourier rispetto a questo sistema è la serie
\[(f,p_0)p_0+\sum_{k\ge 1}^{} (f,p_k)p_k+(f,q_k)q_k\ \ (*)\]
\subsubsection{Modi equivalenti di scrivere (*)}
\[\frac{a_0}{2}+\sum_{k\ge 1}^{} a_k\cos(\xi_kx)+b_k\sin(\xi_kx)\ \ (* *)\]
\[a_k=\frac{2}{T}\int_{I}^{} f(x)\cos(\xi_kx)dx\ \ k\ge 0\]
\[b_k=\frac{2}{T}\int_{I}^{} f(x)\sin(\xi_kx)dx\ \ k\ge 1\]
Oppure 
\[\sum_{k=-\infty}^{+\infty} \hat{f_k}e^{i\xi_kx}\ \ (* * *)\]
\[\hat{f_k}=\frac{1}{T}\int_{-\frac{T}{2}}^{\frac{T}{2}} f(x)e^{-i\xi_kx}dx\ \ z\in \Z\]
\[\hat{f_k}= \frac{a_k-ib_k}{2}\ \ k\in \N\]
\[\hat{f_{-k}}=\frac{a_k+ibk}{2} \]
\begin{tcolorbox}
	\textbf{Teorema: }Il sistema $(p_0,p_k,q_k)$ è ortonormale completo in $L^{2}(I)$
\end{tcolorbox}
\textbf{Dimostrazione} usando la (3) delle equivalenze
\\$\overline{\left< p_0,p_k,q_k \right> }\equiv L^{2}(I)$ 
\\$\subseteq  $ sempre vero $(\overline{M}\subseteq  H)$ 
\\$\supseteq \ \forall f\in L^{2}(I),\ f$ può essere approssimata con elementi di $\left< p_0,p_k,q_k \right> $
\\\textbf{Passo 1} Mostriamo che quanto sopra è vero se $f=\varphi\in C_0^\infty(I)$.
\\Infatti, posso estendere $\varphi$ a $\tilde\varphi\in C_{\text{per}}^\infty(\R)$.
\\Sappiamo (Analisi 2) che la serie di Fourier di $\tilde\varphi$ converge uniformemente su $[-\frac{T}{2},\frac{T}{2}]\implies $converge uniformemente in $L^{2}(-\frac{T}{2},\frac{T}{2})$.
\\$\implies $ La successione delle somme parziali della serie di Fourier di $\tilde\varphi$ fornisce una successione in $\left< p_0,p_k,q_k \right> $, che converge a $\tilde\varphi=\varphi$ in $L^{2}(-\frac{T}{2},\frac{T}{2})$
\\\textbf{Passo 2} Data $f\in L^{2}(I)$ posso approssimarla con una successione $\varphi_k \subset C_0^\infty(I)$.
\[\varphi_k\xrightarrow{L^{2}(I)}f\]
Concludo prendendo una successione diagonale.
\\Oppure:
\[\|f-S\|\le \|f-\varphi\|+\|\varphi-S\|<\varepsilon\]
$S$ somma parziale della serie di Fourier di $\varphi$ 
\\\textbf{Osservazioni}
\begin{itemize}
	\item Data $f\in L^{2}(I)$, per il teorema sopra:
		\[f=(f,p_0)p_0+\sum_{k\ge 1}^{} (f,p_k)p_k+(f,q_k)q_k\]
	\item Data $f\in L^{2}(I)$, vale l'id. di Bessel:
		\[\|f\|_{L^{2}(I)}^2=(f,p_0)^2+\sum_{k\ge 1}^{} (f,p_k)^2+(f,q_k)^2\]
		Dunque $f\in L^{2}(I) \iff $ i suoi coefficienti di Fourier $\in \ell^2$ 
	\item Possiamo sostituire $L^{2}(-\frac{T}{2},\frac{T}{2})$ con 
		\[ L^{2}_T(\R)=\{f\in L_{\text{loc}}^2(\R):\text{T-periodiche}\} \]
		è uno spazio di Hilbert, con $(f,g)=\int_{-T / 2}^{T / 2} fg $ 
	\item I coefficienti di Fourier hanno senso anche per $f\in L^{1}(-\frac{T}{2},\frac{T}{2})$ 
		\[|a_k|=\frac{2}{T}\bigg|\int_{-\frac{T}{2}}^{\frac{T}{2}}f(x)\cos(\xi_kx)dx\bigg|\le \frac{2}{T } \int_{-\frac{T}{2}}^{\frac{T}{2}} |f(x)|<+\infty\]
		lo stesso per i $b_k$ 
	\item Se $f\in \text{A.C.}([-\frac{T}{2},\frac{T}{2}])$ e $f(-\frac{T}{2})=f(\frac{T}{2})$ posso estenderla a una funzione continua periodica su $\R$. (l'estensione $f$ appartiene a $L_T^2(\R)$)
		\\Ha senso calcolare i coefficienti di Fourier sia di $f$ che di $f'$.
		\[a_k,b_k\text{ siano i coefficienti di Fourier di }f\]
		\[a_k',b_k'\text{ siano i coefficienti di Fourier di }f'\]
\[a_k'=\frac{2}{T} \int_{-\frac{T}{2}}^{\frac{T}{2}} f'(x)\cos(\xi_kx)dx=0+\frac{2}{T}\xi_k \int_{-\frac{T}{2}}^{\frac{T}{2}} f(x)\sin(\xi_kx)dx\]
\[=\xi_kb_k\]
Analogamente
\[b_k'=-\xi_ka_k\]
Dunque
\[\hat{f_k'}=i\xi_k \hat{f_k}\ \forall k\in \Z\]

\end{itemize}
\section{Applicazioni delle serie di Fourier alle equazioni differenziali}
Ricerca di soluzioni periodiche di ODE lineari (tramite serie di Fourier)
\\Consideriamo un'ODE su $\R$ della forma
\[\sum_{j=0}^{n} a_ju^{(j)}=f\in L^{2}_T(\R)\ \ x\in \R\]
\textbf{Problema:} esistono soluzioni T-periodiche (in $L^{2}_T(\R)$)?
\\La seguente equazione differenziale equivale a chiedere:
\[\sum_{j=0}^{n} a_j\widehat{u^{(j)}}_k=\hat{f}_k\ \forall k\in \Z\]
\[\sum_{j=0}^{n} a_j(i\xi_k)^{(j)}\hat{u_k}=\hat{f}_k\ \forall k\in \Z\]
Sistema di infinite eq. algebriche.
\[P(i\xi_k)\hat{u_k}=\hat{f_k}\]
(forma equivalente)
\\Ciascuna eq. è un'equazione lineare di 1° grado in $\hat{u_k}$!!!
\begin{itemize}
	\item Caso 1: $P(i\xi_k)\neq 0\ \forall k\in \Z$ 
		\[\implies \hat{u_k}=\frac{\hat{f_k}}{P(i\xi_k)}\]
	\item Caso 2: $P(i\xi_k)=0$ per $k=k_1^*,\ldots,k_p^*$, allora
		\\$\hat{f_k}=0$ per $k=k_1^*,\ldots,k_p^*\implies $infinite soluzioni, ovvero
		\[\hat{u_k}=\begin{cases}
		\frac{\hat{f_k}}{P(i\xi_k)} k\neq k_1^*,\ldots,k_p^*
	\\\text{arbitrario}\ \ k=k_1^*,\ldots,k_p^*

\end{cases}\]
Se però $\hat{f}_k\neq 0$ per qualche $k\in \{k_1^*,\ldots,k^*_p\} \implies $ no soluzioni
\end{itemize}
\textbf{Osservazione:} Ci sono altri sistemi ortonormali completi in $L^{2}(I)$.
\[1,x,x^2,xì3,\ldots\]
$\to $ polinomi di Legendre\\
\textbf{Esempio} in $L^2(-1,1)$ 
\[\frac{1}{\sqrt{2} },, \sqrt{\frac{3}{2}} x, \sqrt{\frac{5}{2}} \bigg(-\frac{1}{3}+x^2\bigg)\]
Data $f\in L^{2}(I)$, per minimizzare la distanza in $L^{2}(I)$ da un polinomio di grado $\le 3$, dovrò considerare la somma delle serie di Fourier di  $f$ fatta rispetto ai polinomio di Legendre
