
\section{Analisi funzionale}
Uno \textbf{spazio vettoriale} (su $\R$) è un insieme (V) su cui sono definite due operazioni:
\\\textbf{Somma} $+:V \times V\to V$
\\\textbf{Prodotto per scalare} $\cdot :\R \times \to V$
\\Tali operazioni godono delle seguenti proprietà
\\Per la somma
\begin{itemize}
	\item $u+v=v+u$ 
	\item $u+(v+w)=(u+v)+w$
	\item $u+\underline 0=u$
	\item $u+(-u)=\underline 0$
\end{itemize}
Per il prodotto per scalare
\begin{itemize}
	\item $(ts)u=t(su)$
	\item $t(u+v)=tu+tv$
	\item $(t+s)u=tu+su$
	\item $1\cdot u=u$
\end{itemize}
\subsubsection{Norma}
\begin{tcolorbox}
	Sia $V$ uno spazio vettoriale su $\R$.
	\\Una norma su $V$ è una funzione $\|.\|:V\to \R^+$ tale che:
	\begin{itemize}
		\item $\|v\|>0\forall v\in V-\{\underline 0\} $ (positività)
		\item $\|tv\|=|t|\|v\|\forall t\in\R,\forall v\in V$ (omogeneità)
		\item $\|u+v\|\le \|u\|+\|v\|\forall u,v\in V$ (dis. triangolare)
	\end{itemize}
\end{tcolorbox}
\begin{tcolorbox}
	$(V,\|.\|)$ si dice \textbf{spazio vettoriale normato}.
\end{tcolorbox}
Seguono le seguenti proprietà (resettare counter)
\begin{enumerate}
	\item $\|\underline 0\|=0$
	\item $| \|u\|-\|v\| |\le \|u-v\|\forall u,v\in V (dim)$
\end{enumerate}
Norma euclidea:
\[\|x\|_2=\sqrt{x_1^2+x_2^2+\ldots+x_n^2} =\sum_{i=1}^{n} (x_i^2)^\frac{1}{2}\]
\[\|x\|_\infty=\max_{i=1,\ldots,n}|x_i|\]
\[\|x\|_p=(\sum_{i=1}^{n}(|x|^p)^{\frac{1}{p}}\]
La disuguaglianza triangolare per la norma $p$, ovvero disuguaglianza di Minkowski, richiede l'ipotesi $p\ge 1$.
\subsubsection{Norma su uno spazio funzionale di dimensione infinita}
$V=C^0([a,b])$
\[\|f\|_\infty:=\text{max}_{x\in[a,b]}|f(x)|\]
\[\|f\|_1:=\int_{a}^{b} |f(x)|dx\]
\begin{tcolorbox}	
\textbf{Definizione:} Sia $(V,\|.\|)$ sp. vettoriale normato. Allora $d(u,v):=\|\underline{u}-\underline{v}\|$ definisce una distanza su $V$, ovvero
\[d:V\times V\mapsto \R\]
tale che
\begin{enumerate}
	\item $d(u,v)\ge 0$ con $=0\iff u=v$ positività
	\item $d(u,v)=d(v,u)$ simmetria
	\item $d(u,v)\le d(u,w)+d(w,v)$ disuguaglianza triangolare
\end{enumerate}
\end{tcolorbox}
\[d_p(x,y)=\|x-y\|_p=(\sum_{i=1}^{n} \|x_i-y_i\|^p)^{\frac{1}{p}}\]
\[d_\infty(f,g)=\max_{x\in[a,b]}|f(x)-g(x)|\]
\[d_1(f,g)=\int_{a}^{b} |f(x)-g(x)|\]
\begin{tcolorbox}
	\textbf{Definizione:} $(V,d)$ si dice spazio metrico
\end{tcolorbox}
\textbf{Osservazione:} In uno spazio metrico possiamo definire le \textbf{sfere}: dato $r\ge 0,v_0\in V$
\[B_r(v_0):=\{v\in V:d(v,v_0)<r\} \]
Si dice sfera chiusa se la disuguaglianza non è stretta.\\
\begin{tcolorbox}	
\textbf{Definizione:} Se $\{v_n\} \subseteq  (V,\|.\|)$, si dice che $v_n\to v$ in $V$ se
\[d(v_n,v)\to 0\text{, oppure }\|v_n-v\|\to 0\]
\[(\forall \varepsilon>0\exists \nu\in\N:d(v_n,v)=\|v_n-v\|<\varepsilon\forall n\ge \nu)\]
\end{tcolorbox}
Alcuni fatti veri in dimensione finita ma falsi in dimensione infinita:
\begin{enumerate}
	\item Tutte le norme sono "equivalenti" fra loro
	\item Tutte le successioni di Cauchy convergono
	\item Tutti i sottospazi vettoriali sono chiusi
\end{enumerate}
1) Norme equivalenti\\
\textbf{Definizione:} Sia $V$ uno spazio vettoriale (su $\R$), consideriamo su $V$ due possibili norme $ \|.\|,|\|.\||$.
\\Queste due norme si dicono \textbf{equivalenti} se:
\begin{enumerate}
	\item $\exists c>0:\forall v\in V,\|v\|\le c|\|v\||$
	\item $\exists c'>0:\forall v\in V,|\|v\||\le c\|v\|$
\end{enumerate}
Le successioni convergenti nelle due norme sono le stesse.
\\Interpretazione geometrica:
\[\|.\|_\infty\le \|.\|_1\iff B_1^1(0)\subseteq  B_1^\infty(0)\]
\begin{tcolorbox}
	\textbf{Teorema:} Se $\text{dim}V<+\infty\implies$ tutte le norme su $V$ sono tra loro equivalenti
\end{tcolorbox}
In uno spazio a dimensione infinita non è generalmente vero, ad esempio nello spazio delle funzioni continue su $[a,b]$
\begin{tcolorbox}
	\textbf{Definizione:} Sia $(V,\|.\|)$ uno spazio vettoriale normato. Una successione $\{v_n\} \subseteq V$ si dice successione di cauchy se 
	\[\forall \varepsilon>0 \exists \nu\in\N:d(v_n,v_m)<\varepsilon\ \forall n,m\ge \nu\]
\end{tcolorbox}
\textbf{Osservazione:} Vale sempre che se $\{v_n\} $converge allora è di Cauchy.
\[(\|v_n-v_m\|=\|v_n-v+v-v_m\|\le \|v_n-v\|+\|v_m-v\|<\frac{\varepsilon}{2}+\frac{\varepsilon}{2}=\varepsilon\]
\begin{tcolorbox}
	\textbf{Teorema:} Se $\text{dim}V<+\infty$ vale anche il viceversa, ovvero
	\[\{v_n\} \text{ converge }\iff \{v_n\} \text{ di Cauchy}\]
\end{tcolorbox}
Questo teorema è falso se $\text{dim}V=+\infty$
\\Prendendo lo spazio delle funzioni $V=C^0([a,b])$ con norma 1, si può costruire una successione di Cauchy che non converge.
\[f_n(x)=\begin{cases}
	-1\ \ \ x\le -\frac{1}{n}
	\\\frac{1}{n}\ \ \ -\frac{1}{n}<x<\frac{1}{n}
	\\1\ \ \ x\ge \frac{1}{n}
\end{cases}
\]
\[\|f_n-f_m\|_1<\varepsilon,\ \int_{a}^{b} |f_n-f_m|=\int_{a}^{b} f_m-f_n\]
\subsection{Spazio di Banach}
\begin{tcolorbox}
	\textbf{Definizione:} Uno spazio vettoriale normato $(V,\|.\|)$ si dice completo o di Banach se tutte le successioni di Cauchy convergono.
\end{tcolorbox}
\textbf{Teorema/osservazione: }$V=(C^0([a,b],\|.\|_\infty)$ è di Banach
\\Generalizzazione:
\[V=C^k([a,b])\ \ \|f\|_{C^k}:=\sum_{|\alpha|\le k}^{} \|D^\alpha f\|_\infty\]
\begin{tcolorbox}
	\textbf{Teorema} Sia $(V,\|.\|)$ spazio vettoriale normato. Se $\text{dim}V<+\infty\implies$ tutti i sottospazi vettoriali $W$ sono \textbf{chiusi}.
	\[\{v_n\} \subseteq  W,v_n\to v\text{ in }V\implies v\in W\]
\end{tcolorbox}
Il teorema diventa falso se $\text{dim}V=+\infty$, ad esempio $V=(C^0([a,b]), \|.\|_\infty)$.
