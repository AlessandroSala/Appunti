
\section{Funzioni olomorfe}
\begin{tcolorbox}
	\textbf{Definizione} \\
	$f$ si dice olomorfa su $\Omega$ se è derivabile in $z_0\forall z_0\in\Omega$
\end{tcolorbox}
\subsection{Invertibilità locale}
\begin{tcolorbox}
\textbf{Teorema}
\\Sia $f:\Omega \subseteq\C\to \C$ olomorfa in $\Omega$, e sia $z_0\in\Omega$ tale che $f'(z_0)\ne0$ allora f è "localmente invertibile in $z_0$"\\
($\exists u(z_0)$ tale che $f|_{u(z_0)}$ invertibile)
\\E la funzione inversa $f^{-1}$ è derivabile in senso complesso in $f(z_0)$ e 
\[(f^{-1})'|_{z_0}=\frac{1}{f'(z_0)}\]	
\end{tcolorbox}
\textbf{Dimostrazione}\\
$\Phi(u,v)$ definito su $\Omega \subseteq \R^2\to \R^2, (x_0,y_0)\in\Omega$, se $detJ\Phi(x_0,y_0)\ne 0\implies\Phi$ "localmente invertibile" e 
\[J\Phi^{-1}(\Phi(x_0,y_0))=(J\Phi(x_0,y_0))^{-1}\]
Dunque se $f=u+iv$ si riformula il teorema con $\Phi=(u,v)$
\[J\Phi(x_0,y_0)=\begin{pmatrix}
	u_x&u_y\\v_x & v_y
\end{pmatrix}
=\begin{pmatrix}
	\alpha & -\beta \\ \beta & \alpha
\end{pmatrix}
\implies detJ\Phi(x_0,y_0)=\alpha^2+\beta^2=|f'(z_0)|^{2}
\]
Poiché $f'=\alpha +i\beta$ e l'ipotesi del teorema è che $|f'(z_0)|^2\ne 0$
\[J\Phi^{-1}(\Phi(x_0,y_0))=\frac{1}{\alpha^2+\beta^2}\begin{pmatrix}
	\alpha&\beta\\-\beta &\alpha
\end{pmatrix}
\]
\[\implies(f^{-1})'(f(z_0))=\frac{\alpha}{\alpha^2+\beta^2}-i\frac{\beta}{\alpha^2+\beta^2}= \frac{\overline{f'(z_0)}}{|f'(z_0)|^2}=\frac{1}{f'(z_0)}\]
\\
\subsection{Ricerca delle primitive - antiderivazione}
\textbf{Problema: }Data $f:\Omega \subseteq \C\to \C$ esiste? unica? $F:\Omega\subseteq\C$ olomorfa in $\Omega$ tale che \[F'(z)=f(z)\]
Una tale $F$ si dice \textbf{primitiva} di $f$.\\
\\\textbf{Richiamo - Teorema fondamentale del calcolo: }Data $f:(a,b)\in\R\to \R$ continua, allora una primitiva di $f$ è data da \[F(x)=\int_{a}^{x} f\]
E tutte le altre primitive si ottengono aggiungendo una costante reale

\noindent\rule{\textwidth}{0.5pt}
\textbf{Unicità: }una primitiva, se esiste, è univocamente determinata a meno di costante additiva.
\begin{itemize}
	\item $F$ primitiva di $f$, $\lambda\in\C\implies F+\lambda \text{ primitiva di }f$ poiché $(F+\lambda)'=F'+\lambda'=f$
	\item $F_1,F_2$ primitive di f $\implies\exists \lambda\in\C:F_1-F_2=\lambda$
\end{itemize}
$G:=F_1-F_2$, Tesi: $G$ è costante, Dim: \[G'=(F_1-F_2)'=f-f=0\]
$G=u+iv$ $G'=u_x-iu_y=v_y+iv_x$ $G'=0\implies \nabla u(x_0,y_0)=\nabla v(x_0,y_0)=\underline 0$
\\$\implies u \text{ costante},\ v \text{ costante}$
\\N.B vale se $\Omega$ è connesso 

\noindent\rule{\textwidth}{0.5pt}
\textbf{Esistenza}
\\$f=u+iv$, $F=U+iV$ ( $f$ data, $F$ incognita )
\\$F'=U_x-iU_y=V_y+iV_x=f=u+iv$
\[
	\implies \begin{cases}
	U_x=u\\U_y=-v
\end{cases}
	\begin{cases}
	V_x=v\\V_y=u
\end{cases}
\]
ovvero $U$ potenziale per $w_1:=udx-vdy$
\\e $V$ potenziale per $w_2:=vdx+udy$
\\Concludiamo che dire $f$ ammette primitive $\iff\omega_1,\omega_2$ esatte $\implies\omega_1,\omega_2$ chiuse
\\Ovvero se la funzione $f$ soddisfa le condizioni di Cauchy-Riemann dunque se $f$ olomorfa
\[f\text{ ammette primitive }\iff\omega_i \text{ esatte }\implies f \text{ olomorfa }\iff w_i\text{ chiuse }\]
L'implicazione inversa è vera se $\Omega$ è semplicemente connesso
\\Dunque $F=U+iV$ dove $U$ potenziale per $\omega_1$, $V$ potenziale per $\omega_2$
\\\textbf{Nota: } $\omega$ chiusa $\implies \oint_\gamma \omega$ non cambia se sostituisco $\gamma$ con un circuito omotopo. 
\begin{tcolorbox}
	\textbf{Definizione} 
	\\Data $f:\Omega\subseteq\C\to \C$, dato $\gamma$ cammino in $\Omega$ parametrizzata da una funzione $r:[a,b]\to \Omega$, $r(t)=r_1(t)+ir_2(t)$
	\[\int_{\gamma}^{} f(z)dz:=\int_{a}^{b} f(r(t))r'(t)dt\]

\end{tcolorbox}
	\[=\int_{a}^{b} (u+iv)(r_1'+ir_2')dt=\int_{a}^{b} (ur_1'-vr_2')+i \int_{a}^{b} vr_1'+ur_2')\]
	\[=\int_{\gamma}^{} \omega_1+i \int_{\gamma}^{}\omega_2\]
Riformulazione del calcolo di $F$
\[F(z)=\int_{\gamma:z_0\to z}^{} f \]
Questo implica che \[\oint_\gamma f=0\]
\textbf{Teorema di Morera}
\\ $\oint_\gamma f=0 \ \forall \gamma $ circuito $\subseteq \Omega \implies f$ olomorfa
\\\\\textbf{Teorema di Cauchy}
\\$f$ olomorfa su $\Omega\implies\oint f$ non cambia se sostituisco un circuito $\gamma \subseteq\Omega$ con uno ad omotopo (In particolare, se $\gamma$ omotopa ad un punto $\oint_\gamma f=0$)
