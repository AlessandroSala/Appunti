\documentclass[a4paper]{article}

\usepackage[utf8]{inputenc}
\usepackage[T1]{fontenc}
\usepackage{textcomp}
\usepackage{amsmath, amssymb}
\usepackage{tcolorbox}
\usepackage[italian]{babel} 

% figure support
\usepackage{import}
\usepackage{xifthen}
\pdfminorversion=7
\usepackage{pdfpages}
\usepackage{transparent}
\newcommand{\incfig}[1]{%
	\def\svgwidth{\columnwidth}
	\import{./figures/}{#1.pdf_tex}
}
\newcommand{\R}{\mathbb{R}}
\newcommand{\C}{\mathbb{C}}
\newcommand{\N}{\mathbb{N}}
\newcommand{\Z}{\mathbb{Z}}
\newcommand{\divider}{\noindent\rule{\textwidth}{0.5pt}}

\pdfsuppresswarningpagegroup=1

\begin{document}
\section{Applicazioni del teorema dei residui in campo reale}
\subsection{Primo tipo}
\begin{tcolorbox}
\[\int_{0}^{2\pi} f(\cos t, \sin t)dt\]
\end{tcolorbox}
\[\cos t = \frac{e^{it}+e^{-it}}{2}\]
\[\sin t = \frac{e^{it}-e^{-it}}{2}\]
Dunque 
\[\int_{0}^{2\pi} f(\cos t, \sin t)dt=\int_{0}^{2\pi} g(e^{it})ie^{it}dt \]
\[\int_{C_1(0)}^{} g(z)dz\]
Se $g$ soddisfa le ipotesi del teorema dei residui su $C_1(0) \subseteq  \Omega$, con $\gamma=C_1(0)$
\[=2\pi i \sum_{|z_0|<1}^{} \text{Res}(g,z_0)\]
Esempio: $\int_{0}^{2\pi} \frac{1}{2+\sin t}dt$
\subsection{Secondo tipo}
\begin{tcolorbox}	
	\[\text{V.P.  }\int_{\R}^{} f(x)dx:= \lim_{R \to +\infty} \int_{-R}^{R} f(x)dx \] 
\end{tcolorbox}
La definizione cambia leggermente nel caso sia presente una singolarità su $\R$.
Se $f$ è integrabile (secondo Riemann) allora 
\[\int_{\R}^{} f(x)dx=\lim_{R \to +\infty} \int_{-R}^{R} f(x)dx\]
In generale può accadere che il V.P. $\int_{\R}^{} f(x)dx \in\R$ ma $f$ non è integrabile
\textbf{Esempio:} 
\[f(x)=\begin{cases}
	\frac{1}{x}\ \ \ x\ge 1\\
	1\ \ \ x\in[0,1]\\
	-1\ \ \ x\in[-1,0]\\
	\frac{1}{x}\ \ \ x\le -1
\end{cases}
\]
$f$ non è integrabile secondo Riemann, ma il V.P. è uguale a 0.\\\divider
\\\textbf{Ipotesi: }$f=f(z)$ abbia un numero finito di singolarità nel semipiano $\{\text{Im}z>0\} $ (e nessuna singolarità sull'asse reale)+ (*) ipotesi di decadimento.
\[I=\lim_{R \to +\infty} \int_{-R}^{-R} f(x)dx+\int_{C_R^+(0)}^{} f(z)dz-\int_{C_R^+(0)}^{} f(z)dz \]
\begin{figure}[ht]
    \centering
    \incfig{semicirconferenza}
    \caption{Semicirconferenza}
    \label{fig:semicirconferenza}
\end{figure}
\[I=\lim_{R \to +\infty} \int_{\gamma_R}{f(z)dz} -\lim_{R \to +\infty} \int_{C_R(0)^+}^{} f(z)dz\]
Dove $\gamma_R=[-R,R]+C_R^+(0)$
\\Per il teorema dei residui
\[I=2\pi i \sum_{\substack{z_0\in S,\\  \text{Im}z_0>0}}^{} \text{Res}(f,z_0)\]
L'indice di avvolgimento è uguale a 1.
\subsubsection{Lemma tecnico di decadimento}
Se $\exists \alpha >1$ tale che $|f(z)|\le \frac{c}{|z|^\alpha}$ (per $|z|$ abbastanza grande) (*), allora
\[\lim_{R \to +\infty} \int_{C_R^+(0)}^{} f(z)dz=0\]
Aggiungendo l'ipotesi di decadimento all'integrale precedente si avrà il risultato scritto.
\\Si ha un calcolo analogo per il semipiano $\{\text{Im}<0\} $
\subsection{Terzo tipo}
\begin{tcolorbox}	
	\[I=\text{V.P.  }\int_{\R}^{} f(x)e^{i\omega x}dx=2 \pi i \sum_{\substack{z_0\in S\\\text{Im}z_0>0}}^{\infty} \text{Res}(f(z)e^{i\omega z},z_0)\] 
\end{tcolorbox}
Dove $\omega \in \R^+$
\\\textbf{Ipotesi: }$f(z)e^{i\omega x}$ abbia un numero finito di singolarità nel semipiano $\{\text{Im}z>0\} $ (e nessuna singolarità sull'asse reale) + (**) lemma di Jordan.
\[I=\lim_{R \to +\infty} \int_{-R}^{-R} f(z)e^{i\omega z}dz+\int_{C_R(0)^+}^{} f(z)e^{i\omega z}dz-\int_{C_R(0)^+}^{} f(z)e^{i\omega z}dz \]
\begin{figure}[ht]
    \centering
    \incfig{semicirconferenza}
    \caption{Semicirconferenza}
    \label{fig:semicirconferenza}
\end{figure}
\[I=\lim_{R \to +\infty} \int_{\gamma_R}{f(z)e^{i\omega z}dz} -\lim_{R \to +\infty} \int_{C_R(0)^+}^{} f(z)e^{i\omega z}dz\]
Dove $\gamma_R=[-R,R]+C_R^+(0)$
\\Per il teorema dei residui
\[I=2\pi i \sum_{\substack{z_0\in S,\\  \text{Im}(z_0)>0}}^{} \text{Res}(f(z)e^{i\omega x},z_0)\]
L'indice di avvolgimento è uguale a 1.\\
Il secondo termine dell'integrale si elide grazie a il 
\subsubsection{Lemma di Jordan}
Sotto l'ipotesi 
\[\lim_{R \to +\infty}\text{sup}_{z\in c_R^+(0)}|f(z)|=0  \ \ (**)\]
\[\lim_{R \to +\infty} \int_{C_R^+(0)}^{} f(z)e^{i\omega x}dz=0\] 
\textbf{Osservazione:} Variante analoga nel semipiano $\{\text{Im}z<0\} $ quando $\omega \in \R^-$
\\Jordan vale anche per $\omega \in \R^-$ in $C_R^-(0)$
\\Esempio: $I= \text{V.P.}\int_{\R}^{} \frac{\cos x}{1+x^2}dx$
\subsection{Quarto tipo}
\begin{tcolorbox}
\[I=\text{V.P.}\int_{\R}^{} f(x)dx\]	
\end{tcolorbox}
\textbf{Ipotesi:} $f(z)$ abbia un numero finito di singolarità su $\{\text{Im}>0\} $, \\$\lim_{R \to +\infty} \int_{C_R^+(0)}^{} f(z)dz=0$ (***) e abbia un numero finito di poli semplici su $\R$.
\[\gamma_{R,\varepsilon}=[-R,x_0-\varepsilon]-C_\varepsilon^+(x_0)+[x_0+\varepsilon,R]+C_R^+(0)\]
\[I=\lim_{R \to +\infty} \int_{\gamma_{R,\varepsilon}}f(z)dz + \lim_{\varepsilon \to 0} \int_{C_\varepsilon^+(x_0)}^{} f(z)dz-\lim_{R \to +\infty} \int_{C_R^+(0)}{f(z)dz}\]
\subsubsection{Lemma del polo semplice}
Se $x_0$ è un polo semplice
\[\lim_{\varepsilon \to 0} \int_{C_\varepsilon^+(x_0)}^{} f(z)dz=\pi i \text{Res}(f,x_0)\] 
\begin{figure}[ht]
    \centering
    \incfig{quarto-tipo}
    \caption{Quarto tipo}
    \label{fig:quarto-tipo}
\end{figure}
Esempio: $I=\text{(V.P.)}\int_{\R}^{} \frac{1-\cos 2x}{x^2}dx$
\section{Cenni aggiuntivi sull'analisi complessa}
\subsection{Residuo all'infinito}
\begin{tcolorbox}
	\textbf{Definizione:} Diciamo che $\infty$ è una singolarità isolata per $f$ se $f$ olomorfa nel complementare di una palla
\end{tcolorbox}
In modo equivalente: $g(z)=f(\frac{1}{z})$ ha una singolarità isolata nell'origine.
\[\text{Olomorfa su }\bigg|\frac{1}{z}\bigg|>R\iff |z|<R\]
\[\text{Res}(f,\infty):=\text{Res}\bigg(-\frac{1}{z^2}f\bigg(\frac{1}{z}\bigg),0\bigg)\]
\begin{tcolorbox}
	\textbf{Teorema:} La somma di tutti i residui di una funzione olomorfa su $\C\setminus \{\text{n. finito di punti}\} $ è zero. (compreso il punto all'infinito).
\end{tcolorbox}
Da utilizzare quando si deve calcolare la somma di tanti residui al finito. (stesso indice di avvolgimento)
\subsection{Funzioni polidrome}
$z=|z|e^{i\text{Arg}z}$, $\text{Arg}z:=\{\text{arg}+2k\pi:k\in\Z\} $, $\text{arg}z$ argomento principale $\in[0,2\pi]$.
\[\sqrt[n]{z}=\{\sqrt[n]{|z|}e^{i \frac{\text{Arg}z}{n}} \}=\{\sqrt[n]{|z|}e^{i( \frac{\text{arg}z}{n}+ \frac{2k\pi}{n})}:k=0,\ldots,n-1 \} \]
\[\log z=\{\log |z|+i\text{Arg}z\}\]
Alla radice sono associati n valori, al logaritmo $\infty$ valori.
\\$z \mapsto \sqrt[n]{z},\log z$ non sono funzioni!
\\Per definire una radice n-esima funzione si può specificare l'intervallo di variabilità di $\text{Arg}z$.
$z\in\C\mapsto \sqrt[n]{|z|} e^{i \frac{\text{Arg}z}{n}}$ con $\text{Arg}z\in[\overline \theta,\overline\theta+2\pi]$: Branca della radice n-esima.
\\\textbf{Osservazione:} Una branca della radice n-esima non è continua su $\C$. (è continua su $\C-\{\theta=\overline\theta\}$
\\Non è possibile incollare due branche diverse ottenendo una funzione olomrofa su $\C$.
\subsection{Funzioni armoniche}
\begin{tcolorbox}
	\textbf{Definizione:} $u:\R^2\to \R$ si dice armonica se 
	\[\nabla^2 u=0=u_{x x}+u_{y y}\]
\end{tcolorbox}
\textbf{Osservazione:} $f=f(z)$ olomorfa, $f=u+iv\implies u,v$ armoniche.
\[\begin{cases}
	u_x=v_y
	\\u_y=-v_x
\end{cases}
\begin{cases}
	u_{x x}=v_{y x}\\
	u_{y y}=-v_{xy}
\end{cases}
\]
Sommando le due equazioni
\[u_{x x}+u_{y y}=0\]
(Analogamente per $\nabla ^2 v=0$
\\\textbf{Osservazione 2:} $u$ armonica su $\Omega$, con $\Omega$ semplicemente connesso $\implies\exists v$ armonica coniugata di $u$ t.c. $f=u+iv$ olomorfa.
\end{document}
