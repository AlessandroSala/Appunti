
\section{Riconoscere le singolarità}
\begin{itemize}
	\item $z_0$ eliminabile $\iff$ parte singolare dello sviluppo $=0$
	\item $z_0$ polo 
	\item $z_0$ sing. essenziale 
\end{itemize}
Idea: $z_0$ è un polo per $f\iff z_0 $ è zero per la funzione $1/f$
\[\lim_{z \to z_0} \frac{1}{f}=0\]
\subsubsection{Principio di identità}
Sia $f:\Omega \subseteq  \C\to \C$ olomorfa e supposto $\Omega$ connesso
\\Sia $Z(f)=\{z\in\Omega : f(z)=0\} $, sono equivalenti i seguenti fatti
\begin{enumerate}
	
	\item $z_0\in \text{acc}(Z(f))$
	\item $f^{(n)}(z_0)=0\ \forall n\in\N$
	\item $Z(f)$ contiene un intorno di $z_0$
	\item $Z(f)\equiv\Omega$
\end{enumerate}
\\In conclusione: $Z(f)$
\begin{itemize}
	\item È fatto da punti isolati, oppure 
	\item Coincide con tutto $\Omega$ 
\end{itemize}
\subsubsection{Ordine di zeri}
Sia $f$ olomorfa su $\Omega$ connesso, $f\neq 0$ su $\Omega$, sia $z_0\in Z(f)$, Per il principio di identità, $z_0$ è uno zero isolato.
\\La (2) è quindi falsa $\implies \{n\in\N: f^{(n)}(z_0)\neq 0\} \neq 0$. Per il principio di buon ordinamento $\nu := \text{min}\{n\in\N: f^{(n)}(z_0)\neq 0\} $: \textbf{Ordine dello zero}.\\
\\\textbf{Osservazione}: $\nu$ è anche caratterizzato da:
\[f(z)=\sum_{n\ge \nu}^{} c_n(z-z_0)^n=c_\nu(z-z_0)^\nu+o(z-z_0)^\nu\]
Inoltre
\[\exists \lim_{z \to z_0} \frac{f(z)}{(z-z_0)^\nu}\in\C\setminus \{0\} \]
\subsection{Ordine dei poli}
$z_0$ polo per $f\iff z_0$ zero per $1 / f$
\begin{tcolorbox}	
\textbf{Definizione:} Sia $z_0$ polo per $f$. Chiamiamo ordine del polo $z_0$ l'ordine di $z_0$ come $z_0$ per $1 / f$
\end{tcolorbox}
\textbf{Definizione: }In particolare si dice polo semplice un polo di ordine 1.
\textbf{Osservazione:} L'ordine di un polo è caratterizzato anche:
\begin{itemize}
	\item $z_0$ polo di ordine $\nu$ per $f \iff z_0$ zero di ordine $\nu$ per $1 / f\iff$
\end{itemize}
		\[\exists \lim_{z \to z_0} \frac{1}{f(z)}\cdot \frac{1}{(z-z_0)^\nu}\in\C\setminus \{0\}\]
		\[\implies \exists \lim_{z \to z_0} (z-z_0)^\nu f(z)\in\C\setminus \{0\} \ \text{ (*)}\]

\begin{itemize}
	\item $z_0$ polo di ordine $\nu$ per $f \iff$
\end{itemize}
\[f(z)=\sum_{n=-\nu}^{+\infty} c_n(z-z_0)^n,\text{ con }c_{-\nu}\neq 0\]
Infatti 
\[f(z)=\sum_{n=-\infty}^{+\infty} c_n(z-z_0)^n\implies(z-z_0)^\nu f(z)=\sum_{n=-\infty}^{+\infty} c_n(z-z_0)^{n+\nu}\]
(*)$\iff$ tutti i coefficienti $c_n$ con $n+\nu<0$ e il coefficiente $c_{-\nu}\neq 0$.
\\Dunque, lo sviluppo di Laurent di una funzione che ha un polo ha parte singolare composta da un numero finito di termini.
\\È quindi possibile classificare le singolarità guardando lo sviluppo in serie di Laurent, guardando la parte singolare
\begin{itemize}
	\item p. singolare nulla: ELIMINABILE
	\item p. singolare con numero finito di termini: POLO
	\item p. singolare con infiniti termini: ESSENZIALE
\end{itemize}
\textbf{Osservazioni su zeri e poli} 
\begin{enumerate}
	\item Se $f,g$ hanno entrambe uno zero in $z_0$ o entrambe un polo, allora
\end{enumerate}
\[\exists  \lim_{z \to z_0} \frac{f}{g}=\lim_{z \to z_0} \frac{f'}{g'}\]
Cenno di dim (Primo caso):
\[f(z)=c_\nu(z-z_0)^\nu+o(z-z_0)^\nu, g(z)=c_\eta(z-z_0)^\eta+o(x-x_0)^\eta\]
Sono possibili solo tre casi: $\eta=\nu\implies$ limite finito, $\nu>\eta$ limite 0, $\nu < \eta$ limite infinito.
\begin{enumerate}
	\setcounter{enumi}{1}
	\item $z_0$ zero di ordine $\nu$ per $f \iff \lim_{z \to z_0} \frac{(z-z_0)}{f(z)}=\nu$
		\\
\end{enumerate}
Questa è una modalità per calcolare l'ordine.
Dim: $f(z)=c_\nu(z-z_0)^\nu+\ldots$, $f'(z)=\nu c_\nu(z-z_0)^{(\nu-1)}+\ldots$
\[\implies \frac{(z-z_0)f'(z)}{f(z)}=\frac{\nu c_nu(z-z_0)^\nu+o(z-z_0)^\nu}{c_\nu(z-z_0)^\nu+o(z-z_0)^\nu}\to_{z-z_0}\nu\]
\begin{enumerate}
	\setcounter{enumi}{2}
	\item $z_0$ zero di ordine $\nu$ per $f$, con $\nu \ge 1\implies z_0$ zero di ordine $\nu-1$ per $f'$.
$z_0$ polo di ordine $\nu$ per $f$, con $\nu \ge -1\implies z_0$ polo di ordine $\nu+1\text{ per }f$.
Controllare su libro.
\end{enumerate} 
\subsubsection{Unicità del prolungamento analitico}
\begin{tcolorbox}
	Sia $\Omega \subseteq  \C$ connesso, sia $S \subseteq  \Omega$ tale che $\text{acc}(S)\cap \Omega \neq 0$.
	\\Data $f:S\to \C$, esiste al più una $\tilde f:\Omega\to \C$ olomorfa tale che $\tilde f|_S=f$
\end{tcolorbox}
\textbf{Dimostrazione: }Supponiamo $\tilde f_1,\tilde f_2:\Omega\to \C$ siano prolungamenti di $f$. Tesi: $\tilde f_1\equiv \tilde f_2$
\\Considerando $g:=\tilde f_1-\tilde f_2$. Tesi: $g\equiv 0$.
\\$g$ è olomorfa, $S \subseteq  Z(g)\implies Z(g)$ ha punti di accumulazione in $\Omega$, quindi $Z(g)\equiv \Omega$ 
