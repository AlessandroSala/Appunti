
\subsection{Supporto e Classe $C_0$}
Ovvero:
\[\forall f \in L^{p}(E)\ \exists \{ \varphi_n\}	\subseteq  C_0^\infty(E) \text{ tale che } \|\varphi_n -\varphi\|\to _{n\to +\infty}0\]
\[\forall f\in L^{p}(E),\forall \varepsilon>0\ \exists \varphi\in C_0^\infty(E)\text{ tale che }\|f-\varphi\|_{L^{p}}<\varepsilon
\] 
Falso nel caso $p=\infty$
\begin{tcolorbox}
	\textbf{Definizione:} Data $\varphi\in C^\infty(E)$ il supporto di $\varphi$ è 
	\[\overline{\{x\in E:\varphi(x)\neq 0\}} \]
\end{tcolorbox}
\begin{tcolorbox}
	Un insieme $K\subseteq  \R^n$ è compatto se e solo se è limitato e chiuso
\end{tcolorbox}
\begin{tcolorbox}
	$C_0^\infty(E):=\{\varphi :E\to \R$ derivabili infinite volte tali che supp$(\varphi)$ è un sottoinsieme compatto di $E\} $
\end{tcolorbox}
\subsection{Prodotto di convoluzione}
\textbf{Osservazione: }$f,g\in L^{1}(E)\centernot\implies f\cdot g\in L^{1}(E)$
\\Nel caso $E=\R$ si può definire un prodotto che rimanga interno a $L^{1}(\R)$ 
\begin{tcolorbox}
	\textbf{Proposizione 1} 
	\\Siano $f,g(x)\in L^{1}(\R)$ Si definisce prodotto di convoluzione
	\[ f*g:=\int_{\R_y}^{} f(x-y)g(y)dy\] 
\end{tcolorbox}
\begin{enumerate}
	\item $f*g$ esiste finito per q.o. $x\in \R$, ovvero q.o. $x\in \R,\ y\mapsto f(x-y)g(y)$ è integrabile su $\R$
	\item$f*g \in L^{1}(\R)$
	\item$\|f*g\|_{L^1}\le \|f\|_{L^1}\|g\|_{L^1}$ 
\end{enumerate}
\textbf{Dimostrazione} 
\\Consideriamo $H(x,y):=f(x-y)g(y)$, a priori non sappiamo se $H\in L^{1}(\R_x \times \R_x)$, dunque non è possibile applicare direttamente fubini.
\\Quindi consideriamo $|H|\ge 0$ e applichiamo il teorema di Tonelli.
\\Verificando le ipotesi:
\begin{itemize}
	\item Integro prima in dx
\end{itemize}
\[\int_{\R_x}^{} |H(x,y)|dx=|g(y)|\int_{\R_x}^{}|f(x,y)|dx\]   
Con la sostituzione $z=x-y$ 
\[=|g(y)|\int_{\R_z}^{} |f(z)|dz=|g(y)|\cdot \|f\|_1<+\infty\] 
\begin{itemize}
	\item Integro in dy
\end{itemize}
\[\int_{\R_y}^{} \bigg[\int_{\R_x}^{} |H_(x,y)|dx\bigg]dy=\|f\|_{L^1}\int_{\R_y}^{} |g(y)|dy=\|f\|_{L^1}\|g\|_{L^1}<+\infty\]
Dunque per Tonelli $|H|\in L^{1}(\R_x \times \R_y)\implies H\in L^1(\R_x \times \R_y)$
\\A questo punto posso applicare Fubini ad $H$ 
\\Dunque per q.o. $x,\ y\mapsto H(x,y)=f(x,y)g(y)\in L^{1}(\R_y)$
\\\textbf{Dimostrazione 3 (che implica 2)}
\[\|f*g\|_{1}=\int_{\R_x}^{} |f*g(x)|dx=\int_{\R_x}^{} \bigg|\int_{\R_y}^{} f(x-y)g(y)dy\bigg|dx\le \]
\[\le \int_{\R_x}^{} \int_{\R_y}{|f(x-y)| |g(y)|dydx}\le \int_{\R_y}^{} \int_{\R_x}{|f(x-y)| |g(y)|dydx}\]
(Per Fubini)
\[=\int_{\R_y}^{} |g(y)| \int_{\R_x}^{} |f(x-y)dxdy=\|f\|_1 \int_{\R_y}^{} |g(y)|dy=\|f\|_1\cdot \|g\|_1\]   
\divider\\
\textbf{Osservazione}
\begin{itemize}
	\item vale la proposizione 1 anche su $\R^n$ 
	\item $f* g=g * f$ 
	\item le funzioni devono essere deinite su tutto lo spazio
\end{itemize}
\textbf{Estensione:} $f\in L^{1}(\R),g\in L^{p}(\R)\implies$
\begin{enumerate}
	\item $f * g(x)$ esiste per q.o. $x$ 
	\item $f*g\in L^{p}(\R)$ 
	\item $\|f * g\|_p\le \|f\|_1 \|g\|_p$
\end{enumerate}
\[H(x,y)=|f(x-y)|^p|g(y)|^p\]
\begin{tcolorbox}
	\textbf{Proposizione 2} 
	\\Siano $f\in C_0^\infty(\R)(\subseteq  L^{1}(\R)),\ g\in L^{1}(\R)$, allora:
	\begin{enumerate}
		\item $f*g\in C^\infty(\R)$ 
		\item $(f*g)^{(k)}=f^{(k)}*g\ \forall k$
	\end{enumerate}
\end{tcolorbox}
Idea della dimostrazione: 
\[f * g(x)=\int_{\R_y}^{} f(x-y)g(y)dy\]
\[(f *g)'(x)=\int_{\R_y}^{} f'(x-y)g(y)dy\]
\textbf{Osservazione 1:} 
Vale con $k$ al posto di $\infty$\\
\textbf{Osservazione 2:} In generale nelle ip. della Prop.2 $f*g$ non è a supporto compatto. 
\\\divider\\
\\\textbf{Idea della dim. del teorema di approssimazione di funzioni $L ^p$ con funzini regolari}
\\Prendiamo $p=1$, data $f\in L^{1}(\R)$, vogliamo costruire $\varphi_n\subseteq  C_0^\infty(\R)$ tale che $\varphi_n\to f$ in $L^{1}(\R)$. Prendiamo
\[f_n:=f*\rho_n\]
Dove $\rho_n$ successione di mollificatori
\[\rho_n(x)=n\rho(nx) \text{ dove }\rho \text{ è un nucleo di convoluzione}\]
\begin{itemize}
	\item $\rho\in C_0^\infty(\R),\ \rho\ge 0,\ \text{supp}(\rho)\subseteq  [-1,1],\ \int_{\R}^{} \rho(x)dx=1$ 
	\item $\rho_n\in C^\infty_0(\R),\ \rho_n\ge 0,\text{ supp}(\rho_n)\subseteq [-\frac{1}{n},\frac{1}{n}],\ \int_{\R}^{} \rho_n(x)dx=1 $
\end{itemize}
Si può dimostrare usando i teoremi di convergenza dominata che $\varphi_n\to f$ in $L^{1}(\R)$.
\\\textbf{Osservazione:} Per guadagnare anche il supporto compatto, occorre prima "trovare" $f$, cioè considerare
\[f_k=f\cdot \chi_{[-k,k]}=\begin{cases}
	f\ \ \text{se }x\in [-k,k]
	\\0\ \ \text{se }x\not\in [-k,k]
\end{cases}\]
Approssimo f per convoluzione:
\[f_k*\rho_n\in C_o^\infty\to _{n\to +\infty}f_k\]
\[\varphi_n=f_{k(n)}*\rho_n\]
\subsection{Teorema fondamentale del calcolo}
\begin{tcolorbox}
\textbf{Teorema di differenziazione nella teoria di Lebesgue}
\\Sia $f\in L^{1}([a,b])$, sia $F(x):=\int_{a}^{x} f(t)dt$, essa è derivabile q.o. su $[a,b]$ e
\[F'(x)=f(x)\text{ per q.o. }x\in (a.b)\]
\end{tcolorbox}
Esempio: $f(x)=\text{sign}(x)$, $x\in [-1,1]$.
\begin{tcolorbox}
	\textbf{Definizione: }Diciamo che $F\in \text{A.C.}([a,b])$, ovvero lo spazio delle funzioni assolutamente continue su $[a,b]$ se $\exists f\in L^{1}([a,b])$ tale che
	\[F(x)=\int_{a}^{x} f(t)dt+c\text{ con }c\in \R\]
\end{tcolorbox}
\textbf{Osservazione 1:} Tale spazio è vettoriale (per la linearità della derivata e dell'integrale).
\\\textbf{Osservazione 2:} $F\in \text{AC}([a,b])\implies$ 
\[F(b)-F(a)=\bigg[\int_{a}^{b} f(t)dt+c \bigg]-\bigg[\int_{a}^{a} f+c \bigg]=\int_{a}^{b} f(t)dt   \]
\textbf{Osservazione 3:} $F\in \text{AC}([a,b]),\ F'=0$ q.o. su  $[a,b]\implies F=c$ $\forall x\in [a,b]$ \\
Questa implicazione è falsa se togliamo l'ipotesi che $F\in \text{AC}([a,b])$ 
\\Esistono funzioni (non in $\text{AC}([a,b])$ che sono derivabili q.o. con derivata prima nulla q.o. ma non costanti.
\\\textbf{Esempio:} una funzione derivabile $f$ continua con  $f'=0$ q.o. su $(0,1)$ ma $f$ non costante (scala di Cantor).
\begin{figure}[ht]
    \centering
    \incfig{scaladicantor}
    \caption{Scala di Cantor}
    \label{fig:scaladicantor}
\end{figure}
\\La successione $\{f_n\} \subseteq  C^0([0,1])$ risulta di Cauchy in $\|.\|_\infty$.
\\Poiché ($C^0([0,1]),\|.\|_\infty$ ) è uno spazio di Banach:
\[\exists f=\lim_{n \to +\infty} f_n\in C^0([0,1])\]
$f(0)=0$, $f(1)=1$, con $f'=0$ q.o. su $(0,1)$ 
\begin{tcolorbox}
\textbf{Proposizione (caratterizzazione puntuale di AC)}
\[F\in \text{AC}([a,b])\iff \forall \varepsilon>0\exists \ \delta>0\text{ tale che } \forall \text{ famiglia }\{(x_i,y_i)\} _{i=1,\ldots,N}\]
di intervalli a 2 a 2 disgiunti $\subseteq  (a,b)$ con
\[\sum_{k=1}^{N} |y_k-x_k|<\delta,\text{ si ha }\sum_{k=1}^{N} |F(x_k)-F(y_i)|<\varepsilon\]
\end{tcolorbox}
\textbf{Osservazione:} Per $N=1$ si ha continuità uniforme
\\$\implies\text{AC}([a,b])\subseteq  \{\text{funzioni uniformemente continue su }[a,b]\} $ 
\\\textbf{Conseguenze della proposizione}
\begin{itemize}
	\item $F,G\in \text{AC}([a,b])\implies F,G\in AC([a,b])$ 
	\item $F,G\in \text{AC}([a,b])\implies$
\end{itemize}
\[F(b)G(b)=F(a)G(a)=\int_{a}^{b} (F\cdot G)'(t)dt=\int_{a}^{b} (F'G+FG') dt  \]
Ovvero
\[\int_{a}^{b} fG=-\int_{a}^{b} Fg+F\cdot G|_a^b \] 
$\implies$ vale in AC la formula di integrazione per parti.

