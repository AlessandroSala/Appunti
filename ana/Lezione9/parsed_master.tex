
\section{Integrazione secondo Lebesgue}
\begin{enumerate}
	\item Misure e funzioni misurabili
	\item Definizione di integrale di Lebesgue
	\item Confronto con Riemann
	\item Teoremi principali
\end{enumerate}
\begin{figure}[ht]
    \centering
    \incfig{lebesgue}
    \caption{Integrale secondo Lebesgue, intuizione geometrica}
    \label{fig:lebesgue}
\end{figure}
\[\int_{a}^{b} f=\lim_{N \to +\infty} \sum_{k=1}^{N}l(f^{-1}(j_k))\cdot y_k\]
\subsubsection{Misure e funzioni misurabili}
\begin{tcolorbox}
	\textbf{Definizione:} Sia $X$ insieme, e sia $F\subseteq P(X)$ una famiglia di sottoinsiemi di $X$.
	\\$F$ si dice una $\sigma$-algebra se:
	\\(i) $\varnothing\in F$
	\\(ii) $A\in F\implies X\setminus A\in F$
	\\(iii) $\{A_n\} _{n \in \N}\subseteq F\implies \bigcup_{n}A_n\in F$
\end{tcolorbox}
\textbf{Osservazione:} $\{A_n\} _{n\in\N}\subseteq F\implies \bigcap_n A_n \in F$
\\Esempi
\begin{itemize}
	\item $X$ qualsiasi, $F=P(X)=$ parti di $X$
	\item $X=\R^n$, $F=$ la pià piccola $\sigma$-algebra contenente gli aperti ($\sigma$ di Borell) 
\end{itemize}
\begin{tcolorbox}
	\textbf{Definizione: } la coppia $(X,F)$ si dice spazio misurabile

\end{tcolorbox}
\begin{tcolorbox}
	\textbf{Definizione:} Sia $(X,F)$ spazio misurabile, una misura positiva su $(X,F) $ è una funzione
\[\mu:F\to \R\cup \{+\infty\} \] 
tale che
\begin{enumerate}
	\item $\mu(A)\ge 0\forall A\in F$ (positività)
	\item Se $\{A_n\} $ è una famiglia al più numerabile di insiemi di $F$ 2 a 2 disgiunti allora 
\end{enumerate}
\[\mu(\cup _n A_n)=\sum_{n\ge 1}^{} \mu(A_n)\]
(additività, eventualmente $+\infty=+\infty$)
\end{tcolorbox}
Esempi:
\begin{itemize}
\item $(X,P(X)),\ \mu(A)=\text{card}A$
\item $(X,P(X))$ fissato $x_0\in X$
	\[\mu(A)=\begin{cases}
		1\text{  se }x_0\in A\\
		0\text{  se }x_0\not\in A
	\end{cases}\]
\end{itemize}
\textbf{Osservazione:} Seguono da 1), 2)
\begin{enumerate}
\setcounter{enumi}{2}
	\item $A_1\subseteq A_2\subseteq \ldots,\ A_i\in F$\[\implies \mu(\cup_n A_n)=\lim_{n \to \infty} \mu(A_n)\]
	\item $A_1 \supseteq A_2\supseteq A_3\supseteq \ldots, \ A_i\in F,\mu (A_1)<+\infty$\[\implies\mu( \cap_n A_n)=\lim_{n \to \infty}\mu(A_n)\] 
\end{enumerate}
\begin{tcolorbox}
	\textbf{Teorema:} Esistono su $\R^n$ una $\sigma$-algebra $M$ (misurabile secondo lebesgue) e una misura positiva m (misura di Lebesgue in $\R^n$) tali che:
	\begin{itemize}
		\item Tutti gli insiemi aperti appartengono a $M$
		\item $A\in M$ e $m(A)=0\implies\forall B\subseteq A,\ B\in M \text{ e } m(B)=0$ (completezza)
		\item $A=\{x\in \R^n: a_i<x_i<b_i\ i=1,\ldots,n\}$\[\implies m(A)=\prod_{i=1}^n(b_i-a_i)=(b_1-a_1)(b_2-a_2)\ldots(b_n-a_n)\]
	\end{itemize}
	[\ldots] 
\end{tcolorbox}
\textbf{Osservazione: }Non tutti i sottoinsiemi di $\R^n$ sono misurabili secondo Lebesgue.
\\\textbf{Osservazione: }La misura di Lebesgue in $\R^n$ estende il concetto di volume n-dimensionale
\\\textbf{Osservazione:} Gli insiemi di misura nulla sono importanti 
\begin{tcolorbox}
	\textbf{Definizione:} Una funzione $f:\R^n\to \R$ si dice misurabile secondo Lebesgue se
	\[\forall A\subseteq \R \text{ aperto},\ f^{-1}(A) \text{ misurabile secondo Lebesgue}\]
	\[\forall C\subseteq\R\text{ chiuso }, \ f^{-1}(C)\text{ misurabile secondo Lebesgue}\]
\end{tcolorbox}
\textbf{Osservazione:} $f$ continua $\implies$f misurabile secondo Lebesgue ($f$ continua $\implies\forall A$ aperto $f^{-1}(A)$ aperto$\implies \forall  A$ aperto $f^{-1}(A)$ misurabili
\\\textbf{Osservazione 2:} Sono misurabili anche limiti, inferiore, superiore di funzinoni continue (di funzioni misurabili)\bigbreak
Più in generale se 
\[f: E\to \R\] con $E$ misurabile, $f$ si dice misurabile secondo Lebesgue se $\forall  A\subseteq\R$ aperto $E\cap f^{-1}(A)$ misurabile secondo Lebesgue
\subsubsection{Definzione di integrale secondo Lebesgue}
Sia $f:E$ misurabile $\subseteq\R^n\to \R$ misurabile.
\\\textbf{Funzioni semplici}
\\$S$ funzione semplice è una funzione (misurabile) che assume un numero finito di valori (ciascuno su un insieme misurabile).
\[S=\sum_{k=1}^{N} \alpha_k\chi_{E_i},\ \chi_E=\begin{cases}
	1\ \ x\in E
	\\0\ \ x\not\in E
\end{cases}\]
Dove gli $E_i$ sono insiemi misurabili 2 a 2 disgiunti 
\[\int_{E}^{} S:=\sum_{k=1}^{N} \alpha_k m(E_k)\]
\textbf{Precisazione:} con la convenzione $0\cdot \infty=0$
\\\textbf{Funzioni misurabili} $f\ge 0$
\[\int_{E}^{} f:=\sup_{\substack{S \text{ semplici}\\S\ge  f}}\int_{E}^{}S\ \ \ \bigg(=\inf_{\substack{S \text{ semplici}\\S\ge f}} \int_{E}^{S}\bigg)\]
\textbf{Funzioni misurabili di segno qualsiasi}
\\Data $f$ misurabile su $E$ misurabile, scriviamo:
\\$f=f^+-f^-$ con $f^+,f^-\ge 0$, $f^+:=\max \{f,0\} $, $f^-:=-\min \{f,0\} $
\begin{figure}[ht]
    \centering
    \incfig{funzioni-di-segno-qualsiasi}
    \caption{Funzioni di segno qualsiasi}
    \label{fig:funzioni-di-segno-qualsiasi}
\end{figure}
\[\int_{E}^{} f:=\int_{E}^{} f^+-\int_{E}^{} f^-\]
A patto che almeno uno tra i due integrali sia finito, (eventualmente l'integrale vale $\pm \infty$
\begin{tcolorbox}
	\textbf{Definizione:} $f:E\to \R$ misurabile si dice \emph{integrabile secondo Lebesgue} se
	\[\int_{E}^{} f\in\R\] 
\end{tcolorbox}
\textbf{Osservazione:} $f$ è integrabile secondo Lebesgue $\iff \int_{E}^{} f^{\pm}\in\R $
\\Quindi $f$ integrabile seconodo Lebesgue $\iff|f|$ integrabile secondo Lebesgue, infatti
\[|f|=f^++f^-\]
\subsubsection{Proprietà principali dell'integrale di Lebesgue}
1) \textbf{Linearità: }$f,g$ Lebesgue integrabili, $\alpha, \beta\in\R\implies \alpha f+\beta g$ Lebesgue integrabile e 
\[\int_{E}^{} (\alpha f +\beta g)=\alpha \int_{E}^{} f+\beta \int_{E}^{} g\]
2) \textbf{Monotonia: }$f,g$ Lebesgue integrabili, $f\le g$ q.o. su $E$
\[\implies \int_{E}^{f} \le \int_{E}^{} g\]
3) \textbf{Maggiorazione del modulo: }$f$ Lebesgue integrabile 
\[\implies \bigg|\int_{E}^{} f\bigg|\le \int_{E}^{} |f|\]
Segue da 2), $-|f|\le f\le |f|\implies-\int_{E}^{} |f|\le \int_{E}^{} f\le \int_{E}^{} |f|$
\bigbreak4) L'integrale di Lebesgue "non vede" gli insiemi di misura nulla.
\\Sia $S$ semplice, $E\to \R$
\[S(x)=\begin{cases}
	0\text{ su }E\setminus N
	\\1\text{ su }N
\end{cases}\ \ m(N)=0\]
\[\int_{E}^{} S=m(E\setminus N)\cdot 0+m(N)\cdot 1=0\]
Più in generale, se $f$ misurabile: $E\to \R$ se $f$ si annulla su $E$ tranne che su un insieme di misura nulla
\[\int_{E}^{} f=0\] 
Conseguenza: se $f,g$ misurabili: $E\to \R$ se $f=g$ su $E$ tranne che su un insieme di misura nulla
\[\int_{E}^{} f=\int_{E}^{} g\]  
\begin{tcolorbox}
	\textbf{Definizione:} Si dice che una proprietà $P(x)$ vale per q.o. $x\in E$ se $P(x)\text{ vale }\forall x\in E\setminus N,\text{ con }m(N)=0$
\end{tcolorbox}
Quindi
\begin{itemize}
	\item $f=0$ q.o. su $E\implies \int_{E}^{} =0 $
	\item $f=g$ q.o. su $E\implies \int_{E}^{} f=\int_{E}^{} g $
\end{itemize}
\subsection{Confronto Riemann-Lebesgue}
\subsubsection{Integrali propri}
$f$ R-integrabile $\implies f$ L-integrabile, in caso affermativo, i valori degli integrali coincidono, in generale non vale il viceversa.
\\($\implies$) Le funzioni semplici seconodo Lebesgue $S_L$ sono una classe più ampia delle funzinoi semplici secondo Riemann $S_R$
\[\sup_{\substack{s\in S_R\\s\le f}}\int_{E}^{}s\le \sup_{\substack{s\in S_L\\s\le f}}\int_{E}^{} s \le \inf_{\substack{s \in S_L\\s\ge f}}\int_{E}^{}  s\le   \inf_{\substack{s\in S_R\\s\ge f}}\int_{E}^{}s\]
Controesempio: $\exists f$ L-integrabile ma non R-integrabile.
\[f(x)=\begin{cases}
	1\text{ se }x\in\Q\\
	0\text{ se }x\in \R-\Q
\end{cases}\]
Non R-integrabile poiché approssimando da sotto e da sopra non si trova lo stesso valore
\[s\in S_R,s\ge f\implies s\ge 1\text{ su }(0,1)\implies \int_{0}^{1} s\ge 1\]
\[s\in S_R, s\le f\implies s\le 0\text{ su }(0,1)\implies \int_{0}^{1} s\le 0\]
$f$ è però L-integrabile, $\int_{0}^{1} f=0$
\[\int_{0}^{1} f=1\cdot m(0,1)\cap\Q+0\cdot m((0,1)\cap(\R\setminus\Q))=0\]
\subsubsection{Integrali impropri}
In $\R$, supponiamo che $f$ limitata, sia R-integrabile su $[-L,L]\forall L>0$.
\\Allora: $f$ L-integrabile su $\R \iff|f|$ R-integrabile (in senso improprio su $\R$).
\\E in tal caso l'integrale di Lebesgue di $f$ coincide con l'integrale improprio di $f$.
\\Analogamente se $f$ non è limitata.
\\Controesempio: una funazione R-integrabile ma non R-integrabile in modulo e quindi non L-integrabile.
\[f(x)= \frac{\sin x}{x}\text{  su }(0,+\infty)\]
Riemann integrabile su $(0,+\infty)$ (tramite analisi complessa)
\\Ma non è Riemann integrabile in modulo
\[\int_{0}^{+\infty} \frac{|\sin x|}{x}=+\infty\]
(Tramite serie)
