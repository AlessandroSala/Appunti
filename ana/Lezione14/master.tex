\documentclass[a4paper]{article}

\usepackage[utf8]{inputenc}
\usepackage[T1]{fontenc}
\usepackage{textcomp}
\usepackage{amsmath, amssymb}
\usepackage{tcolorbox}
\usepackage[italian]{babel} 

% figure support
\usepackage{import}
\usepackage{xifthen}
\pdfminorversion=7
\usepackage{pdfpages}
\usepackage{transparent}
\newcommand{\incfig}[1]{%
	\def\svgwidth{\columnwidth}
	\import{./figures/}{#1.pdf_tex}
}
\newcommand{\R}{\mathbb{R}}
\newcommand{\C}{\mathbb{C}}
\newcommand{\N}{\mathbb{N}}
\newcommand{\Z}{\mathbb{Z}}
\newcommand{\divider}{\noindent\rule{\textwidth}{0.5pt}}

\pdfsuppresswarningpagegroup=1

\begin{document}
\section{Distribuzioni}
\begin{tcolorbox}
\textbf{Definizione: }Sia $\Omega$ aperto di $\R^n$ 
\[C_0^\infty(\Omega)=\{\text{funzioni }C^\infty\text{ su }\Omega\text{ con supporto compatto in }\Omega\}\]
È uno spazio vettoriale
\end{tcolorbox}
Muniamo $C_0^\infty$ di una \textbf{convergenza} 
\begin{tcolorbox}
	\textbf{Definizione: }Sia $\{\varphi_k\} \subseteq  C_0^\infty(\Omega)$. Diciamo che 
	\[\varphi_k\to 0\text{ in }C_0^\infty(\Omega)\text{ se }\]
\begin{enumerate}
	\item $\exists K$ compatto, indipendente da h, tale che $\text{supp}(\varphi_h)\subseteq  K\ \forall h> >\nu$
	\item $\varphi_h\to 0$ uniformemente su $K$ con tutte le derivate $\forall \alpha$ multiindice $D^\alpha\varphi_h\to 0$ unif. su $K$
\end{enumerate}
\end{tcolorbox}
\begin{tcolorbox}
	\textbf{Definizione: }Lo spazio $C_0^\infty(\Omega)$ munito della convergenza definita sopra si indica con $\mathcal D(\Omega)$ e si chiama \emph{spazio delle funzioni test}
\end{tcolorbox}
\begin{tcolorbox}
	\textbf{Definizione: }Lo spazio delle distribuzioni su $\Omega $, che si indica con $\mathcal D'(\Omega)$ è lo spazio degli operatori $T:\mathcal D(\Omega)\to \R$ lineari e continui rispetto alla convergenza introdotta su $\mathcal D(\Omega)$.\\Ovvero, una distribuzione è un operatore $T:\mathcal D(\Omega)\to \R$ tale che 
	\begin{itemize}
		\item $T$ lineare
		\item $T$ continuo ($\varphi_h\to 0$ in $\mathcal D(\Omega)\impliesT(\varphi_h)\to 0$ in $\R$)
	\end{itemize}
\end{tcolorbox}
\textbf{Esempi} 
\begin{enumerate}
	\item Sia $u\in L^{1}(\Omega)$, ad $u$ posso associare una distribuzione $T_u\in \mathcal D'(\Omega)$
\end{enumerate}
\[T_u(\varphi):=\int_{\Omega}^{} u\varphi\ \ \forall \varphi\in \mathcal D(\Omega)\]
È ben definito:
\[\bigg| \int_{\Omega}^{} u\varphi\bigg|\le \int_{\Omega}^{} |u\varphi|\le \int_{K}^{} \max |\varphi | |u|\le \max_k|\varphi|\int_{K}^{} |u|  \]  
È lineare:
\[T_u(\alpha\varphi+\beta\psi)=\int_{\Omega}^{} u(\alpha\varphi+\beta\psi\]
completare

\end{document}
