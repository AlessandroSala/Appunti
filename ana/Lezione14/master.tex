\documentclass[a4paper]{article}

\usepackage[utf8]{inputenc}
\usepackage[T1]{fontenc}
\usepackage{textcomp}
\usepackage{amsmath, amssymb}
\usepackage{tcolorbox}
\usepackage[italian]{babel} 

% figure support
\usepackage{import}
\usepackage{xifthen}
\pdfminorversion=7
\usepackage{pdfpages}
\usepackage{transparent}
\newcommand{\incfig}[1]{%
	\def\svgwidth{\columnwidth}
	\import{./figures/}{#1.pdf_tex}
}
\newcommand{\R}{\mathbb{R}}
\newcommand{\C}{\mathbb{C}}
\newcommand{\N}{\mathbb{N}}
\newcommand{\Z}{\mathbb{Z}}
\newcommand{\divider}{\noindent\rule{\textwidth}{0.5pt}}

\pdfsuppresswarningpagegroup=1

\begin{document}
\section{Distribuzioni}
\begin{tcolorbox}
\textbf{Definizione: }Sia $\Omega$ aperto di $\R^n$ 
\[C_0^\infty(\Omega)=\{\text{funzioni }C^\infty\text{ su }\Omega\text{ con supporto compatto in }\Omega\}\]
È uno spazio vettoriale
\end{tcolorbox}
Muniamo $C_0^\infty$ di una \textbf{convergenza} 
\begin{tcolorbox}
	\textbf{Definizione: }Sia $\{\varphi_h\} \subseteq  C_0^\infty(\Omega)$. Diciamo che 
	\[\varphi_h\to 0\text{ in }C_0^\infty(\Omega)\text{ se }\]
\begin{enumerate}
	\item $\exists K$ compatto, indipendente da h, tale che $\text{supp}(\varphi_h)\subseteq  K\ \forall h> >\nu$
	\item $\varphi_h\to 0$ uniformemente su $K$ con tutte le derivate $\forall \alpha$ multiindice $D^\alpha\varphi_h\to 0$ unif. su $K$
\end{enumerate}
\end{tcolorbox}
\begin{tcolorbox}
	\textbf{Definizione: }Lo spazio $C_0^\infty(\Omega)$ munito della convergenza definita sopra si indica con $\mathcal D(\Omega)$ e si chiama \emph{spazio delle funzioni test}
\end{tcolorbox}
\begin{tcolorbox}
	\textbf{Definizione: }Lo spazio delle distribuzioni su $\Omega $, che si indica con $\mathcal D'(\Omega)$ è lo spazio degli operatori $T:\mathcal D(\Omega)\to \R$ lineari e continui rispetto alla convergenza introdotta su $\mathcal D(\Omega)$.\\Ovvero, una distribuzione è un operatore $T:\mathcal D(\Omega)\to \R$ tale che 
	\begin{itemize}
		\item $T$ lineare
		\item $T$ continuo ($\varphi_h\to 0$ in $\mathcal D(\Omega)\implies T(\varphi_h)\to 0$ in $\R$)
	\end{itemize}
\end{tcolorbox}
\textbf{Esempi} 
\begin{enumerate}
	\item Sia $u\in L^{1}(\Omega)$, ad $u$ posso associare una distribuzione $T_u\in \mathcal D'(\Omega)$
\end{enumerate}
\[T_u(\varphi):=\int_{\Omega}^{} u\varphi\ \ \forall \varphi\in \mathcal D(\Omega)\]
È ben definito:
\[\bigg| \int_{\Omega}^{} u\varphi\bigg|\le \int_{\Omega}^{} |u\varphi|\le \int_{K}^{} \max |\varphi | |u|\le \max_k|\varphi|\int_{K}^{} |u|  \]  
È lineare:
\[T_u(\alpha\varphi+\beta\psi)=\int_{\Omega}^{} u(\alpha\varphi+\beta\psi)=\alpha \int_{\Omega}^{} u\varphi +\beta \int_{\Omega}^{}u\psi =\alpha T_u(\varphi)+\beta T_u(\psi)   \]
È continuo:
\[\{\varphi_h\} \to 0\text{ in }\mathcal D(\Omega)\implies T_u(\varphi_h)\to 0\]
Poiché, sia $\{\varphi_h\} \to 0\text{ in }\mathcal D(\Omega)$
\[|T_u(\varphi_h)|=\bigg|\int_{\Omega}^{} u\varphi_h \bigg|\le \max_K|\varphi_h|\cdot \int_{K}^{} |u|\to 0\]  
\\\textbf{Osservazioni sull'esempio} 
\\L'associazione tra $u,\ T_u$ è iniettiva su $L^{1}(\Omega)$
\\Se $u_1=u_2\text{ q.o. su }\Omega\implies T_{u_1}=T_{u_2}\text{ in }\mathcal D'(\Omega)$ poiché $T_{u_1}(\varphi)=T_{u_2}(\varphi)$
\\Si può dimostrare che $T_{u_1}=T_{u_2}\text{ in }\mathcal D'(\Omega)\implies u_1=u_2 \text{ q.o. su }\Omega$ (*)
\[\int_{\Omega}^{} u_1\varphi = \int_{\Omega}^{} u_2\varphi\ \forall \varphi\in \mathcal D(\Omega)\implies u_1=u_2\text{ q.o. su } \Omega\] 
\[\int_{\Omega}^{} (u_1-u_2)\varphi =0 \forall \varphi\in \mathcal D(\Omega)\implies u_1=u_2\text{ q.o. su } \Omega\] 
\divider
\begin{tcolorbox}
\textbf{Notazione:} Invece di $T_u(\varphi)$ si scrive spesso $<u,\varphi>_{(\mathcal D'(\Omega), \mathcal D(\Omega))}$
\end{tcolorbox}
Per definire $T_u$, basta una condizione più debole:
\[u\in L^1_{\text{loc}}(\Omega):=\{u:\Omega\to \R: \int_{K}^{} |u|<+\infty\ \forall K\text{ compatto }\subseteq  \Omega \}\]
\textbf{Esempio:} $\Omega=(0,1),\ u(x)=\frac{1}{x}\not\in L^1(\Omega)$ ma $u\in L^1_{\text{loc}}(\Omega)$
\\In particolare, possiamo associare una distribuzione a qualsiasi $u\in L^{p}(\Omega)\text{ con } p \in [1,+\infty]$ 
\\Infatti $L^{p}(\Omega)\not \subseteq  L^1(\Omega), $ ma 
\[L^{p}(\Omega)\subseteq  L^p_{\text{loc}}(\Omega),\ \forall  p \in [1,+\infty]:\]
\[u\in L^{p}(\Omega)\implies u\in L^{p}(K)\ \forall K \subset \subset \Omega\]
Poiché $|K|<+\infty$.
\[\implies u \in L^{1}(K)\ \forall K \subset  \subset \Omega \implies u\in L_{\text{loc}}^1(\Omega)\]
Tutte le funzioni $u\in L^{p}(\Omega)$ possono essere viste come distribuzioni.
\[u\in L^{p}(\Omega)\mapsto T_u\]
\[\left<u,\varphi \right>_{\mathcal D'(\Omega), \mathcal D(\Omega)}:=\int_{\Omega}^{} u\varphi dx\]
\divider\\
\\Essendo $\mathcal D'$ vettoriale
\[(T_1+T_2)(\varphi):=T_1(\varphi)+T_2(\varphi)\ \forall \varphi\in \mathcal D(\Omega)\]
\[(\lambda T):=\lambda T(\varphi) \ \forall \varphi \in \mathcal D(\Omega)\]
\subsection{Convergenza}
\begin{tcolorbox}
\textbf{Definizione: }
\[\{T_h\} \subseteq  \mathcal D'(\Omega),\ T_h\to^{\text{in }\mathcal D'(\Omega)}0\text{ se }T_h(\varphi)\to 0\ \forall \varphi\in \mathcal D(\Omega)\] 
\end{tcolorbox}
\[T_h\to T\text{ se }T_h(\varphi)\to T(\varphi)	\ \forall \varphi\in \mathcal D(\Omega)\]
\\\textbf{Esempio: }$T_h=T_{u_h}$, con $u_h\subseteq  L^{1}(\Omega)$ \\
\[u_h\to 0\text{ in }L^{1}(\Omega)\implies T_{u_h}\to 0\text{ in }\mathcal D'(\Omega)\]
Dato che
\[|T_{u_h}(\varphi)|=\bigg|\int_{\Omega}^{} u_h\varphi \bigg|\le \int_{K=\text{supp}\varphi}^{} |u_h| |\varphi|\le \max_K|\varphi|\cdot \int_{K}^{}  |u_h|\to 0\]
\subsection{Delta di Dirac}
\[\{u_h\} \subseteq  L^{1}(\R)\]
\begin{figure}[ht]
    \centering
    \incfig{dirac}
    \caption{Delta di Dirac}
    \label{fig:dirac}
\end{figure}
Questa successione non converge in $L^{1}(\R)$\\
$u_h\to 0$ q.o. su $\R\implies $ se $\exists \lim_{h \to +\infty} u_h$ in $L^{1}(\R)$ allora $\lim_{h \to +\infty} u_h=0$.
\\Ma $\lim_{h \to +\infty} u_h\neq 0$ in $L^{1}(\R)$ perché
\[\|u_h\|_{L^{1}(\R)}=\int_{-\frac{1}{2h}}^{\frac{1}{2h}} h=1\]
Converge però in $\mathcal D'(\R)$ 
\[\left<u_h,\varphi \right> =T_{u_h}(\varphi)=\int_{\R}^{} u_h\varphi=h \int_{-\frac{1}{2h}}^{\frac{1}{2h}} \varphi\to \varphi(0)\]  
\begin{tcolorbox}
	\textbf{Definizione: }$\delta_0$ \emph{delta di Dirach} in 0
	\[<\delta_0,\varphi>:=\varphi(0)\]
\end{tcolorbox}
\textbf{Osservazioni:} 
\begin{itemize}
	\item Se $u_h=h\cdot \chi_{[-\frac{1}{2h},\frac{1}{2h}]},$ allora $u_h\to \delta_0$ in $\mathcal D'(\R)$ 
	\item Verifica che $\delta_0\in \mathcal D'(\R)$
\end{itemize}
(i) lineare: $\delta_0(\alpha\varphi+\beta\psi)=\alpha\delta_0(\varphi)+\beta\delta_0(\psi)$?
\[(\alpha\varphi+\beta\psi)(0)=\alpha\varphi(0)+\beta\psi(0)\]
(ii) continuo: $\varphi_h\to 0$ in $\mathcal D(\R)\implies\delta_0(\varphi_h)\to 0$
\\vero per la definizione di convergenza in $\mathcal D(\R)$, $\text{supp}(\varphi_h)\subseteq  K$ compatto, $\varphi_h\to 0$ uniformemente.
\subsubsection{Ovvie generalizzazioni}
Punto generico $x_0$
\[\delta_{x_0}(\varphi)=\varphi(x_0)\]
Caso n-dimensionale, $x_0\in \R^n$
\[\delta_{x_0}(\varphi):=\varphi(x_0),\ \delta_{x_0}\in \mathcal D'(\R^n)\]
\subsubsection{idk}
$\delta_0$ non è associata ad alcuna funzione di $u\in L^1_{\text{loc}}(\Omega)$
\\\textbf{Dimostrazione} 
\\Supponiamo per assurdo $\delta_0=T_u,$ con $u\in L^1loc(R)$
\[\int_{\R}^{} u\varphi dx=\varphi(0)\ \forall \varphi \in \mathcal D(\R)\]
In particolare, posso prendere $\varphi \in \mathcal D(\R \setminus \{0\} )$ 
\[\int_{\R}^{} u\varphi=\varphi(0)=0\ \forall \varphi\in\mathcal D(\R\setminus \{ 0\} )\]
Ricordando l'osservazione sull'esempio 
\[0=\int_{\R}^{} 0\cdot \varphi\]
Applicando tale proprietà si avrà
\[u=0\text{ q.o. su }\R-\setminus \{ 0\} \implies u=0\text{ q.o. su }\R\]
\[\implies \int_{\R}^{} u\varphi dx=0\ \forall \varphi \in \mathcal D(\R)\]
Assurdo.
\subsubsection{Derivazione di distribuzioni}
\begin{tcolorbox}
\textbf{Definizione: }$\Omega \subseteq  \R$
\\Data $T\in \mathcal D'(\Omega)$, definisco $T' \in \mathcal D'(\Omega)$ come:
\[<T', \varphi>:=- <T,\varphi'>\ \ \forall \varphi \in \mathcal D(\Omega)\]

\end{tcolorbox}
$T'$ è una distribuzione
\\(i) è lineare:
\[<T',\alpha\varphi +\beta\psi> = -<T,\alpha \varphi'+\beta\psi '>\] \[= -\alpha <T,\varphi'>-\beta<T,\psi'> = +\alpha <T',\varphi>+\beta<T',\psi>\]
(ii) è continuo
\[\varphi_h\to 0\text{ in }\mathcal D(\Omega)\implies \left<T',\varphi_h \right>\to 0\]
Infatti $\varphi_h\to 0\text{ in }\mathcal D(\Omega)\implies\varphi_h'\to 0\text{ in }\mathcal D(\Omega)$.
\\Questo perché $\exists K$ tale che $\text{supp}\varphi'_h\subset K\forall h$ e $\varphi'_h\to 0$ uniformemente su $K$ con tutte le derivate.
\\Quindi $\left<T,\varphi'_h \right>\to 0$ perché $T\in \mathcal D'(\Omega)$
\subsubsection{Motivo della definizione di derivata}
Considerando il caso $T=T_u$ con $u\in C^1(\Omega)\subseteq  L^1_{\text{loc}}(\Omega)$
\\Si avrà in $\mathcal D'(\Omega)$ che $(T_u)'=T_{u'}$ 
\\\textbf{Dimostrazione} 
\[\left<(T_u)', \varphi \right> = -\left<T_u,\varphi' \right> =-\int_{\Omega}^{} u\varphi'\]
\[\left<T_{u'}, \varphi \right> = \int_{\Omega}^{} u'\varphi\]
Integrando per parti si avrà
\[\int_{\Omega}^{} u'\varphi = u\varphi - \int_{\Omega}^{} u\varphi'\]
Essendo $\varphi \in \mathcal D(\Omega)\implies u\varphi|_a^b=0$ (dove $a,b$ estremi di $\Omega$)
Si avrà:
\[\int_{\Omega}^{} u'\varphi=-\int_{\Omega}^{} u\varphi'\]  

\end{document}
