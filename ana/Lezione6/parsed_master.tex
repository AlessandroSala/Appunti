\begin{document}

\section{Teorema dei residui}
\textbf{Motivazione dello studio del teorema: }è il calcolo di integrali in campo complesso e anche in campo reale.
\\Se $f$ è olomorfa su $\Omega \subseteq  \C \implies \int_{\gamma}^{} f(z)dz=0$ dove $\gamma$ è un circuito omotopo a un punto.
Se $f$ è olomorfa su $\Omega$ tranne che in un numero finito di punti, come si calcola $\int_{\gamma}^{} f(z)dz$?
\begin{tcolorbox}
	\textbf{Definizione}\\ 
	Se $z_0$ è una singolarità isolata per $f$ si dice residuo di $f$ in $z_0$ il coefficiente $c_{-1}$ dello sviluppo in serie di Laurent di $f$ di centro $z_0$.
\end{tcolorbox}
\subsection{Calcolo dei residui}
\begin{itemize}
	\item Se $z_0$ è una singolarità eliminabile: $\text{Res}(f,z_0)=0$ poiché la parte singolare dello sviluppo $\equiv 0$
	\item $z_0$ singolarità essenziale: non c'è modo diretto di calcolare il residuo (serve calcolare lo sviluppo)
	\item Se $z_0$ è un polo di ordine $\nu$
\end{itemize}
\[\text{Res}(f,z_0)=\lim_{z \to z_0} \frac{1}{(\nu-1)!}D^{(\nu-1)}[(z-z_0)^\nu f(z)]\]
In particolare se $z_0$ è un polo semplice
\[\text{Res}(f,z_0)=\lim_{z \to z_0} [(z-z_0)f(z)]\]
\subsubsection*{Dimostrazione polo semplice}
$z_0$ polo semplice $\implies f(z)=\sum_{n\ge -1}^{} c_n(z-z_0)^n$, con $c_{-1}\neq  0$
\[(z-z_0)f(z)=\sum_{n\ge -1}^{} c_n(z-z_0)^{n+1}=c_{-1}+c_0(z-z_0)+c_1(z-z_0)^2+o(z-z_0)^2\]
\[\lim_{z \to z_0} [(z-z_0)f(z)]=c_{-1}\]
\textbf{Osservazione:} $\text{Res}(\frac{g}{h},z_0)=\frac{g(z_0)}{h'(z_0)}$ con $g$ olomorfa, $h$ con uno zero di ordine 1 in $z_0$.\\
\textbf{Dimostrazione}\\
Caso $g(z_0)\neq 0\implies z_0$ polo semplice 
\[(z-z_0) \frac{g(z)}{h(z)}= \frac{g(z_0)(z-z_0)+o(z-z_0)}{h'(z_0)(z-z_0)+o(z-z_0)}\to  \frac{g(z_0)}{h'(z_0)}\]
Tramite la formula per il residuo del polo semplice
\[\text{Res}( \frac{g}{h},z_0)= \frac{g(z_0)}{h'(z_0)}\]
Caso G(z0)=0
Dico che $z_0$ è una singolarità eliminabile
\[\frac{g}{h}= \frac{g'(z_0)(z-z_0)+o(z-z_0)}{h'(z_0(z-z_0)+o(z-z_0)}\to \frac{g'(z_0)}{h'(z_0)}\in\C\]
\subsection{Definizione e calcolo dell'indice di avvolgimento}
\begin{tcolorbox}
	\textbf{Definizione (intuitivia)}
	\\Sia $\gamma$ circuito $\subseteq \C$ e sia $z_0 \not\in \gamma$.
	\\Si dice indice di avvolgimento di $\gamma$ rispetto a $z_0$ è il numero di volte che $\gamma$ gira attorno a $z_0$, contate con segno + nel caso di verso antiorario
\end{tcolorbox}
\begin{tcolorbox}
	\textbf{Definizione (formale)} \\
	Sia $r(t):[a,b]\to \C$ parametrizzazione di $\gamma$ ($\gamma$) circuito $\subseteq \C,\ z_0 \not\in \C $.
	\\Sia $\rho (t):=|r(t)-z_0|$. Allora $\exists \theta:[a,b]\to \C$ tale che $r(t)=z_0+\rho(t)e^{i\theta(t)}$.
	\[\text{Ind}(\gamma,z_0):= \frac{\theta(b)-\theta(a)}{2\pi}\in\Z\]
\end{tcolorbox}
L'indice è un numero $\in\Z$ poiché $r(a)=r(b)\implies\rho(a)=|r(a)-z_0|=|r(b)-z_0|=\rho(b)$
\[r(a)=\rho (a)+e^{i\theta(a)}\]
\[r(b)=\rho(b)+e^{i\theta(b)}\]
\[\implies e^{i\theta(a)}=e^{i\theta(b)}\]
\[\implies i\theta(a)-i\theta(b)=2k\pi i = \theta(a)-\theta(b)=2k \pi\]
\textbf{Osservazioni}
\begin{enumerate}
	\item L'indice non cambia per parametrizzazioni equivalenti (dello stesso circuito)
	\item L'indice di avvolgimento non cambia sostituendo $\gamma$ con un circuito omotopo a $\gamma$ in $\C\setminus \{z_0\} $
\end{enumerate}
\subsubsection{Modalità analitica per calcolare l'indice}
\[\text{Ind}(\gamma, z_0)=\frac{1}{2\pi i} \int_{\gamma}^{} \frac{1}{z-z_0}dz\]
\textbf{Dimostrazione} 
\\$r(t)=z_0+\rho(t)e^{i\theta(t)},\ t\in[a,b]$.
\[\int_{\gamma}^{} \frac{1}{z-z_0}dz=\int_{a}^{b} \frac{\rho'(t)e^{i\theta(t)}+\rho(t)i\theta'(t)e^{i\theta(t)}}{z_0+\rho(t)e^{i\theta(t)}-z_0}dt\]
\[=\int_{a}^{b} \frac{\rho'(t)etsi\theta(t)}{\rho(t)e^{i\theta(t)}}dt+i \int_{a}^{ b} \frac{\rho(t)\theta'(t)e^{i\theta(t)}}{\rho(t) e^{i\theta(t)}}dt\]
\[=\log\rho(t)|_a^b+i[\theta(b)-\theta(a)]=i[\theta(b)-\theta(a)]\]
Dunque dividendo
\[\frac{1}{2\pi i}\int_{\gamma}^{} \frac{1}{z-z_0}dz=\frac{\theta(b)-\theta(a)}{2\pi}\]
\subsection{Teorema dei residui}
\begin{tcolorbox}
	Sia $\Omega$ aperto $\subseteq\C$ e sia  $\gamma \subseteq\C$ circuito omotopo a un punto (in $\Omega$).
	\\Sia $f:\Omega\setminus S\to \C$ olomorfa, dove $S$ "insieme singolare" soddisfa
	\begin{itemize}
		\item $\gamma \subseteq\Omega \setminus S$
		\item $\text{acc} (S)\cap\Omega = \emptyset$
	\end{itemize}
	Allora:
	\\$\text{Ind}(\gamma,z_0)\neq 0$ per al più un numero finito di punti e vale
	\[\int_{\gamma}^{} f(z)dz=2\pi i \sum_{z_0\in S}^{} \text{Res}(f,z_0)\text{Ind}(\gamma,z_0)\] 
\end{tcolorbox}

\end{document}
