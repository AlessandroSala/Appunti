
\section{Spazi di Lebesgue}
\begin{tcolorbox}
	\textbf{Definizione: }Sia  $E$  misurabile  $\subseteq\R^n$ 
	  \[L^1(E):= \{f:E\to  \R\     \text{L-integrabili} \} /_{\sim}\]  
\end{tcolorbox}
Tale insieme è uno spazio vettoriale per la linearità dell'integrale.
\begin{tcolorbox}
	\textbf{Definizione: }Data  $f \in  L ^1(E)$ 
  \[\|f\|_1:=\int_{E}^{} |f|\]  
\end{tcolorbox}
Tale norma rispetta le tre proprietà necessarie.
\\C'è un problema,  $\int_{E}^{} |f|=0\centernot\implies f=0$  su $E$,  $\implies f=0$  q.o. su  $E$.
\begin{tcolorbox}
	\textbf{Definizione: }Date  $f,g\in  L ^1(E)$  diciamo che  $f$  è equivalente a  $g$  se  $f=g$  q.o. su  $E$.
\end{tcolorbox}
Proprietà di una relazione di equivalenza:
\begin{itemize}
	\item $f\sim f$ 
	\item $f\sim g \iff g\sim f$ 
	\item $ f\sim g$ e $g\sim h\implies f\sim h$ 
\end{itemize}
Dunque identifichiamo le funzioni equivalenti secondo l'ultima definizione.
\begin{tcolorbox}
	\textbf{Teorema: }$(L^1(E),\|.\|)$  è uno spazio di Banach
\end{tcolorbox}
\begin{tcolorbox}
	\textbf{Definizione: }\\$\{f_n\} \subseteq L^1(E),f_n\to f$ in $L^1(E) \iff \lim_{n \to +\infty} \|f_n-f\|=0$ 
\end{tcolorbox}
ovvero
\[\lim_{n\to +\infty} \int_{E}^{} |f_n-f|dx=0\] 
Consideriamo per semplicità  $f=0$ 
\\Q: $f_n\to  0$  puntualmente q.o. su  E , allora  $\int_{E}^{} f_n=0$ ? 
\\Controesempio 1
\[\exists f_n\subseteq  L^1(\R): \begin{cases}
	f_n\to 0\text{ q.o. su } \R\\
	f_n \not\to \text{ in }L^1(\R)
\end{cases}
\]
\[f_n=\chi_{n, n+1}=\begin{cases}
	1\ \ x\in(n,n+1)
	\\0\ \ x \not\in (n,n+1)
\end{cases}\ \ n\in \N
\]
Fissato $x_0\in\R,\ f_(x_0)=0$ definitivamente (per $n\gg 1$)
\[\int_{\R}^{} \|f_n\|_{L^1(\R)}=\int_{\R}^{} \chi_{(n,n+1)}=1 \forall n\in\N\]
Controesempio 2
\[\exists f_n \subseteq L^1(0,1): \begin{cases}
	f_n\to 0 \text{ in }L^1(0,1)\\
	f_n \not\to 0\text{ q.o. su }(0,1)
\end{cases} 
\]
\begin{figure}[ht]
    \centering
    \incfig{disegno}
    \caption{Successione}
    \label{fig:disegno}
\end{figure}
$f_n\to 0$ in $L^1(0,1), \ \|f_n\|_{L^1(0,1)}=\int_{0}^{1} |f_n|\to 0 $
\\$f_n\not\to 0\forall x_0\in(0,1)$
\\Fissato $x_0\in(0,1), \exists\  K(n):f_{K(n)}(x_0)=1$ 
\begin{tcolorbox}
	\textbf{Proposizione:} Se $f_n\to 0$ in $L^1(E)$, allora $\exists f_{K(n)}\to 0$ q.o. su $E$.
\end{tcolorbox}
\textbf{Osservazioni:} 
\begin{itemize}
	\item Si può mettere $f$ al posto di 0.
	\item Nell'esempio è vero
	\item Conseguenza: Se una successione $f_n$ ammette limite in $L^1(E)$ allora questo limite deve coincidere col limtie puntuale q.o. 
\end{itemize}
Infatti, $f_n\to f$ q.o. in $L^{1}(E)$
Allora $\exists $ $f_{K(n)}  \to f$ q.o. su $E$. (per la proposizione)
\\Quindi se $f_n\to g$ q.o. su $E$. ($\implies f_{K(n)}\to g$ q.o. su $E$, per l'unicità del limite puntuale quasi ovunque, $f=g$ q.o. su $E$)
\begin{tcolorbox}
	\textbf{Teorema di convergenza dominata (di Lebesgue)}
	\\Sia $\{f_n\} \subseteq  L^1(E)$ e sia $f_n\to f$ q.o. su $E$\\
	Supponiamo che $\exists g\in L^1(E)$ indipendente da $n$ tale che
	\[(*)\ |f_n(x)|\le g(x)\text{ q.o. }x\in E, \forall n \in \N\text{ (definitivamente)}\]
	Allora $f_n\to f$ in $L^1(E)$
\end{tcolorbox}
\textbf{Osservazioni}
\begin{itemize}
	\item La (*) è un'ipotesi molto più debole della convergenza uniforme
	\item In particolare per $f_n(x)=x^n$ su $(0,1)$ la (*) è verificata, prendendo $g\equiv 1$
	\item Invece nel controesempio 1, se $|f_n(x)|\le g$ q.o. su $\R,g \not\in L^1(\R)$
	\item È un teorema di passaggio al limite sotto integrale.
\end{itemize}
\[|f_n-f|\to 0\text{ su }E\implies \int_{E}^{} |f_n-f|\to 0\]
$\implies$ il limite degli integrali $\equiv$ l'integrale del limite.
\begin{tcolorbox}
	\textbf{Teorema di convergenza monotona (di Beppo Levi)}
	\\Sia $\{f_n\} \subseteq L^1(E)$, supponiamo che:
	\[(* *)\ f_n\ge 0\text{ q.o. su }E,\ f_{n+1}\ge f_n\text{q.o. su }E\]
	Allora
	\[\int_{E}^{} \lim_{n \to +\infty} f_ndx=\lim_{n \to +\infty} \int_{E}^{} f_n\]  
\end{tcolorbox}
\textbf{Osservazioni} 
\begin{itemize}
	\item Il teorema si applica anche se $f_n\le 0$ decrescente, basta considerare $g_n=-f_n\ge 0$
\end{itemize}
\[\text{B.L. a }g_n\implies \int_{E}^{} \lim g_n=\lim\int_{E}^{}g_n=\int_{E}^{} \lim(-f_n)=\lim \int_{E}^{} (-f_n)\]
\begin{itemize}
	\item Può valere come uguaglianza $+\infty=+\infty$
\end{itemize}
\subsubsection{Integrali multipli}
\begin{tcolorbox}
	\textbf{Teorema di Fubini} 
	\\Sia $f$ integrabile secondo Lebesgue, su $I=I_1\times I_2\ (I_1\subseteq  \R^m, I_2\subseteq\R^n)$
	Allora:
	\begin{enumerate}
		\item Per q.o. $x_1\in I_1,\ x_2\mapsto f(x_1,x_2)$ è L-integrabile su $I_2$
		\item $x_1\mapsto \int_{I_2}^{} f(x_1,x_2)dx_2$ L-integrabile su $I_1$
		\item $\int_{I}^{} f(x_1,x_2)dx_1dx_2=\int_{I_1}^{} ( \int_{I_2}^{} f(x_1,x_2)dx_2)dx_1$  
	\end{enumerate}
\end{tcolorbox}
\textbf{Osservazione:} Si può scambiare il ruolo delle variabili.
\[\int_{I}^{} f(x_1,x_2)dx_1dx_2=\int_{I_1}^{} \bigg(\int_{I_2}^{} f(x_1,x_2)dx_2\bigg)dx_1=\int_{I_2}^{} \bigg(\int_{I_1}^{} f(x_1,x_2)dx_1\bigg)dx_2\]
\begin{tcolorbox}
	\textbf{Teorema di Tonelli} 
	\\Sia $f\ge 0$ misurabile sul precedente $I=I_1\times I_2$.
	\\Supponiamo che:
	\begin{itemize}	
		\item Per q.o. $x_1\in I_1,\ x_2\mapsto f(x_1,x_2)$ è L-integrabile su $I_2$
		\item $x_1\mapsto \int_{I_2}^{} f(x_1,x_2)dx_2$ L-integrabile su $I_1$
	\end{itemize}
	Allora: $f$ L-integrabile su $I_1\times I_2$ (e quindi per Fubini $\int_{I}^{} f=\int_{I_1}^{} \int_{I_2}^{} f$)   
\end{tcolorbox}
\textbf{Osservazione:} Se ho una $f$ che cambia segno, posso provare ad applicare Tonelli a $|f|$: se $|f|$ soddisfa 1) 2), Tonelli $\implies |f|$ L-integrabile $\implies$ $f$ L-integrabile $\implies$ posso applicare Fubini. 
\subsubsection{Spazi di Lebesgue (o spazi $L^p$)}
\begin{tcolorbox}
	\textbf{Definizione:} $p\in[1,+\infty)$, $L^p(E)=\{f:E\to \R:|f|^p\text{ L-integrabile}\}/_{\sim}$, anch'esso risulta essere uno spazio vettoriale normato (di Banach)
	\[\|f\|_p:=\bigg(\int_{E}^{} |f(x)|^pdx\bigg)^{\frac{1}{p}}\] 
\end{tcolorbox}
\begin{tcolorbox}
	\textbf{Teorema:} $(L^p,\|.\|_p)$ è uno spazio di Banach.
\end{tcolorbox}
\begin{itemize}
	\item Caso particolarmente importante: $p=2$
	\item Caso limite: $p=+\infty$
\end{itemize}
