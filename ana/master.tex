
\section{Introduzione Analisi Complessa}
\textbf{Definizione}
\\Una funzione di variabile complessa è una funzione $f:\Omega\subseteq \mathbb{C}\to \mathbb{C}$
\\Numero complesso: $z=x+iy$ dove $x,y\in \mathbb{R}\ i^{2}=-1$
\\$f(z)=u(z)+iv(z)=u(x,y)+iv(x,y)$
\\Dunque ad ogni funzione complessa è possibile associare due funzioni reali in due variabili
\\$f\leftrightarrow u,v:\Omega\subseteq \mathbb{R}^2\to \mathbb{R}$
\\\\\textbf{Esempi di funzioni elementari}
\[
	f(z)=z_0\in\mathbb{C},\ z_0=x_0+iy_0,\ u=x_0,\  v=y_0
\]
\[
	f(z)=Rez,\  z=x+iy,\  u(x,y)=x
	,\ v(x,y)=0
\]
\[
	f(z)=Imz,\ z=x+iy\implies f(z)=y
\]
\[
	f(z)=|z|,\ f(z)=\sqrt{x^2+y^2} \implies u(x,y)=\sqrt{x^2+y^2} , \ v(x,y)=0
\]
\[
	f(z)=P(z)=a_nz^n+\ldots+a_1z+a_0
\]
In questo ultimo caso è necessario calcolare manualmente le funzioni $u,v$ associate
\[
	f(z)=\frac{P(z)}{Q(z)} \text{ con } P,Q \text{ funzioni polinomiali }
\]
Quest'ultima funzione non è definita $\forall z \in \mathbb{C}$
\[
	f(z)=e^z, \sin z, \cos z, \sinh z, \cosh z
\]
\\\textbf{Funzione Esponenziale}
\\È l'estensione della funzione esponenziale nel campo dei reali
\[
	z=x+iy\implies e^z:=e^x\cdot e^{iy}:=e^x\cdot(\cos y+i\sin y)
\]
Se $z=x\in R \implies e^z=e^x\implies$ è estensione della funzione reale
\\$u(x,y)=e^x \cos y,\ v(x,y)=e^x\sin y $
\[ e^{z+2\pi i}=e ^{x+iy+2\pi i}=e ^{x + i(y +2 \pi }=e^x(\cos (y+2 \pi)+i \sin (y+2 \pi))=e^z\]
Questo implica che la funzione esponenziale nel campo complesso è periodica di periodo $T=2 \pi i$
Inoltre,
\[e^{z_1+z_2}=e^{z_1}e^{z_2}
\]
\[
	e^z=0\iff|e^z|=0
\]
\[	
	|e^x(\cos y + i \sin y)|=|e^x| |\cos y +i\sin y|=e^x=0\ \cancel\forall z  
\]
\\\textbf{Funzioni coseno, seno}
\[\cos z:=\frac{1}{2}(e^{iz}+e^{-iz})\]
\[\sin z:=\frac{1}{2i}(e^{iz}-e^{-iz})\]
Sono funzioni definite $\forall z\in \mathbb C$
\\Come per la funzione esponenziale sono estensioni delle funzioni reali
\\Dunque se $z=x\in R \implies \cos z = \cos x$
\[\cos(z+2\pi)=\cos z\]
$\implies \cos z$ è periodica sia nel campo dei reali sia nel campo dei complessi
\\Inoltre
\[\cos z =0\iff z=\frac{\pi}{2}+k\pi,k\in Z\]
Per le altre funzioni valgono proprietà analoghe essendo definite come estensioni
\\\textbf{Formule alternative}


\section{Limiti}
\textbf{Definizione: } \[f:\Omega\subseteq\C\to \C, z_0\in \text{acc}(\Omega),l\in\C\]
\[\lim_{z \to z_0} f(z)=l\iff\forall V(l)\exists
u(z_0) \text{ tc }\forall z\in (u(z_0) \cap \Omega\setminus\{z_0\}), f(z)\in V(l)\]
\textbf{Definizione di continuità}
\[f:\Omega\subseteq\C\to \C, z_0\in a(\Omega)\cap \Omega,\  f \text{ continua } z_0 \iff\exists \lim_{z \to z_0} f(z)=f(z_0)\]
\textbf{Osservazioni}
\\$z=x+iy, z_0=x_0+iy_0, l=l_1+il_2, f=u+iv$
\[\lim_{z \to z_0} f(z)=l\iff 
\begin{cases}	
	\lim_{(x,y) \to (x_0,y_0)} u(x,y)=l_1\\ 
	\lim_{(x,y) \to (x_0,y_0)} v(x,y)=l_2
\end{cases}
\]
Con le stesse notazioni
\[f \text{ continua in }z_0\iff u,v \text{ continue in } (x_0,y_0)\]
Sono continue (sul loro dominio di definizione) tutte le funzioni elementari introdotte nella lezione scorsa
\\Vale l'algebra dei limti e il teorema del limite della funzione composta ($\implies$ composizione di continue rimane continua)
\\Vale il teorema di unicità del limite
\subsubsection{Infinito nei complessi}
Un intorno di $\infty$ nei complessi è il complementare di un qualsiasi disco
\[z\to \infty \iff z\in u(\infty)\iff |z|>R\iff |z|\to +\infty\]
\[f(z)\to \infty \iff f(z)\in u(\infty) \iff |f(z)|>R \iff |f(z)|\to +\infty\]

\subsection{Derivabilità}
\textbf{Definizione: }
\[f:\Omega \subseteq \C\to \C, z_0\in \text{acc}(\Omega)\cap \Omega\]
\[f \text{ derivabile (in senso complesso) in } z_0 \iff\exists \lim_{z \to z_0} \frac{f(z)-f(z_0)}{z-z_0}(\in\C)\]
e tale limite si dice $f'(z_0)$ 
\\Definizioni alternative:
\[\exists \lim_{h \to 0} \frac{f(z_0+h)-f(z_0)}{h}\ldots\]
\[f(z_0+h)=f(z_0)+\lambda\cdot h + o(h)\ \ \ \frac{o(h)}{h}\to 0\]
Dove $\lambda$ è la derivata prima della funzione nel punto $z_0$
\\Quest'ultima è la definizione di differenziabilità
\\\textbf{Attenzione} Se $u,v$ sono differenziabili ciò non implica la differenziabilità/derivabilità di $f$ (Un esempio è $f(z)=Imz$)
\begin{tcolorbox}
\textbf{Teorema (caratterizzazione della derivaibilità)}
\[f:\Omega \subseteq \C\to \C, z_0\in \Omega \cap \text{acc}(\Omega),\ z_0=x_0+iy_0,\ f=u+iv\]
\[f \text{ derivabile in }z_0\iff u,v \text{ differenziabile in } (x_0,y_0),\]
\[\text{Inoltre } \begin{cases}	
u_x(x_0,y_0)=v_y(x_0,y_0)
\\u_y(x_0,y_0)=-v_x(x_0,y_0)
\end{cases}
\]
(sistema, condizione di cauchy riemann)
Inoltre, in tal caso 
\[f'(z_0)=u_x(x_0,y_0)-iu_y(x_0,y_0)\]
\[f'(z_0)=v_y(x_0,y_0)+iv_x(x_0,y_0)\]

\end{tcolorbox}
\textbf{Dimostrazione}
\[(\implies) \text{Per Hp } \exists f'(z_0)=\alpha +i\beta\in\C\implies f(z_0+h)=f(z_0)+f'(z_0)h+g,\ g=o(h)\]
\[
	u(x_0+h_1,y_0+h_1)+iv(x_0+h_1,y_0+h_2)=u(x_0,y_0)+iv(x_0,y_0)+(\alpha+i\beta)(h_1+ih_2)+g_1+ig_2
\]
\[u(x_0+h_1+h_1,y_0+h_2)=u(x_0,y_0)+(\alpha h_1-\beta h_2)+g_1\]
\[v(x_0+h_1,y_0+h_2)=v(x_0,y_0)+(\beta h_1+\alpha h_2)+g_2\]
Queste due equazioni indicanon che $u,v$ sono differenziabili in $(x_0,y_0)$, con 
\[
	\begin{cases}
\nabla u(x_0,y_0)=(\alpha, -\beta)\\ 
\nabla v(x_0,y_0)=(\beta, \alpha)
	\end{cases}
\]
Questo dimostra inoltre che $f'(z_0)=\alpha+i\beta$
\\$(\impliedby)$ Procedere al contrario














\section{Spazi di Lebesgue}
\begin{tcolorbox}
	\textbf{Definizione: }Sia  $E$  misurabile  $\subseteq\R^n$ 
	  \[L^1(E):= \{f:E\to  \R    \text{L-integrabili} \} /_{\sim}\]  
\end{tcolorbox}
Tale insieme è uno spazio vettoriale per la linearità dell'integrale.
\begin{tcolorbox}
	\textbf{Definizione: }Data  $f \in  L ^1(E)$ 
  \[\|f\|_1:=\int_{E}^{} |f|\]  
\end{tcolorbox}
Tale norma rispetta le tre proprietà necessarie.
\\C'è un problema,  $\int_{E}^{} |f|=0\centernot\implies f=0$  su $E$,  $\implies f=0$  q.o. su  $E$.
\begin{tcolorbox}
	\textbf{Definizione: }Date  $f,g\in  L ^1(E)$  diciamo che  $f$  è equivalente a  $g$  se  $f=g$  q.o. su  $E$.
\end{tcolorbox}
Proprietà di una relazione di equivalenza:
\begin{itemize}
	\item $f\sim f$ 
	\item $f\sim g \iff g\sim f$ 
	\item $ f\sim g$ e $g\sim h\implies f\sim h$ 
\end{itemize}
Dunque identifichiamo le funzioni equivalenti secondo l'ultima definizione.
\begin{tcolorbox}
	\textbf{Teorema: }$(L^1(E),\|.\|)$  è uno spazio di Banach
\end{tcolorbox}
\begin{tcolorbox}
	\textbf{Definizione: }\\$\{f_n\} \subseteq L^1(E),f_n\to f$ in $L^1(E) \iff \lim_{n \to +\infty} \|f_n-f\|=0$ 
\end{tcolorbox}
ovvero
\[\lim_{n\to +\infty} \int_{E}^{} |f_n-f|dx=0\] 
Consideriamo per semplicità  $f=0$ 
\\Q: $f_n\to  0$  puntualmente q.o. su  E , allora  $\int_{E}^{} f_n=0$ ? 
\\Controesempio 1
\[\exists f_n\subseteq  L^1(\R): \begin{cases}
	f_n\to 0\text{ q.o. su } \R\\
	f_n \not\to \text{ in }L^1(\R)
\end{cases}
\]
\[f_n=\chi_{n, n+1}=\begin{cases}
	1\ \ x\in(n,n+1)
	\\0\ \ x \not\in (n,n+1)
\end{cases}\ \ n\in \N
\]
Fissato $x_0\in\R,\ f_(x_0)=0$ definitivamente (per $n\gg 1$)
\[\int_{\R}^{} \|f_n\|_{L^1(\R)}=\int_{\R}^{} \chi_{(n,n+1)}=1 \forall n\in\N\]
Controesempio 2
\[\exists f_n \subseteq L^1(0,1): \begin{cases}
	f_n\to 0 \text{ in }L^1(0,1)\\
	f_n \not\to 0\text{ q.o. su }(0,1)
\end{cases} 
\]
\begin{figure}[ht]
    \centering
    \incfig{disegno}
    \caption{Successione}
    \label{fig:disegno}
\end{figure}
$f_n\to 0$ in $L^1(0,1), \ \|f_n\|_{L^1(0,1)}=\int_{0}^{1} |f_n|\to 0 $
\\$f_n\not\to 0\forall x_0\in(0,1)$
\\Fissato $x_0\in(0,1), \exists\  K(n):f_{K(n)}(x_0)=1$ 
\begin{tcolorbox}
	\textbf{Proposizione:} Se $f_n\to 0$ in $L^1(E)$, allora $\exists f_{K(n)}\to 0$ q.o. su $E$.
\end{tcolorbox}
\textbf{Osservazioni:} 
\begin{itemize}
	\item Si può mettere $f$ al posto di 0.
	\item Nell'esempio è vero
	\item Conseguenza: Se una successione $f_n$ ammette limite in $L^1(E)$ allora questo limite deve coincidere col limtie puntuale q.o. 
\end{itemize}
Infatti, $f_n\to f$ q.o. su $E$  finire
\\Quindi se $f_n\to g$ q.o. su $E$. ($\implies f_{K(n)}\to g$ q.o. su $E$, per l'unicità del limite puntuale quasi ovunque, $f=g$ q.o. su $E$)
\begin{tcolorbox}
	\textbf{Teorema di convergenza dominata (di Lebesgue)}
	\\Sia $\{f_n\} \subseteq  L^1(E)$ e sia $f_n\to f$ q.o. su $E$\\
	Supponiamo che $\exists g\in L^1(E)$ indipendente da $n$ tale che
	\[(*)\ |f_n(x)|\le g(x)\text{ q.o. }x\in E, \forall n \in \N\text{ (definitivamente)}\]
	Allora $f_n\to f$ in $L^1(E)$
\end{tcolorbox}
\textbf{Osservazioni}
\begin{itemize}
	\item La (*) è un'ipotesi molto più debole della convergenza uniforme
	\item In particolare per $f_n(x)=x^n$ su $(0,1)$ la (*) è verificata, prendendo $g\equiv 1$
	\item Invece nel controesempio 1, se $|f_n(x)|\le g$ q.o. su $\R,g \not\in L^1(\R)$
	\item È un teorema di passaggio al limite sotto integrale.
\end{itemize}
\[|f_n-f|\to 0\text{ su }E\implies \int_{E}^{} |f_n-f|\to 0\]
$\implies$ il limite degli integrali $\equiv$ l'integrale del limite.
\begin{tcolorbox}
	\textbf{Teorema di convergenza monotona (di Beppo Levi)}
	\\Sia $\{f_n\} \subseteq L^1(E)$, supponiamo che:
	\[(* *)\ f_n\ge 0\text{ q.o. su }E,\ f_{n+1}\ge f_n\text{q.o. su }E\]
	Allora
	\[\int_{E}^{} \lim_{n \to +\infty} f_ndx=\lim_{n \to +\infty} \int_{E}^{} f_n\]  
\end{tcolorbox}
\textbf{Osservazioni} 
\begin{itemize}
	\item Il teorema si applica anche se $f_n\le 0$ decrescente, basta considerare $g_n=-f_n\ge 0$
\end{itemize}
\[\text{B.L. a }g_n\implies \int_{E}^{} \lim g_n=\lim\int_{E}^{}g_n=\int_{E}^{} \lim(-f_n)=\lim \int_{E}^{} (-f_n)\]
\begin{itemize}
	\item Può valere come uguaglianza $+\infty=+\infty$
\end{itemize}
\subsubsection{Integrali multipli}
\begin{tcolorbox}
	\textbf{Teorema di Fubini} 
	\\Sia $f$ integrabile secondo Lebesgue, su $I=I_1\times I_2\ (I_1\subseteq  \R^m, I_2\subseteq\R^n)$
	Allora:
	\begin{enumerate}
		\item Per q.o. $x_1\in I_1,\ x_2\mapsto f(x_1,x_2)$ è L-integrabile su $I_2$
		\item $x_1\mapsto \int_{I_2}^{} f(x_1,x_2)dx_2$ L-integrabile su $I_1$
		\item $\int_{I}^{} f(x_1,x_2)dx_1dx_2=\int_{I_1}^{} ( \int_{I_2}^{} f(x_1,x_2)dx_2)dx_1$  
	\end{enumerate}
\end{tcolorbox}
\textbf{Osservazione:} Si può scambiare il ruolo delle variabili.
\[\int_{I}^{} f(x_1,x_2)dx_1dx_2=\int_{I_1}^{} \bigg(\int_{I_2}^{} f(x_1,x_2)dx_2\bigg)dx_1=\int_{I_2}^{} \bigg(\int_{I_1}^{} f(x_1,x_2)dx_1\bigg)dx_2\]
\begin{tcolorbox}
	\textbf{Teorema di Tonelli} 
	\\Sia $f\ge 0$ misurabile sul precedente $I=I_1\times I_2$.
	\\Supponiamo che:
	\begin{itemize}	
		\item Per q.o. $x_1\in I_1,\ x_2\mapsto f(x_1,x_2)$ è L-integrabile su $I_2$
		\item $x_1\mapsto \int_{I_2}^{} f(x_1,x_2)dx_2$ L-integrabile su $I_1$
	\end{itemize}
	Allora: $f$ L-integrabile su $I_1\times I_2$ (e quindi per Fubini $\int_{I}^{} f=\int_{I_1}^{} \int_{I_2}^{} f$)   
\end{tcolorbox}
\textbf{Osservazione:} Se ho una $f$ che cambia segno, posso provare ad applicare Tonelli a $|f|$: se $|f|$ soddisfa 1) 2), Tonelli $\implies |f|$ L-integrabile $\implies$ $f$ L-integrabile $\implies$ posso applicare Fubini. 
\subsubsection{Spazi di Lebesgue (o spazi $L^p$)}
\begin{tcolorbox}
	\textbf{Definizione:} $p\in[1,+\infty)$, $L^p(E)=\{f:E\to \R:|f|^p\text{ L-integrabile}\}/_{\sim}$, anch'esso risulta essere uno spazio vettoriale normato (di Banach)
	\[\|f\|_p:=\bigg(\int_{E}^{} |f(x)|^pdx\bigg)^{\frac{1}{p}}\] 
\end{tcolorbox}
\begin{tcolorbox}
	\textbf{Teorema:} $(L^p,\|.\|_p)$ è uno spazio di Banach.
\end{tcolorbox}
\begin{itemize}
	\item Caso particolarmente importante: $p=2$
	\item Caso limite: $p=+\infty$
\end{itemize}

\begin{itemize}
	\item Definizione e teoremi di completezza
	\item Criteri di convergenza
	\item Risultati di confronto
	\item Approssimazione con funzioni regolari (prodotto di convoluzione)
	\item Teorema di differenziazione (funzioni assolutamente continue)
\end{itemize}
Appartenenza a $L^p$: verifica dell'integrale
\[f\in L^p(E) \iff \int_{E}^{} |f|^p<+\infty\]
Convergenza in $L^p$
\[\{f_n\} \subseteq L^p(E),\ f\in L^p(E),\  \bigg(\int_{E}^{} |f_n-f|^p\bigg)^{\frac{1}{p}}\to 0\]
Candidato limite: $f$ limite puntuale q.o.
\[\lim_n \int_E|f_n-f|^p=\int_{E}^{} \lim |f_n-f|^p \ ?\] 
\subsubsection{Caso limite L infinito}
\begin{tcolorbox}
	\textbf{Definizione:} \[L^\infty:=\{f:E\to \R\text{ misurabili}:\text{ ess-}\sup_{x\in E}|f(x)|<+\infty\}/_\sim \]

\end{tcolorbox}	
\[\sup_{x\in E}|f(x)|:=\min \{M:|f(x)|\le M\ \forall x\in E\}\]
\[\text{ess-}\sup_{x\in E}|f(x)|:=\min \{M:|f(x)|\le M\text{ q.o. } x\in E\}\]
\begin{tcolorbox}
	\textbf{Teorema:} $(L^\infty(E),\|.\|_{\infty})$ è uno spazio di Banach
\end{tcolorbox}
\textbf{Osservazioni} 
\[f\in L^\infty(E) \iff\text{ess-sup}_{x\in E}|f(x)|<+\infty\]
Convergenza
\[\{f_n\} \subseteq L^\infty(E),\ f\in L^\infty(E): \text{ess-sup}_{x\in E}|f_n(x)-f(z)|\to 0\]
Dunque convergenza uniforme a meno di un insieme di misura nulla.
\\\textbf{Esempi} di funzioni in $L^\infty(\R)$
\[f(x)=c>0,\ \|f\|_\infty=c\]
\[f(x)=\begin{cases}
	1\ \ \ x\not\in \N
	\\n\ \ \ x=n\in \N
\end{cases}
\]
\textbf{Osservazione:} Se $f\in L^p(E),\ \forall p\in[1,+\infty]$
\[\implies \lim_{p \to +\infty} \|f\|_{L^p(E)}=\|f\|_{L^\infty(E)}\]
Analogo in $\R^2$
\[\lim_{p \to +\infty} \|x\|_p=\lim_{p \to +\infty} (|x_1|^p+|x_2|^p)^{\frac{1}{p}}=\max \{|x_1|,|x_2|\}\] 
\subsection{Risultati di confronto}
$p\le q\ p,q\in[1,+\infty]\implies  L^p(E)\subseteq \text{ oppure }\supseteq L^q(E)?$
\\In generale no
\\\textbf{Controesempio 1:} $L^1(0,+\infty), \ L^2(o,+\infty),L^\infty(0,+\infty)$
\[f(x)=1,\ \text{ess-sup}_{x\in\R}|f(x)|=1,\ \int_{\R_+}^{} |f|=\int_{\R_+}^{} |f|^2=+\infty\]  
\[f\in L^{\infty}(\R_+)\text{ ma }f\not\in L^1(\R_+),f\not\in L^2(\R_+)\]
\textbf{Controesempio 2:} 
\begin{figure}[ht]
    \centering
    \incfig{esempio2}
    \caption{Controesempio 2}
    \label{fig:esempio2}
\end{figure}
\[\int_{0}^{+\infty} |f|=\int_{0}^{1} \frac{1}{\sqrt{x} }<+\infty,\ \text{ess-sup}_{x\in\R_+}|f(x)|=+\infty=\int_{0}^{+\infty}|f|^2\]
\[f\in L^1(\R_+)\text{ ma }f\not\in L^\infty(\R_+),\ f\not\in L^2(\R_+)\]
\textbf{Controesempio 3:} 
\begin{figure}[ht]
    \centering
    \incfig{controesempio3}
    \caption{Controesempio 3}
    \label{fig:controesempio3}
\end{figure}
Si ricava in modo immediato che 
\[f\in L^2(\R_+)\text{ ma }f\not\in L^\infty(\R_+),\ f\not\in L^1(\R_+)\]
\subsubsection{Disuguaglianza di Holder}
\begin{tcolorbox}
	Sia $E$ misurabile $\subseteq \R^n$ qualsiasi, e $p \in [1,+\infty]$.
	\\Siano $f\in L^p(E),\ g\in L^{p'}(E)$, con $p':=$ esponente coniugato di $p$
	\[\frac{1}{p}+\frac{1}{p'}=1\]
	Con la convenzione $\frac{1}{\infty}=0$ 
\end{tcolorbox}
\begin{tcolorbox}
	\textbf{Disuguaglianza di Holder: }Sia $f\in L^p(E),\ g\in L^{p'}(E)$
	\[\|f\cdot g\|_1\le \|f\|_p\|g\|_{p'}\]
\end{tcolorbox}
\subsubsection{Conseguenze di Holder sul confronto tra i vari spazi}
\textbf{Proprietà di immersione (1)}\\ 
Sia $E\subseteq \R^n$ con $m(E)<+\infty$ e sia $q\ge p$, allora $L^q(E)\subseteq L^p(E)$, e 
\[\|f\|_{L^p(E)}\le m(E)^{\frac{q-p}{qp}}\ \|f\|_{L^q(E)}\ \ \forall f\in L^q(E)\]
In particolare se $q=+\infty$ ho che $\forall p\in[1,+\infty),L^\infty(E)\subseteq L^p(E)$
\[\|f\|_{L^p(E)}\le m(E)^{1 / p}\|f\|_{L^\infty(E)}\]
Infatti
\[\int_{E}^{} |f|^p\le \int_{E}^{} \text{ess-sup}_{x\in E}|f|^p=m(E)\cdot (\text{ess-sup}_{x\in E}|f|)^p\]
Elevando a $\frac{1}{p}$
\[\|f\|_p=\bigg(\int_{E}{|f|^p\bigg)^{\frac{1}{p}}}\le (m(E))^{\frac{1}{p}}\text{ess-sup}_{x\in E}|f|=m(E)^{1 / p}\|f\|_{L^\infty(E)}  \]
\textbf{Dimostrazione di (*) a partire da Holder}\\
Suppongo $f\in L^q(E)$, $\implies f\in L^{\frac{q}{p}}$
\[\int_{E}^{} |f|^p=\int_{E}^{} |f|^p\chi_E \le \|f^p\|_{L^{\frac{q}{p}}}\cdot \|\chi_E\|_{L^{(q / p)'}}\]
\begin{itemize}
	\item $f\in L^{q / p}$ infatti 
	\item $\chi_E\in L^{q / p )'}$ infatti
\end{itemize}
\[\int_{E}^{} |\chi_E|^{(q / p)'}=\bigg(\frac{q}{p}\bigg)'=\frac{q}{q-p}\]
\[\ \| |f|^p\|_{L^{q / p}}=\bigg(\int_{E}^{} |f|^q\bigg)^{\frac{p}{q}}\]
\[\|\chi_E\|_{L^{(q / p)'}}=\bigg( \int_{E}^{} |\chi_E|^{(\frac{q}{p})'}\bigg)^{\frac{1}{\frac{q}{p})'}}=m(E)^{\frac{q-p}{q}}\]
Quindi
\[\int_{E}^{} |f|^p\le m(E)^{\frac{q-p}{q}}\cdot \bigg(\int_{E}^{} |f|^q\bigg)^{\frac{p}{q}}\]
Elevando tutto alla $\frac{1}{p}$ 
\[\|f\|_{L^p(E)}\le m(E)^{\frac{q-p}{pq}}\cdot \|f\|_{L^{q}(E)}\]
\textbf{Proprietà di interpolazione (2)}\\
Se $f\in L^p(E)\cap L^q(E)$, con $p\le q\implies f\in L^r(E)$ $\forall  r\in [p,q]$ e
\[\|f\|_{L^r(E)}\le \|f\|_{L^p(E)}^\alpha\cdot \|f\|_{L^{q}(E)}^{1-\alpha}\]
Dove $\alpha\in (0,1)$ tale che $\frac{1}{r}=\frac{\alpha}{p}+\frac{1-\alpha}{q}$\\ 
\textbf{Esempio:} Se $f\in L^{1}(E)\cap L^{\infty}(E)\implies f\in L^{r}(E)\forall r\in [1,+\infty]$
\subsection{Approssimazione con funzioni regolari}
\begin{tcolorbox}
	\textbf{Teorema di approssimazione con funzioni regolari}\\
	Sia $p\in[1,+\infty)$, e sia $E$ misurabile in $\R^n$
	\\$C_o^\infty(E)$ è un sottospazio \textit{denso} in $L^p(E)$
	\\$C_0^\infty(E):=\{f:E\to \R\text{ di classe }C^\infty \text{ e aventi supporto compatto in }E\}$ 
\end{tcolorbox}
Ovvero 
\[\forall f\in L^p(E) \exists \{f_n\} \subseteq C_0^\infty(E)\text{ tale che }\|f_n-f\|_{L^p}\to_{n\to +\infty}0\]
\[\forall f\in L^p(E),\ \forall \varepsilon>0 \exists \varphi\in C_0^\infty(E)\text{ tale che }\|\varphi-f\|_{L^p}<\varepsilon\]
\textbf{Osservazione:} Falso nel caso $p=+\infty$



\subsection{Supporto e Classe $C_0$}
Ovvero:
\[\forall f \in L^{p}(E)\ \exists \{ \varphi_n\}	\subseteq  C_0^\infty(E) \text{ tale che } \|\varphi_n -\varphi\|\to _{n\to +\infty}0\]
\[\forall f\in L^{p}(E),\forall \varepsilon>0\ \exists \varphi\in C_0^\infty(E)\text{ tale che }\|f-\varphi\|_{L^{p}}<\varepsilon
\] 
Falso nel caso $p=\infty$
\begin{tcolorbox}
	\textbf{Definizione:} Data $\varphi\in C^\infty(E)$ il supporto di $\varphi$ è 
	\[\overline{\{x\in E:\varphi(x)\neq 0\}} \]
\end{tcolorbox}
\begin{tcolorbox}
	Un insieme $K\subseteq  \R^n$ è compatto se e solo se è limitato e chiuso
\end{tcolorbox}
\begin{tcolorbox}
	$C_0^\infty(E):=\{\varphi :E\to \R$ derivabili infinite volte tali che supp$(\varphi)$ è un sottoinsieme compatto di $E\} $
\end{tcolorbox}
\subsection{Prodotto di convoluzione}
\textbf{Osservazione: }$f,g\in L^{1}(E)\centernot\implies f\cdot g\in L^{1}(E)$
\\Nel caso $E=\R$ si può definire un prodotto che rimanga interno a $L^{1}(\R)$ 
\begin{tcolorbox}
	\textbf{Proposizione 1} 
	\\Siano $f,g(x)\in L^{1}(\R)$ Si definisce prodotto di convoluzione
	\[ f*g:=\int_{\R_y}^{} f(x-y)g(y)dy\] 
\end{tcolorbox}
\begin{enumerate}
	\item $f*g$ esiste finito per q.o. $x\in \R$, ovvero q.o. $x\in \R,\ y\mapsto f(x-y)g(y)$ è integrabile su $\R$
	\item$f*g \in L^{1}(\R)$
	\item$\|f*g\|_{L^1}\le \|f\|_{L^1}\|g\|_{L^1}$ 
\end{enumerate}
\textbf{Dimostrazione} 
\\Consideriamo $H(x,y):=f(x-y)g(y)$, a priori non sappiamo se $H\in L^{1}(\R_x \times \R_x)$, dunque non è possibile applicare direttamente fubini.
\\Quindi consideriamo $|H|\ge 0$ e applichiamo il teorema di Tonelli.
\\Verificando le ipotesi:
\begin{itemize}
	\item Integro prima in dx
\end{itemize}
\[\int_{\R_x}^{} |H(x,y)|dx=|g(y)|\int_{\R_x}^{}|f(x,y)|dx\]   
Con la sostituzione $z=x-y$ 
\[=|g(y)|\int_{\R_z}^{} |f(z)|dz=|g(y)|\cdot \|f\|_1<+\infty\] 
\begin{itemize}
	\item Integro in dy
\end{itemize}
\[\int_{\R_y}^{} \bigg[\int_{\R_x}^{} |H_(x,y)|dx\bigg]dy=\|f\|_{L^1}\int_{\R_y}^{} |g(y)|dy=\|f\|_{L^1}\|g\|_{L^1}<+\infty\]
Dunque per Tonelli $|H|\in L^{1}(\R_x \times \R_y)\implies H\in L^1(\R_x \times \R_y)$
\\A questo punto posso applicare Fubini ad $H$ 
\\Dunque per q.o. $x,\ y\mapsto H(x,y)=f(x,y)g(y)\in L^{1}(\R_y)$
\\\textbf{Dimostrazione 3 (che implica 2)}
\[\|f*g\|_{1}=\int_{\R_x}^{} |f*g(x)|dx=\int_{\R_x}^{} \bigg|\int_{\R_y}^{} f(x-y)g(y)dy\bigg|dx\le \]
\[\le \int_{\R_x}^{} \int_{\R_y}{|f(x-y)| |g(y)|dydx}\le \int_{\R_y}^{} \int_{\R_x}{|f(x-y)| |g(y)|dydx}\]
(Per Fubini)
\[=\int_{\R_y}^{} |g(y)| \int_{\R_x}^{} |f(x-y)dxdy=\|f\|_1 \int_{\R_y}^{} |g(y)|dy=\|f\|_1\cdot \|g\|_1\]   
\divider\\
\textbf{Osservazione}
\begin{itemize}
	\item vale la proposizione 1 anche su $\R^n$ 
	\item $f* g=g * f$ 
	\item le funzioni devono essere deinite su tutto lo spazio
\end{itemize}
\textbf{Estensione:} $f\in L^{1}(\R),g\in L^{p}(\R)\implies$
\begin{enumerate}
	\item $f * g(x)$ esiste per q.o. $x$ 
	\item $f*g\in L^{p}(\R)$ 
	\item $\|f * g\|_p\le \|f\|_1 \|g\|_p$
\end{enumerate}
\[H(x,y)=|f(x-y)|^p|g(y)|^p\]
\begin{tcolorbox}
	\textbf{Proposizione 2} 
	\\Siano $f\in C_0^\infty(\R)(\subseteq  L^{1}(\R)),\ g\in L^{1}(\R)$, allora:
	\begin{enumerate}
		\item $f*g\in C^\infty(\R)$ 
		\item $(f*g)^{(k)}=f^{(k)}*g\ \forall k$
	\end{enumerate}
\end{tcolorbox}
Idea della dimostrazione: 
\[f * g(x)=\int_{\R_y}^{} f(x-y)g(y)dy\]
\[(f *g)'(x)=\int_{\R_y}^{} f'(x-y)g(y)dy\]
\textbf{Osservazione 1:} 
Vale con $k$ al posto di $\infty$\\
\textbf{Osservazione 2:} In generale nelle ip. della Prop.2 $f*g$ non è a supporto compatto. 
\\\divider\\
\\\textbf{Idea della dim. del teorema di approssimazione di funzioni $L ^p$ con funzini regolari}
\\Prendiamo $p=1$, data $f\in L^{1}(\R)$, vogliamo costruire $\varphi_n\subseteq  C_0^\infty(\R)$ tale che $\varphi_n\to f$ in $L^{1}(\R)$. Prendiamo
\[f_n:=f*\rho_n\]
Dove $\rho_n$ successione di mollificatori
\[\rho_n(x)=n\rho(nx) \text{ dove }\rho \text{ è un nucleo di convoluzione}\]
\begin{itemize}
	\item $\rho\in C_0^\infty(\R),\ \rho\ge 0,\ \text{supp}(\rho)\subseteq  [-1,1],\ \int_{\R}^{} \rho(x)dx=1$ 
	\item $\rho_n\in C^\infty_0(\R),\ \rho_n\ge 0,\text{ supp}(\rho_n)\subseteq [-\frac{1}{n},\frac{1}{n}],\ \int_{\R}^{} \rho_n(x)dx=1 $
\end{itemize}
Si può dimostrare usando i teoremi di convergenza dominata che $\varphi_n\to f$ in $L^{1}(\R)$.
\\\textbf{Osservazione:} Per guadagnare anche il supporto compatto, occorre prima "trovare" $f$, cioè considerare
\[f_k=f\cdot \chi_{[-k,k]}=\begin{cases}
	f\ \ \text{se }x\in [-k,k]
	\\0\ \ \text{se }x\not\in [-k,k]
\end{cases}\]
Approssimo f per convoluzione:
\[f_k*\rho_n\in C_o^\infty\to _{n\to +\infty}f_k\]
\[\varphi_n=f_{k(n)}*\rho_n\]
\subsection{Teorema fondamentale del calcolo}
\begin{tcolorbox}
\textbf{Teorema di differenziazione nella teoria di Lebesgue}
\\Sia $f\in L^{1}([a,b])$, sia $F(x):=\int_{a}^{x} f(t)dt$, essa è derivabile q.o. su $[a,b]$ e
\[F'(x)=f(x)\text{ per q.o. }x\in (a.b)\]
\end{tcolorbox}
Esempio: $f(x)=\text{sign}(x)$, $x\in [-1,1]$.
\begin{tcolorbox}
	\textbf{Definizione: }Diciamo che $F\in \text{A.C.}([a,b])$, ovvero lo spazio delle funzioni assolutamente continue su $[a,b]$ se $\exists f\in L^{1}([a,b])$ tale che
	\[F(x)=\int_{a}^{x} f(t)dt+c\text{ con }c\in \R\]
\end{tcolorbox}
\textbf{Osservazione 1:} Tale spazio è vettoriale (per la linearità della derivata e dell'integrale).
\\\textbf{Osservazione 2:} $F\in \text{AC}([a,b])\implies$ 
\[F(b)-F(a)=\bigg[\int_{a}^{b} f(t)dt+c \bigg]-\bigg[\int_{a}^{a} f+c \bigg]=\int_{a}^{b} f(t)dt   \]
\textbf{Osservazione 3:} $F\in \text{AC}([a,b]),\ F'=0$ q.o. su  $[a,b]\implies F=c$ $\forall x\in [a,b]$ \\
Questa implicazione è falsa se togliamo l'ipotesi che $F\in \text{AC}([a,b])$ 
\\Esistono funzioni (non in $\text{AC}([a,b])$ che sono derivabili q.o. con derivata prima nulla q.o. ma non costanti.
\\\textbf{Esempio:} una funzione derivabile $f$ continua con  $f'=0$ q.o. su $(0,1)$ ma $f$ non costante (scala di Cantor).
\begin{figure}[ht]
    \centering
    \incfig{scaladicantor}
    \caption{Scala di Cantor}
    \label{fig:scaladicantor}
\end{figure}
\\La successione $\{f_n\} \subseteq  C^0([0,1])$ risulta di Cauchy in $\|.\|_\infty$.
\\Poiché ($C^0([0,1]),\|.\|_\infty$ ) è uno spazio di Banach:
\[\exists f=\lim_{n \to +\infty} f_n\in C^0([0,1])\]
$f(0)=0$, $f(1)=1$, con $f'=0$ q.o. su $(0,1)$ 
\begin{tcolorbox}
\textbf{Proposizione (caratterizzazione puntuale di AC)}
\[F\in \text{AC}([a,b])\iff \forall \varepsilon>0\exists \ \delta>0\text{ tale che } \forall \text{ famiglia }\{(x_i,y_i)\} _{i=1,\ldots,N}\]
di intervalli a 2 a 2 disgiunti $\subseteq  (a,b)$ con
\[\sum_{k=1}^{N} |y_k-x_k|<\delta,\text{ si ha }\sum_{k=1}^{N} |F(x_k)-F(y_i)|<\varepsilon\]
\end{tcolorbox}
\textbf{Osservazione:} Per $N=1$ si ha continuità uniforme
\\$\implies\text{AC}([a,b])\subseteq  \{\text{funzioni uniformemente continue su }[a,b]\} $ 
\\\textbf{Conseguenze della proposizione}
\begin{itemize}
	\item $F,G\in \text{AC}([a,b])\implies F,G\in AC([a,b])$ 
	\item $F,G\in \text{AC}([a,b])\implies$
\end{itemize}
\[F(b)G(b)=F(a)G(a)=\int_{a}^{b} (F\cdot G)'(t)dt=\int_{a}^{b} (F'G+FG') dt  \]
Ovvero
\[\int_{a}^{b} fG=-\int_{a}^{b} Fg+F\cdot G|_a^b \] 
$\implies$ vale in AC la formula di integrazione per parti.


\section{Operatori lineari tra spazi vettoriali normati}
\begin{tcolorbox}
\textbf{Definizione: }Siano $(V,\|.\|_V)$ e $(W,\|.\|_W)$ due spazi vettoriali normati.
\\Un operatore lineare da $V$ in $W$ è una funzione  $T:V\to W$ tale che
\[T(\lambda_1v_1+\lambda_2v_2)=\lambda_1T(v_1)+\lambda_2T(v_2) \ \forall v_1,v_2\in V,\  \forall \lambda_1,\lambda_2\in \R\]
\end{tcolorbox}
\textbf{Esempi} 
\\1) $V=W=\R^n$, $T:\R^n\to \R^n$ 
\[T(v)=A\cdot v,\text{ con }A\in \mathcal M (n\times n,\R)\]
2) $V=C^0([a,b])$, fisso $x_0\in (a,b),$ $W=R$
\[T:V\to W\text{ definita da }T(f)=f(x_0)\]
3) $V=C^1([a,b])$, $W=C^0([a,b])$
\[T:V \to W\text{ definita da }T(f)=f'\]
\textbf{Osservazione:} $T$ operatore lineare $\implies T(0)=0$
\begin{tcolorbox}
\textbf{Definizione: } $T:V\to W$ op. lineare, si dice \emph{continuo} se, $\forall v\in V$, $T$ è continuo in $v$, ovvero:
\[v_n \to v\implies T(v_n)\to T(v)\]
Rispettivamente nella norma di $V$ e $W$.
\end{tcolorbox}
\textbf{Osservazione: }Sia $T:V\to W$ op. lineare, allora $T$ è continuo su $V\iff T$ è continuo in $v=0$.
\\\textbf{Dimostrazione} 
\\$(\implies)$ è immediata 
\\($\impliedby$) Verifichiamo che se la proprietà vale per $v=0$, vale per $v$ qualsiasi.
\\Sia $v$ qualsiasi, e sia $v_n\to v$; considero $v_n-v\to 0$, quindi, per ipotesi $T(v_n-v)\to T(0)$
\\Ovvero $T(v_n)- T(v)\to 0$, cioè $T(v_n)\to T(v)$.
\\\divider
\begin{tcolorbox}
	\textbf{Definizione: } Sia $T$ op. lineare$:(V,\|.\|_V)\to (W,\|.\|_W)$.
	\\Si dice che T è limitato se:
	\[\exists  M>0\text{ tale che }\|T(v)\|_W\le M\|v\|_V\ \forall v\in V\]
	ovvero
	\[\exists  M>0 \text{ tale che } \frac{\|T(v)\|_W}{\|v\|_V}\le M\ \forall v\in V\setminus\{0\}.\]
	\[\exists  M>0 \text{ tale che } \sup_{v\in V\setminus \{0\} }\frac{\|T(v)\|_W}{\|v\|_V}\le M\]
\end{tcolorbox}
\textbf{Esempi:} 
\\1) $T:(\R^2,\|.\|_2)\to (\R,|.|)$ definito da $T(v)=v_0\cdot v$ operatore lineare.
\\$T$ è limitato, $M=\|v_0\|$ 
\\2) $T:(C^1([a,b]),\|.\|_{C^1})\to (C^0([a,b]),\|.\|_{C^0})$, $T(f)=f'$ op. lineare.
\\$T$ è limitato con la scelta $M=1$
\\3) $T:(L^{2}(0,1),\|.\|_2)\to (\R,|.|)$, $T(f)=\int_{0}^{1} f_0\cdot fdx $ dove $f_0\in L^{2}(0,1)$ 
\\$T$ è limitato con la scelta $M=\|f_0\|_2$ (Tramite disuguaglianza di Holder)
\\\textbf{Osservazione:} Considerando $T:(L^{p}(0,1),\|.\|_p)\to (\R,|.|)$ definito da $T(f)=\int_{0}^{1} f_0fdx $ questo è lineare continuo prendendo $f_0\in L^{p'}(0,1)$.
\\\divider
\begin{tcolorbox}
	\textbf{Proposizione: }Sia $T:(V,\|.\|_V)\to (W,\|.\|_W)$ lineare. Allora
	\[T\text{ continuo }\iff T \text{ limitato}\]
\end{tcolorbox}
\textbf{Dimostrazione} 
\\($\impliedby$) Supposto $T$ limitato, basta mostrare che $T$ è continuo in $0$, ovvero: se $v_n\to 0$, allora $T(v_n)\to T(0)=0$ 
\[\|T(v_n)\|_W\le M \|v_n\|_V\to 0\]
($\implies$) Supposto $T$ \emph{non limitato} mostriamo $T$ non continuo
\[\sup_{v\in V\setminus \{0\} }\frac{\|T\|_W}{\|v\|_V}=+\infty\implies \exists \{v_n\} \subseteq  V\setminus \{0\} :\frac{\|T(v_n)\|_W}{\|v_n\|_V}\to +\infty\]
ovvero, siccome $T$ è lineare:
\[\bigg\|T\bigg(\frac{v_n}{\|v_n\|_V}\bigg)\bigg\|_W\to +\infty\]
Quindi se considero $u_n:= \frac{v_n}{\|v_n\|_V}$, ha che 
\[\begin{cases}
	\|u_n\|_V=1\\
	\|T(u_n)\|_W\to +\infty
\end{cases}\]
Posso costruire una successione $y_n$ tale che $y_n\to 0$ ma $T(y_n)\not \to 0$
Ponendo $y_n= \frac{u_n}{\|T(u_n)\|_W}$
\begin{itemize}
	\item $y_n\to 0$ poiché 
\end{itemize}
\[\|y_n\|_V=\bigg\|\frac{u_n}{\|T(u_n)_W\|}\bigg\|_V\to 0\]
\begin{itemize}
	\item $T(y_n)=1$ perché
\end{itemize}
\[T(y_n)=T\bigg( \frac{u_n}{\|T(u_n)\|_W} \bigg)= \frac{T(u_n)}{\|T(u_n)\|_W}\not\to 0\]
\begin{tcolorbox}
	\textbf{Definizione: }Dati $(V,\|.\|_V),(W,\|.\|_W)$ spazi normati
	\[\mathcal L(V,W):=\{\text{op. lineari limitati da } V \text{ in }W\}\] 
	È uno spazio vettoriale munito delle operazioni naturali
\end{tcolorbox}
È possibile introdurre su questo spazio una norma, ponendo 
\[\|T\|_{\mathcal L(V,W)}:=\sup_{v\in V\setminus \{0\} }\frac{\|T(v)\|_W}{\|v\|_V}\]
ovvero, per definizione, la più piccola costante $M$ tale che $\|T(v)\|_W\le M \|v\|_V\ \forall v\in V$.
\\\textbf{Osservazione:} Si può verificare che quella definita sopra è effettivamente una norma.\\
In particolare
\begin{tcolorbox}
	\textbf{Definizione: }Quando $W=(\R,|.|)$ 
	\[\mathcal L (V,W)=V'\text{ spazio duale di }V\]
\end{tcolorbox}
\[\|T\|_{V'}:=\sup_{v\in V\setminus \{0\} } \frac{|T(v)|_{\R}}{\|v\|_V}\]
\textbf{Esempi: }Vedere i casi 1) e 3)


\section{Distribuzioni}
\begin{tcolorbox}
\textbf{Definizione: }Sia $\Omega$ aperto di $\R^n$ 
\[C_0^\infty(\Omega)=\{\text{funzioni }C^\infty\text{ su }\Omega\text{ con supporto compatto in }\Omega\}\]
È uno spazio vettoriale
\end{tcolorbox}
Muniamo $C_0^\infty$ di una \textbf{convergenza} 
\begin{tcolorbox}
	\textbf{Definizione: }Sia $\{\varphi_h\} \subseteq  C_0^\infty(\Omega)$. Diciamo che 
	\[\varphi_h\to 0\text{ in }C_0^\infty(\Omega)\text{ se }\]
\begin{enumerate}
	\item $\exists K$ compatto, indipendente da h, tale che $\text{supp}(\varphi_h)\subseteq  K\ \forall h> >\nu$
	\item $\varphi_h\to 0$ uniformemente su $K$ con tutte le derivate $\forall \alpha$ multiindice $D^\alpha\varphi_h\to 0$ unif. su $K$
\end{enumerate}
\end{tcolorbox}
\begin{tcolorbox}
	\textbf{Definizione: }Lo spazio $C_0^\infty(\Omega)$ munito della convergenza definita sopra si indica con $\mathcal D(\Omega)$ e si chiama \emph{spazio delle funzioni test}
\end{tcolorbox}
\begin{tcolorbox}
	\textbf{Definizione: }Lo spazio delle distribuzioni su $\Omega $, che si indica con $\mathcal D'(\Omega)$ è lo spazio degli operatori $T:\mathcal D(\Omega)\to \R$ lineari e continui rispetto alla convergenza introdotta su $\mathcal D(\Omega)$.\\Ovvero, una distribuzione è un operatore $T:\mathcal D(\Omega)\to \R$ tale che 
	\begin{itemize}
		\item $T$ lineare
		\item $T$ continuo ($\varphi_h\to 0$ in $\mathcal D(\Omega)\implies T(\varphi_h)\to 0$ in $\R$)
	\end{itemize}
\end{tcolorbox}
\textbf{Esempi} 
\begin{enumerate}
	\item Sia $u\in L^{1}(\Omega)$, ad $u$ posso associare una distribuzione $T_u\in \mathcal D'(\Omega)$
\end{enumerate}
\[T_u(\varphi):=\int_{\Omega}^{} u\varphi\ \ \forall \varphi\in \mathcal D(\Omega)\]
È ben definito:
\[\bigg| \int_{\Omega}^{} u\varphi\bigg|\le \int_{\Omega}^{} |u\varphi|\le \int_{K}^{} \max |\varphi | |u|\le \max_k|\varphi|\int_{K}^{} |u|  \]  
È lineare:
\[T_u(\alpha\varphi+\beta\psi)=\int_{\Omega}^{} u(\alpha\varphi+\beta\psi)=\alpha \int_{\Omega}^{} u\varphi +\beta \int_{\Omega}^{}u\psi =\alpha T_u(\varphi)+\beta T_u(\psi)   \]
È continuo:
\[\{\varphi_h\} \to 0\text{ in }\mathcal D(\Omega)\implies T_u(\varphi_h)\to 0\]
Poiché, sia $\{\varphi_h\} \to 0\text{ in }\mathcal D(\Omega)$
\[|T_u(\varphi_h)|=\bigg|\int_{\Omega}^{} u\varphi_h \bigg|\le \max_K|\varphi_h|\cdot \int_{K}^{} |u|\to 0\]  
\\\textbf{Osservazioni sull'esempio} 
\\L'associazione tra $u,\ T_u$ è iniettiva su $L^{1}(\Omega)$
\\Se $u_1=u_2\text{ q.o. su }\Omega\implies T_{u_1}=T_{u_2}\text{ in }\mathcal D'(\Omega)$ poiché $T_{u_1}(\varphi)=T_{u_2}(\varphi)$
\\Si può dimostrare che $T_{u_1}=T_{u_2}\text{ in }\mathcal D'(\Omega)\implies u_1=u_2 \text{ q.o. su }\Omega$ (*)
\[\int_{\Omega}^{} u_1\varphi = \int_{\Omega}^{} u_2\varphi\ \forall \varphi\in \mathcal D(\Omega)\implies u_1=u_2\text{ q.o. su } \Omega\] 
\[\int_{\Omega}^{} (u_1-u_2)\varphi =0 \forall \varphi\in \mathcal D(\Omega)\implies u_1=u_2\text{ q.o. su } \Omega\] 
\divider
\begin{tcolorbox}
\textbf{Notazione:} Invece di $T_u(\varphi)$ si scrive spesso $<u,\varphi>_{(\mathcal D'(\Omega), \mathcal D(\Omega))}$
\end{tcolorbox}
Per definire $T_u$, basta una condizione più debole:
\[u\in L^1_{\text{loc}}(\Omega):=\{u:\Omega\to \R: \int_{K}^{} |u|<+\infty\ \forall K\text{ compatto }\subseteq  \Omega \}\]
\textbf{Esempio:} $\Omega=(0,1),\ u(x)=\frac{1}{x}\not\in L^1(\Omega)$ ma $u\in L^1_{\text{loc}}(\Omega)$
\\In particolare, possiamo associare una distribuzione a qualsiasi $u\in L^{p}(\Omega)\text{ con } p \in [1,+\infty]$ 
\\Infatti $L^{p}(\Omega)\not \subseteq  L^1(\Omega), $ ma 
\[L^{p}(\Omega)\subseteq  L^p_{\text{loc}}(\Omega),\ \forall  p \in [1,+\infty]:\]
\[u\in L^{p}(\Omega)\implies u\in L^{p}(K)\ \forall K \subset \subset \Omega\]
Poiché $|K|<+\infty$.
\[\implies u \in L^{1}(K)\ \forall K \subset  \subset \Omega \implies u\in L_{\text{loc}}^1(\Omega)\]
Tutte le funzioni $u\in L^{p}(\Omega)$ possono essere viste come distribuzioni.
\[u\in L^{p}(\Omega)\mapsto T_u\]
\[\left<u,\varphi \right>_{\mathcal D'(\Omega), \mathcal D(\Omega)}:=\int_{\Omega}^{} u\varphi dx\]
\divider\\
\\Essendo $\mathcal D'$ vettoriale
\[(T_1+T_2)(\varphi):=T_1(\varphi)+T_2(\varphi)\ \forall \varphi\in \mathcal D(\Omega)\]
\[(\lambda T):=\lambda T(\varphi) \ \forall \varphi \in \mathcal D(\Omega)\]
\subsection{Convergenza}
\begin{tcolorbox}
\textbf{Definizione: }
\[\{T_h\} \subseteq  \mathcal D'(\Omega),\ T_h\to^{\text{in }\mathcal D'(\Omega)}0\text{ se }T_h(\varphi)\to 0\ \forall \varphi\in \mathcal D(\Omega)\] 
\end{tcolorbox}
\[T_h\to T\text{ se }T_h(\varphi)\to T(\varphi)	\ \forall \varphi\in \mathcal D(\Omega)\]
\\\textbf{Esempio: }$T_h=T_{u_h}$, con $u_h\subseteq  L^{1}(\Omega)$ \\
\[u_h\to 0\text{ in }L^{1}(\Omega)\implies T_{u_h}\to 0\text{ in }\mathcal D'(\Omega)\]
Dato che
\[|T_{u_h}(\varphi)|=\bigg|\int_{\Omega}^{} u_h\varphi \bigg|\le \int_{K=\text{supp}\varphi}^{} |u_h| |\varphi|\le \max_K|\varphi|\cdot \int_{K}^{}  |u_h|\to 0\]
\subsection{Delta di Dirac}
\[\{u_h\} \subseteq  L^{1}(\R)\]
\begin{figure}[ht]
    \centering
    \incfig{dirac}
    \caption{Delta di Dirac}
    \label{fig:dirac}
\end{figure}
Questa successione non converge in $L^{1}(\R)$\\
$u_h\to 0$ q.o. su $\R\implies $ se $\exists \lim_{h \to +\infty} u_h$ in $L^{1}(\R)$ allora $\lim_{h \to +\infty} u_h=0$.
\\Ma $\lim_{h \to +\infty} u_h\neq 0$ in $L^{1}(\R)$ perché
\[\|u_h\|_{L^{1}(\R)}=\int_{-\frac{1}{2h}}^{\frac{1}{2h}} h=1\]
Converge però in $\mathcal D'(\R)$ 
\[\left<u_h,\varphi \right> =T_{u_h}(\varphi)=\int_{\R}^{} u_h\varphi=h \int_{-\frac{1}{2h}}^{\frac{1}{2h}} \varphi\to \varphi(0)\]  
\begin{tcolorbox}
	\textbf{Definizione: }$\delta_0$ \emph{delta di Dirach} in 0
	\[<\delta_0,\varphi>:=\varphi(0)\]
\end{tcolorbox}
\textbf{Osservazioni:} 
\begin{itemize}
	\item Se $u_h=h\cdot \chi_{[-\frac{1}{2h},\frac{1}{2h}]},$ allora $u_h\to \delta_0$ in $\mathcal D'(\R)$ 
	\item Verifica che $\delta_0\in \mathcal D'(\R)$
\end{itemize}
(i) lineare: $\delta_0(\alpha\varphi+\beta\psi)=\alpha\delta_0(\varphi)+\beta\delta_0(\psi)$?
\[(\alpha\varphi+\beta\psi)(0)=\alpha\varphi(0)+\beta\psi(0)\]
(ii) continuo: $\varphi_h\to 0$ in $\mathcal D(\R)\implies\delta_0(\varphi_h)\to 0$
\\vero per la definizione di convergenza in $\mathcal D(\R)$, $\text{supp}(\varphi_h)\subseteq  K$ compatto, $\varphi_h\to 0$ uniformemente.
\subsubsection{Ovvie generalizzazioni}
Punto generico $x_0$
\[\delta_{x_0}(\varphi)=\varphi(x_0)\]
Caso n-dimensionale, $x_0\in \R^n$
\[\delta_{x_0}(\varphi):=\varphi(x_0),\ \delta_{x_0}\in \mathcal D'(\R^n)\]
\subsubsection{idk}
$\delta_0$ non è associata ad alcuna funzione di $u\in L^1_{\text{loc}}(\Omega)$
\\\textbf{Dimostrazione} 
\\Supponiamo per assurdo $\delta_0=T_u,$ con $u\in L^1loc(R)$
\[\int_{\R}^{} u\varphi dx=\varphi(0)\ \forall \varphi \in \mathcal D(\R)\]
In particolare, posso prendere $\varphi \in \mathcal D(\R \setminus \{0\} )$ 
\[\int_{\R}^{} u\varphi=\varphi(0)=0\ \forall \varphi\in\mathcal D(\R\setminus \{ 0\} )\]
Ricordando l'osservazione sull'esempio 
\[0=\int_{\R}^{} 0\cdot \varphi\]
Applicando tale proprietà si avrà
\[u=0\text{ q.o. su }\R-\setminus \{ 0\} \implies u=0\text{ q.o. su }\R\]
\[\implies \int_{\R}^{} u\varphi dx=0\ \forall \varphi \in \mathcal D(\R)\]
Assurdo.
\subsubsection{Derivazione di distribuzioni}
\begin{tcolorbox}
\textbf{Definizione: }$\Omega \subseteq  \R$
\\Data $T\in \mathcal D'(\Omega)$, definisco $T' \in \mathcal D'(\Omega)$ come:
\[<T', \varphi>:=- <T,\varphi'>\ \ \forall \varphi \in \mathcal D(\Omega)\]

\end{tcolorbox}
$T'$ è una distribuzione
\\(i) è lineare:
\[<T',\alpha\varphi +\beta\psi> = -<T,\alpha \varphi'+\beta\psi '>\] \[= -\alpha <T,\varphi'>-\beta<T,\psi'> = +\alpha <T',\varphi>+\beta<T',\psi>\]
(ii) è continuo
\[\varphi_h\to 0\text{ in }\mathcal D(\Omega)\implies \left<T',\varphi_h \right>\to 0\]
Infatti $\varphi_h\to 0\text{ in }\mathcal D(\Omega)\implies\varphi_h'\to 0\text{ in }\mathcal D(\Omega)$.
\\Questo perché $\exists K$ tale che $\text{supp}\varphi'_h\subset K\forall h$ e $\varphi'_h\to 0$ uniformemente su $K$ con tutte le derivate.
\\Quindi $\left<T,\varphi'_h \right>\to 0$ perché $T\in \mathcal D'(\Omega)$
\subsubsection{Motivo della definizione di derivata}
Considerando il caso $T=T_u$ con $u\in C^1(\Omega)\subseteq  L^1_{\text{loc}}(\Omega)$
\\Si avrà in $\mathcal D'(\Omega)$ che $(T_u)'=T_{u'}$ 
\\\textbf{Dimostrazione} 
\[\left<(T_u)', \varphi \right> = -\left<T_u,\varphi' \right> =-\int_{\Omega}^{} u\varphi'\]
\[\left<T_{u'}, \varphi \right> = \int_{\Omega}^{} u'\varphi\]
Integrando per parti si avrà
\[\int_{\Omega}^{} u'\varphi = u\varphi - \int_{\Omega}^{} u\varphi'\]
Essendo $\varphi \in \mathcal D(\Omega)\implies u\varphi|_a^b=0$ (dove $a,b$ estremi di $\Omega$)
Si avrà:
\[\int_{\Omega}^{} u'\varphi=-\int_{\Omega}^{} u\varphi'\]  


\section{Esempi di distribuzioni}
\begin{enumerate}
	\item $T=T_u$ con $u\in C^1(\Omega)\implies (T_u)'=T_{u'}$ 
	\item $T=T_u$ con $u(x)=|x|$ su $\Omega=(-1,1)$ 
		\[\left<(T_u)',\varphi \right> =- \left<T_u,\varphi' \right> =-\int_{-1}^{1} |x|\varphi'(x)dx\]
		\[=\int_{0}^{1} \varphi(x)dx+x\varphi \bigg|_0^1-\int_{-1}^{0} \varphi(x)dx+x\varphi(x)\bigg|_{-1}^0=\int_{-1}^{1} \varphi(x)\cdot \text{sign}(x)dx\]
		\[\implies (T_{|x|})'=T_{\text{sign}(x)}\text{ in }\mathcal D'(\Omega) \]
		\textbf{Notazione:} $(|x|)'=\text{sign}(x)\text{ in }\mathcal D'(\Omega)$ 
		\\Più in generale: se $u\in L^1_{\text{loc}}(\Omega),v\in L^1_{\text{loc}}(\Omega)$  $u'=v\text{ in }\mathcal D'(\Omega)$, significa $(T_u)'=T_v$ ovvero
		\[ \ \forall \varphi\in \mathcal D(\Omega)\ \left<(T_u)',\varphi \right> = - \left<T_u,\varphi' \right> = \left<T_v,\varphi \right> \]
			\[-\int_{\Omega}^{} u\varphi'=\int_{\Omega}^{} v\varphi\ \forall \varphi\in \mathcal D(\Omega)  \]
		\item $u(x)=\text{sign}(x)$, $u'=?$ 
			\[-\int_{\Omega}^{} \text{sign}(x)\varphi'(x)dx=-\int_{0}^{1} \varphi'+\int_{-1}^{0} \varphi'=-\varphi(1)+2\varphi(0)-\varphi(-1)=2\varphi(0)\]
			\[=2 \left<\delta_0,\varphi \right>\]
\item $T=\delta_0$ $T'=?$ 
	\[\left<T',\varphi \right> = - \left<T,\varphi' \right> =- \left<\delta_0,\varphi' \right> =-\varphi'(0)\ \forall \varphi\in \mathcal D(\Omega)\]
\end{enumerate}
\subsection{Generalizzazione }
\begin{itemize}
	\item $n=1$ Data $T\in \mathcal D'(\Omega),\ \forall k\in \N\ T^{(k)}\in \mathcal D'(\Omega)$
		\[\left<T^{(k)},\varphi \right>:= (-1)^k\left<T,\varphi^{k} \right>  \]
\end{itemize}
\textbf{Osservazione:} $T^{(k)}$ definisce una distribuzione, lineare e continua, infatti se $\varphi_h\to 0\text{ in }\mathcal D(\Omega)$, $\varphi_h^{(k)}\to 0\text{ in }\mathcal D(\Omega)\implies \left<T,\varphi_h^{(k)} \right> \to 0$.\\
\textbf{Osservazione 2:} se $T=T_u$ con $u\in C^k(\Omega)\subseteq  L^1_{\text{loc}}(\Omega)\implies (T_u)^{\left( k \right) }=T_{u^{(k)}}$\\
\textbf{Esempio: }$u(x)=|x|\implies u''=2\delta_0$
\\\divider
\begin{itemize}
	\item $n\ge 1$ Data $T\in \mathcal D'(\Omega)\ \forall \alpha$ multiindice 
		\[\left<D^\alpha T,\varphi \right> =(-1)^{|\alpha|}\left<T,D^{\alpha}\varphi \right> \ \forall \varphi\in \mathcal D(\Omega)\]

\end{itemize}
\textbf{Osservazione: }$D^\alpha T$ definiscono delle distribuzioni $\ \forall \alpha$ 
\\\textbf{Osservazione: }Si possono calcolare le derivate di tutti gli ordini, di qualsiasi $T\in \mathcal D'(\Omega)$ 
\\\textbf{Osservazione:} Il risultato non dipende dall'ordine di derivazione
\subsubsection{Operatori differenziali}
Data $T\in \mathcal D'(\Omega)$, si possono definire $\nabla T,\nabla^2 T,\text{rot}T,\ldots$ 
\section{Spazi di Sobolev}
Sono gli spazi dove si trovano le soluzioni di problemi al contorno per P.D.E.
\\Esempio Equazione di Poisson
\[\begin{cases}
	-\nabla ^2u=f\text{ su }\Omega
	\\u=0\text{ su }\partial\Omega
\end{cases}\]
\begin{tcolorbox}
	\textbf{Definizione: }Fissato $\Omega$ aperto $\subseteq  \R^n,\ p\in [1,+\infty]$
	\[W^{1,p}(\Omega):=\{u\in L^p(\Omega): \frac{\partial u}{\partial x_i} \in L^p(\Omega)\ \forall i=1,\ldots,n \} \]
	Con $\frac{\partial }{\partial x_i }$ intesa nel senso delle distribuzioni
\end{tcolorbox}
\[\frac{\partial u}{\partial x_i} \in L^p(\Omega) \iff \exists v_i\in L^p(\Omega)\text{ tali che }\]
\[\int_{\Omega}^{} \frac{\partial u}{\partial x_i} \cdot \varphi=\int_{\Omega}^{}v_i\varphi\ \forall \varphi\in \mathcal D(\Omega) \]
\textbf{Esempi} ($n=1,\ \Omega=(-1,1)$)
\begin{itemize}
	\item $u\in C^1_0(\Omega)\implies u\in W^{1,p}(\Omega)$
		\[(1) \ \ p<+ \infty\ \ \int_{\Omega}^{} |u|^p<+\infty\ ;\ \int_{\Omega}^{} |u'|^p<+\infty\]
		\[(2)\ \ p=+\infty\ \ \underset{\Omega}{\text{ess-sup}}|u|<+\infty\ ;\ \underset{\Omega}{\text{ess-sup}|u'|}<+\infty\]
	\item $u(x)=\text{sign}(x)$ $u\not\in W^{1,2}(\Omega)$ 
		\[\int_{\Omega}^{} |u|^2=\int_{-1}^{1} |\text{sign}x|^2<+\infty\implies u\in L^2(\Omega)  \]
		MA: $u'(x)=2\delta_0\not\in L^{2}(\Omega)$
\end{itemize}
\begin{tcolorbox}
	\textbf{Definizione: }Fissato $\Omega$ aperto $\subseteq  \R^n$, $p\in [1,+\infty]$, $k\in \N$
	\[W^{k,p}(\Omega):=\{u\in L^p(\Omega):D^\alpha u\in L^{p}(\Omega)\ \forall \alpha\text{ multiindice con }|\alpha|\le k\} \]
	
\end{tcolorbox}
\textbf{Caso particolare} $p=2$ 
\[W^{k,2}(\Omega)=H^k(\Omega)\]
\\\textbf{Osservazione: }$W^{k,p}(\Omega)$ sono spazi vettoriali
\begin{tcolorbox}
	\textbf{Definizione:} Norma su $W^{1,p}(\Omega)$ sia $u\in W^{1,p}(\Omega)$ 
	\[\|u\|_{1,p}:=\|u\|_p+\sum_{k=1}^{n} \bigg\|\frac{\partial u}{\partial x_i} \bigg\|_p\]

\end{tcolorbox}
\begin{tcolorbox}
	\textbf{Definizione:} Norma su $W^{k,p}(\Omega)$ sia $u\in W^{k,p}(\Omega)$ 
	\[\|u\|_{k,p}:=\|u\|_p+\sum_{|\alpha|\le k}^{}\|D^\alpha\|_p \]
\end{tcolorbox}
\begin{tcolorbox}
	\textbf{Teorema: }Per ogni $p\in [1,+\infty],$ $W^{1,p}(\Omega)$ sono spazi di Banach
\end{tcolorbox}
\textbf{Osservazione:} $u_h\to u$ in $W^{1,p}(\Omega)$ se
\[\|u_h-u\|_{1,p}\to 0\]\[=\|u_h-u\|_p+\bigg\| \frac{\partial u_h}{\partial x_i} -\frac{\partial u}{\partial x_i} \bigg\|_p\]
ovvero
\[\begin{cases}
	u_h\to u\text{ in }L^{p}(\Omega)\\
\frac{\partial u_h}{\partial x_i} \to \frac{\partial u}{\partial x_i} \text{ in }L^{p}(\Omega)
\end{cases}\]

\begin{tcolorbox}
\textbf{Definizione}
\[W_0^{1,p}(\Omega):=\text{chiusura di }\mathcal D(\Omega)\text{ in }W^{1,p}(\Omega)\]
ovvero
\[=\{u\in W^{1,p}(\Omega):\exists \{\varphi_n\}\subseteq  \mathcal D(\Omega)\text{ tale che}\varphi_n\to u\text{ in }W^{1,p} \}\]
\[=\{u\in W^{1,p}(\Omega):\exists \{\varphi_n\} \subseteq  \mathcal D(\Omega)\text{ tale che }\varphi_n\to u \text{ in }L^{p}\text{ e }\frac{\partial \varphi_n}{\partial x_i} \to u\text{ in }L^{p}\}\]
\end{tcolorbox}
\textbf{Osservazione} 
Se $u\in W^{1,p}(\Omega)\cap C(\overline\Omega)$ allora 
\[u\in W_o^{1,p}(\Omega) \iff u=0\text{ su }\Omega\]
\subsubsection{Disuguaglianza di Poincaré}
\begin{tcolorbox}
\textbf{Teorema}
Sia $\Omega$ aperto, limitato di $\R^n$. Allora esiste una costante $C_p=C_p(\Omega)$ tale che, per ogni $u\in W_0^{1,p}(\Omega)$ 
\[\|u\|_{L^{p}(\Omega)}\le C_p(\Omega)\cdot \|\nabla u\|_{L^{p}(\Omega)}\]
\end{tcolorbox}

Dunque, su $W_0^{1,p}(\Omega)$
\[\begin{cases}
	\|u\|_{1,p}=\|u\|_p+\|\nabla u\|_p\text{  norma su }W^{1,p}(\Omega)
\\\|\nabla u\|_p\text{  norma equivalente}
\end{cases}\]
\\\textbf{Falso }su $W_0^{1,p}(\Omega)$, verificabile prendendo $u=1$ 
\\\divider\\
\\\textbf{Osservazione} per $n=1$ 
\[u(x)=u(0)+\int_{0}^{x} u'(t)dt\implies |u(x)|\le \int_{0}^{x} |u'|\le \int_{0}^{1} |u'|\le \bigg(\int_{0}^{1} |u'|^2\bigg)^{1 / 2}\]
Integrando 
\[\|u\|_{L^{1}(0,1)}\le \|u'\|_{L^{2}(0,1)}\]
\[W_0^{1,p}(\Omega)\]

\section{Spazi di Hilbert}
\begin{tcolorbox}
\textbf{Definizione: }Sia $H$ uno spazio vettoriale su $\R$ 
\\Un prodotto scalare su $H$ è un'applicazione $(\ ,):H\times H\to \R$ tale che
\begin{enumerate}
	\item $(x,x)\ge 0\ \forall x\in H$ con $(x,x)=0\iff x=0$ \emph{positività} 
	\item $(x,y)=(y,x)\ \forall x,y\in H$ \emph{simmetria}
	\item $(\alpha_1x_1+\alpha_2x_2,y)=\alpha_1(x_1,y)+\alpha_2(x_2,y)$ \emph{bilinearità}
\end{enumerate}
\end{tcolorbox}
\begin{tcolorbox}
	\textbf{Definizione: }$\|x\|:=\sqrt{(x,x)} $ è detta \emph{norma associata} (o indotta) dal prodotto scalare

\end{tcolorbox}
\textbf{Esempi} 
\begin{itemize}
	\item $H=\R^n$ ;   $(x,y)=\sum_{k=1}^{n} x_ky_k$ ;   $\sqrt{ (x,x)}=\sqrt{\sum_{k=1}^{n} x_k^2}=\|x\|_2 $ ; ovvero la norma euclidea
	\item $H=L^{2}(\Omega)$   $(f,g)=\int_{\Omega}^{} fg $ ; $\sqrt{(f,f)} =(\int_{\Omega}^{} f^2)^{1 / 2}=\|f\|_2 $ 
	\item $H=W^{1,2}(\Omega)$ ; $(f,g):=\int_{\Omega}^{} fg+\sum_{k=1}^{n} \frac{\partial f}{\partial x_i} \frac{\partial g}{\partial x_i} =\int_{\Omega}^{}fg+\nabla f\cdot \nabla g   $       
		\[\sqrt{(f,f)}=\bigg(\int_{\Omega}^{} f^2+|\nabla f|^2\bigg)^{1 / 2}\simeq\|f\|_{H^1} \]
\end{itemize}
norma equivalente alla norma di $H^1$ 
\subsection{Disuguaglianza di Cauchy Schwartz}
Se $(\ ,)$ è un prodotto scalare su $H$, allora
\[|(x,y)|\le \|x\|\cdot \|y\|\ \forall x,y\in H\]
Inoltre vale $=\iff x=\lambda y$ con $\lambda \in \R$
\\\textbf{Dimostrazione} 
\\$\ \forall t\in \R$ $0\le (x-ty,x-ty)=(x,x)-2t(x,y)+t^2(y,y)$
\\Dunque
\[0\le \|x\|^2-2t(x,y)+t^2\|y\|^2\implies \Delta\le 0\]
\[\Delta = 4(x,y)^2-4\|x\|^2\|y\|^2\le 0\]
\[\implies |(x,y)|\le \|x\|\|y\|\]
Se vale $=$, $\Delta = 0\implies \exists \lambda\in \R:(x-\lambda y,x-\lambda y)=0\implies x-\lambda y=0$
(Viceversa se $x=\lambda y$ )
\begin{tcolorbox}
	\textbf{Proposizione:} Se $(,):H\times H\to \R$ è un prodotto scalare,
	\[\|x\|:=\sqrt{(x,x)} \text{ è una norma}\]

\end{tcolorbox}
\textbf{Dimostrazione} 
\begin{itemize}
	\item $\|x\|\ge 0$ con $=\iff x=0$ vera per la prop. (1)
	\item $\|\lambda x\|=\sqrt{(\lambda x,\lambda x} =|\lambda|\sqrt{(x,x)} =\lambda \|x\|$
	\item $\|x+y\|=\sqrt{(x+y,x+y)} =\sqrt{\|x\|^2+2(x,y)+\|y\|^2} $
\end{itemize}
\[\le \sqrt{\|x\|^2+2\|x\| \|y\|+\|y\|^2} =\|x\|+\|y\|\]
\subsubsection{Legge del parallelogramma}
\begin{tcolorbox}
	\textbf{Teorema: }Sia $H$ uno spazione vettoriale con prodotto scalare $(,)$ e sia $\|.\|$ la norma indotta da esso. Allora 
	\[\|x+y\|^2+\|x-y\|^2=2\|x\|^2+2\|y\|^2\ \forall x,y\in H\]

\end{tcolorbox}
\begin{figure}[ht]
    \centering
    \incfig{parallelogramma}
    \caption{Legge del parallelogramma in $\R^2$}
    \label{fig:parallelogramma}
\end{figure}
\textbf{Dimostrazione}
\[\|x+y\|^2+\|x-y\|^2=(x+y,x+y)+(x-y,x-y)=\|x\|^2+2(x-y)+\|y\|^2+\|x\|^2-2(x,y)+\|y\|^2=2\|x\|^2+2\|y\|^2\]
\textbf{Osservazione: }Può servire a verificare se una norma proviene o meno da un prodotto scalare.
\\Le norme di $\R^n,L^{p}(\Omega), W^{1,p}(\Omega)$ con $p\neq 2$ non provengono da un prodotto scalare.
\\\textbf{Esempio: }$\Omega=(0,1)$ in $L^{p}(0,1)$ con  $p\neq 2$, la norma non proviene da un prodotto scalare
\\Fisso $t\in (0,1)$, considero le funzioni
 \begin{itemize}
	 \item $f=\chi_{(0,t)}$ 
	 \item $g=\chi_{(t,1)} $
\end{itemize}
\[\|f\|_p=\bigg(\int_{0}^{1} |f|^p\bigg)^{\frac{1}{p}}=\bigg(\int_{0}^{t} 1\bigg)^{\frac{1}{p}}=t ^{\frac{1}{p}}\]
\[\|g\|_p=\bigg(\int_{0}^{1} |g|^p\bigg)^{\frac{1}{p}}=\bigg(\int_{t}^{1} 1\bigg)^{\frac{1}{p}}=(1-t) ^{\frac{1}{p}}\]
\[\|f+g\|_p=1\]
\[\|f-g\|_p=1\]
L'identità del parallelogramma diventa:
\[2=2t ^{\frac{2}{p}}+2(1-t)^{\frac{2}{p}}\]
\[1=t ^{\frac{2}{p}}+(1-t)^{\frac{2}{p}}\]
Valida $\iff p=2$
\begin{tcolorbox}
	\textbf{Definizione: }Uno \emph{spazio di Hilbert} è uno spazio di Banach in cui la norma proviene da un prodotto scalare.  \end{tcolorbox} \textbf{Esempi:} sono spazi di Hilbert
\begin{itemize}
	\item $(\R^n,\|.\|_2)$ 
	\item $L^{2}(\Omega)$ 
	\item $H^1(\Omega)$
\end{itemize}
Non sono di Hilbert
\begin{itemize}
	\item $(\R^n,\|.\|_p$ con $p\neq 2$ 
	\item $L^{p}(\Omega)$ con $p\neq 2$ 
	\item $W^{1,p}(\Omega)$ con $p\neq 2$ 
	\item $C^0([a,b]),\ \|f\|_2=(\int_{a}^{b} |f|^2)^{1 / 2} $ la norma viene da un prodotto scalare MA non è uno spazio di Banach, dunque non è uno spazio di Hilbert
\end{itemize}
\subsubsection{Teorema di proiezione su un convesso chiuso}
Un insieme $K$ si dice \emph{convesso} se $\ \forall x,y\in K,\ \forall \lambda\in (0,1)\implies \lambda x +(1-\lambda)y\in K$
\\Un insieme $K$ si dice \emph{chiuso} se $\ \forall \{x_n\} \subseteq  K:x_n\to x\in H\implies x\in K$
\begin{tcolorbox}
\textbf{Teorema: }Sia $H$ uno spazio di Hilbert, e sia $K\subseteq  H$ un convesso chiuso
\\Allora $\ \forall f\in H$ esiste unico $u\in K$ tale che
\[\|f-u\|=\min_{v\in K}\|f-v\|\]
Inoltre: $u=P_kf\iff (f-u,v-u)\le 0 \ \forall v\in K$
\end{tcolorbox}
\begin{figure}[ht]
    \centering
    \incfig{convesso}
    \caption{Rappresentazione grafica della proiezione su convesso}
    \label{fig:convesso}
\end{figure}
\begin{tcolorbox}
\textbf{Corollario, Teorema di proiezione su un sottospazio chiuso}
\\Sia $H$ uno spazio di Hilbert e $M$ un sottospazio vettoriale chiuso.
\\($M$ è convesso, non è necessariamente chiuso senza ipotesi)
\\Allora: $\ \forall f\in H\exists \text{ unico }u=P_Mf$ tale che
\[\|f-u\|=\min_{v\in M}\|f-v\|\]
Inoltre 
\[u=P_Mf  \iff (f-u,v)=0\ \forall v\in M\]
\end{tcolorbox}
\begin{figure}[ht]
    \centering
    \incfig{sottchiuso}
    \caption{Rappresentazione grafica della proiezione su un sottospazio chiuso}
    \label{fig:sottchiuso}
\end{figure}
\begin{tcolorbox}
	\textbf{Definizione: }Se $(,)$ è un prodotto scalare su $H$ 
	\begin{itemize}
		\item $x\perp y \iff (x,y)=0$ (definizione)
		\item $M^{\perp}:=\{x\in H:(x,y)=0 \ \forall y\in M\} $
	\end{itemize}
\end{tcolorbox}
\textbf{Osservazione: }$f\perp g$ in $L^{2}(0,1)$ se $\int_{0}^{1} fg=0 $ 
\\\textbf{Esempio:} $M=\{\text{funzioni costanti in }L^{2}(0,1)\} $
\[M^{\perp}=\{f\in L^{2}(0,1): \int_{0}^{1} fc=0\ \forall c\in \R \} \]
\[=\{f\in L^{2}(0,1): \int_{0}^{1} f=0 \} \]
\textbf{Osservazione:} 
\[x\perp y\implies \|x+y\|^2=\|x\|^2+\|y\|^2\]
\textbf{Dimostrazione}
\[\|x+y\|^2=(x+y,x+y)=\|x\|^2+2(x,y)+\|y\|^2=\|x\|^2+\|y\|^2\]
\textbf{Osservazione: }$M\cap M^\perp=\{0\} $. Infatti $x\in M\cap M^\perp\implies (x,x)=0$ valido $\iff x=0$
\begin{tcolorbox}
\textbf{Teorema delle proiezioni}
\\Sia $H$ insieme di Hilbert e $M$ un sottospazio chiuso.
\\Allora $\ \forall x\in H\ \exists $ un'unica rappresentazione di $x$ come:
\[x=y+z\text{ con }y\in M\text{ e }z\in M^\perp\]
Inoltre, le applicazioni $x\mapsto y=P_M(x)$, $x\mapsto z=P_{M^\perp}(x)$, sono operatori lineari, limitati, di norma $1$.
\end{tcolorbox}
\textbf{Dimostrazione} 
\\Basta prendere come $y=P_M(x)$ (che esiste dal teorema precedente): sappiamo che $(x-P_M(x),v)=0\ \forall v\in M\implies x-P_M(x)\in M^\perp$, ovvero $z:=x-y\in M^\perp$
\\L'unicità è data da $x=y_1+z_1=y_2+z_2\implies y_1-y_2 \in M, \ z_2-z_1\in M^\perp$ ma $y_1-y_2=z_2-z_1$, dunque per queste ultime due condizioni si avrà $y_1-y_2=z_2-z_1=0$
\\\textbf{Dimostrazione Linearità} 
\\$x_1=y_1+z_1$ 
\\$x_2=y_2+z_2$
\\Dunque $x_1+x_2=y_1+y_2+z_1+z_2$, con $y_k\in M,\ z_k\in M^\perp$, per l'unicità $y_1+y_2=P_M(x_1+x_2),\ z_1+z_2=P_{M^\perp}(x_1+x_2)$
\\\textbf{Limitatezza}\\
$P_m$ limitato: $x=y+y=P_M(x)+P_{M^\perp}(x)$
\[\|x\|^2=\|P_M(x)\|^2+\|P_{M^\perp}(x)\|^2\]\[\implies \|P_m(x)\|\le \|x\|^2\implies P_M\text{ limitato con norma }\le 1\]
$\|P_M\|=1$ : basta prendere $x\in M\implies x=P_M(x)\implies $ vale l'uguaglianza $\|P_M(x)\|=\|x\|$
\subsection{Teoremi di Rappresenzatione}
\subsubsection{Teorema di Reisz}
\textbf{Problema:} Dato $H$ di Hilbert, caratterizzare $H'$ (duale di $H$ ).
\[H'=\{\varphi:H\to \R\text{ lineari e continui}\} =\mathcal L(H,\R)\]
\textbf{Osservazione:} Fissato $u\in H$ possiamo associare ad $u$ un elemento $\varphi_u\in H'$ 
\[\varphi_u(v):H\to \R,\ \varphi_u(v)=(u,v)\ \forall v\in H\]
Verifica che $\varphi_u\in H'$ :
\begin{itemize}
	\item lineare: $\varphi_u(\alpha v_1+\alpha_2v_2)=(u,\alpha_1v_1+\alpha_2v_2=\alpha_1\varphi_u(v_1)+\alpha_2\varphi_u(v_2)$
	\item continuo (limitato): $|\varphi_u(v)|\le M \|v\|$ valida con $M=\|u\|$ per la disuguaglianza di Cauchy Scwhartz

\end{itemize}
Inoltre:
\[\|\varphi_u\|_{H'}=\|u\|_H\]
cioé $M=\|u\|$ è la costante migliore possibile $(v=u)$
\\In conclusione, $H\subseteq  H'$ (immersione isometrica), ovvero la norma si conserva.
\\\textbf{Esempi:} 
\begin{itemize}
	\item $H=\R^n$  $(u,v)=\sum_{k=1}^{n} u_kv_k$  $\varphi_u(v)=\sum_{k=1}^{n} u_kv_k$ 
	\item $H=L^{2}(\Omega)$,  $(u,v)=\int_{\Omega}^{} uv $, $\varphi_u(v)=\int_{\Omega}^{} uv \ \forall v\in L^{2}(\Omega)$
	\item $H=H^1(\Omega)$, $(u,v)=\int_{\Omega}^{} uv+\nabla u\cdot \nabla v $
		\[\varphi_u(v)=\int_{\Omega}^{} uv+\nabla u\cdot \nabla v \ \forall v\in H^1(\Omega)\] 
\end{itemize} 

\begin{tcolorbox}
\textbf{Teorema di Riesz }
\\Sia $H$ spazio di Hilbert e sia $\varphi\in H'$.
\\Allora, esiste unico  $u\in H$ tale che $\varphi=\varphi_u$ ovvero
\[\varphi(v)=(u,v)\ \forall v\in H\]
Inoltre
\[\|\varphi\|_{H'}=\|u\|_H\]
Dunque $H"="H'$.
\end{tcolorbox}
\subsubsection{Forme bilineari}
\begin{tcolorbox}
\textbf{Definizione: }Sia $H$ di Hilbert. Una \emph{forma bilineare} su $H$ è per definizione un'applicazione 
\[a:H\times H\to \R\]
tale che:
\begin{itemize}
	\item $a(\alpha_1u_1+\alpha_2u_2,v)=\alpha_1a(u_1,v)+\alpha_2a(u_2,v)$
\end{itemize}
\end{tcolorbox}
\textbf{Esempi:} 
\begin{itemize}
	\item In $H$ Hilbert qualsiasi $a(u,v)=(u,v)$ 
	\item $H=H^1(\Omega)$, $a(u,v)=\int_{\Omega}^{} uv+\nabla u\cdot \nabla v $, $a(u,v)=\int_{\Omega}^{} uv $, $a(u,v)=\int_{\Omega}^{}  \nabla u\cdot \nabla v$

\end{itemize}
\begin{tcolorbox}
\textbf{Definizione: }Sia $a:H\times H\to \R$ una forma bilineare
\begin{itemize}
	\item a \emph{simmetrica} se
\end{itemize}
\[a(u,v)=a(v,u)\ \forall u,v\in H\]
\begin{itemize}
	\item $a$ \emph{continua} se
\end{itemize}
\[\exists C>0 \text{ tale che }|a(u,v)|\le C\|u\|\|v\|\ \forall u,v\in H\]
\begin{itemize}
\item $a$ \emph{coerciva} se 
	\[\exists \alpha>0\text{ tale che }a(u,v)\ge \alpha \|u\|^2\ \forall u\in H\]
\end{itemize}

\end{tcolorbox}
\textbf{Esempi:} 
\\1) In $H$ di Hilbert qualsiasi, $a(u,v)=(u,v)$ è
\begin{itemize}
	\item simmetrica (per definizione di prodotto scalare
	\item continua (limitata per Cauchy Schwartz)
	\item coerciva ($(u,u)=1\cdot \|u\|^2$)
\end{itemize}
2) In $H=H_0^1(\Omega)$, $a(u,v)=\int_{\Omega}^{} \nabla u\cdot \nabla v $.
\begin{itemize}
	\item simmetrica
	\item continua: (tramite Holder)
		\[|a(u,v)|=\bigg|\int_{\Omega}^{} \nabla u\cdot \nabla v\bigg|\le \int_{\Omega}^{} |\nabla u\cdot \nabla v|\le \|\nabla v\|_2\|\nabla v\|_2\]
		\[\le \|u\|_{H^1}\|v\|_{H^1}\ \forall u,v\in H^1(\Omega)\]
	\item coerciva: (per Poincaré)
		\[a(u,u)=\int_{\Omega}^{} |\nabla u|^2\ge \alpha \|u\|^2_{H^1}\] 
\end{itemize}
\textbf{Osservazione:} $a(u,v)$ non sarebbe coerciva su $H^1(\Omega)$ poiché non vale la disuguaglianza di Poicaré (verificabile con $u=\text{cost}> 0$ 
\subsubsection{Teorema di Lax Milgram}
\begin{tcolorbox}
\textbf{Teorema di Lax-Milgram}
\\Sia $H$ Hilbert, e sia $\varphi'\in H'$ 
\\Sia $a:H\times H\to \R $ forma bilineare simmetrica, continua e coerciva.
Allora esiste unico $u\in H$ tale che
\[\varphi(v)=a(u,v)\ \forall v\in H\]
Inoltre $u$ è caratterizzata dalla seguente proprietà: 
\[E(v):=\frac{1}{2}a(v,v)=\varphi(v)\ \forall v\in H\]
si ha 
\[\min_{v\in H}E(v)=E(u)\]
\end{tcolorbox}
\textbf{Esempio} ($\Omega$ limitato)
\\$H=H_0^1(\Omega)$, $a(u,v)=\int_{\Omega}^{} \nabla u\cdot \nabla v $, $\varphi(v)=\int_{\Omega}^{} fv $ dove $f\in L^{2}(\Omega)$ 
\[\varphi\in H':\bigg|\int_{\Omega}^{} fv\bigg|\le \int_{\Omega}^{} |fv|\le \|f\|_{L^{2}(\Omega)}\|v\|_{L^{2}(\Omega)}\le \|f\|_{L^{2}(\Omega)}\|v\|_{H^1}\]
Per Lax-Milgram: $\exists u$ unico $u\in H^1_0(\Omega)$ tale che $\varphi(v)=a(u,v)\ \forall v\in H_0^1(\Omega)$ 
\[\int_{\Omega}^{} fvdx=\int_{\Omega}^{} \nabla u\cdot \nabla vdx\ \forall v\in H_0^1(\Omega)\]
Quest'ultima è una formulazione debole del seguente problema:
\[\begin{cases}
	-\nabla u=f&\text{ in }\Omega
	\\u=0&\text{ su }\partial\Omega
\end{cases}\]
Inoltre il teorema dice che $u$ risolve 
\[\min_{v\in H_0^1(\Omega)}E(v)=\frac{1}{2} \int_{\Omega}^{} |\nabla v|^2-\int_{\Omega}^{} fv\]
\subsubsection{Commenti sulla proprietà variazionale di $u$ }
$E(u+\varepsilon v)=\frac{1}{2}a(u+\varepsilon v,u+\varepsilon v)-\varphi(u+\varepsilon v)$
\[=\frac{1}{2}[a(u,u)+2\varepsilon a(u,v)+\varepsilon^2a(v,v)]-\varphi(u)-\varepsilon\varphi(v)\]
\[=[\frac{1}{2}a(u,u)-\varphi(u)]+\varepsilon[a(u,v)-\varphi(v)]+ \frac{\varepsilon^2}{2}a(u,v)\]
\[\implies E(u+\varepsilon v)-E(u)=\varepsilon [a(u,v)-\varphi(v)]+o(\varepsilon)\]
\[\implies \lim_{\varepsilon \to 0} \frac{E(u+\varepsilon v)-E(u)}{\varepsilon}=a(u,v)-\varphi(v)=0\]
Se $a(u,v)=\varphi(v)\ \forall v\in H$, 
\[E(u+\varepsilon v)-E(u)= \frac{\varepsilon^2}{2}a(v,v)\ge 0\]
Quindi $u$ minimizza $E$.
\\Viceversa, se $u $ minimizza $E$ :
\[E(u+\varepsilon v)\ge E(u)\ \forall v\in H\ \forall \varepsilon\in \R\]
\[\implies a(u,v)-\varphi(v)=0\ \forall v\in H\]


\section{Equazioni alle derivate parziali}
\subsection{Formulazione variazionali di problemi ellittici}
\[-a\nabla ^2u+cu=f\text{ in }\Omega\subseteq  \R^n aperto limitato regolare,\ u=u(x_1,\ldots,x_n)\]
\textbf{Ipotesi}
\begin{itemize}
	\item $a>0$ 
	\item $c\in L^{\infty}(\Omega)$ 
	\item $f(x)\in L^{2}(\Omega)$
\end{itemize}
Se $c=0,\ a=1\implies -\nabla ^2 u=f$ (Equazione di Poisson)\\
\textbf{Condizione di Dirichlet (omogenea):} $u=0$ su $\partial\Omega$
\\\textbf{Condizione di Neumann (omogenea):} $\frac{\partial u}{\partial \nu} =0$ su $\partial\Omega$
\subsubsection{PDE ellittiche del secondo ordine}
ODE lineare del 2° ordine
\[au''+bu'+cu=f\]
Alle derivate parziali (PDE del 2° ordine), $u=u(x),\ x\in \R^n$
\[-A(x)\cdot \nabla ^2 u(x)+b(x)\cdot \nabla u(x)+cu=f\]
si dice ellittica se $A$ è definita positiva
\[\sum_{i,j=1}^{n} A_{i,j}(x)\xi_i\xi_j\ge 0\ \forall \xi \in \R^n\]
In particolare se $A(x)=\text{I}$ 
\[\sum_{i,j=1}^{n} A_{i,j}(x)u_{x_i,x_j}=\sum_{i=1}^{n} u_{x_i,x_j}=\Delta u\]
\subsubsection{Formulazione variazionale del problema di Dirichlet}
$(D)_c$ Trovare $u\in C^2(\overline\Omega)$ tale che:
\[\begin{cases}
	-a\Delta u+cu=f&\text{ in }\Omega
	\\u=0&\text{ su }\partial \Omega
\end{cases}\]
$(D)_v$ Trovare $u\in H_0^1(\Omega)$ tale che:
\[\int_{\Omega}^{} a\nabla u\cdot \nabla v+cuv=\int_{\Omega}^{} \ \forall v\in H_0^1(\Omega)\]
\begin{tcolorbox}
	\textbf{Proposizione (D):} Nelle ipotesi sopra:
	\begin{enumerate}
		\item $u$ sol. classica $\implies $ u sol. variazionale
		\item $u$ sol. variazionale, $c,f$ continue, $u\in C^2(\overline\Omega)\implies u$ sol. classica
	\end{enumerate}
\end{tcolorbox}

\subsubsection{Formulazione variazionale del problema di Neumann}
$(N)_c$ Trovare $u\in C^2(\overline\Omega)$ tale che:
\[\begin{cases}
	-a\Delta u+cu=f&\text{ in }\Omega
	\\\frac{\partial u}{\partial \nu} =0&\text{ su }\partial \Omega
\end{cases}\]
$(N)_v$ Trovare $u\in H^1(\Omega)$ tale che:
\[\int_{\Omega}^{} a\nabla u\cdot \nabla v+cuv=\int_{\Omega}^{} \ \forall v\in H^1(\Omega)\]
\begin{tcolorbox}
	\textbf{Proposizione (N):} Nelle ipotesi sopra:
	\begin{enumerate}
		\item $u$ sol. classica $\implies $ u sol. variazionale
		\item $u$ sol. variazionale, $c,f$ continue, $u\in C^2(\overline\Omega)\implies u$ sol. classica
	\end{enumerate}
\end{tcolorbox}
\subsubsection{Esistenza delle soluzioni}
\begin{tcolorbox}
	\textbf{Teorema: }Nelle ip. sopra definite, il problema $(D)_v$: trovare $u\in H_0^1(\Omega)$ tale che
	\[\int_{\Omega}^{} a\nabla u\cdot \nabla v+cuv=\int_{\Omega}^{} fv\ \forall v\in H_0^1(\Omega)\]
	Ammette una e una sola soluzione. Inoltre $u$ è caratterizzata nel modo seguente:
	\[\min_{v\in H_0^1(\Omega)}E(v):=\frac{1}{2}\int_{\Omega}^{} (a|\nabla v|^2+cv^2)-\int_{\Omega}^{} fv\]

\end{tcolorbox}
\textbf{Dimostrazione} 
\\Considero $H=H^1_0(\Omega)$, munito di $\|\nabla u\|_{L^{2}(\Omega)}=\|\nabla u\|_2$
\begin{itemize}
	\item $\varphi(v)=\int_{\Omega}^{} fv\ \forall v\in H $ 
	\item $b(u,v)=\int_{\Omega}^{} a\nabla u\cdot \nabla v+cuv $ 

\end{itemize}
$\varphi$ è lineare continuo, $b(u,v)$ è bilineare simmetrica, continua, coerciva
Per Lax-Milgram $\exists $ unico $u\in H$ tale che 
\[\varphi(v)=b(u,v)\ \forall v\in H\]
ovvero:
\[\int_{\Omega}^{} a\nabla u\cdot \nabla v +cuv=\int_{\Omega}^{}fv \ \forall v\in H\]
Inoltre $u$ risolve
\[\min_{H}E(v):=\frac{1}{2}b(u,v)-\varphi(v)\]
Verifica ip. Lax-Milgram:
\begin{itemize}
	\item $\varphi$ (lineare) continuo: $\exists M:|\varphi(v)|\le M\|v\|_H$
\end{itemize}
\[\int_{\Omega}^{} fv|\le \int_{\Omega}^{} |fv|\le_H\|f\|_2\|v\|_2\le_PC_p(\Omega)\|f\|_2\|\nabla v\|_2\]
Si avrà dunque $M=C_p(\Omega)\|f\|_2$ e $\|\nabla v\|_2=\|v\|_2$
\begin{itemize}
	\item $b(u,v)$ è bilineare simmetrica (dimostrazione semplice)
\end{itemize}
\begin{itemize}
	\item $b(u,v)$ continua

\end{itemize}
\[|b(u,v)|=\bigg|\int_{\Omega}^{} a\nabla u\cdot \nabla v+cuv\bigg|\le \int_{\Omega}^{} |a\nabla u\cdot \nabla v+cuv|\le \int_{\Omega}^{} a|\nabla u\cdot \nabla v|+c|uv| \]
\[\le \int_{\Omega}^{} a|\nabla u\cdot \nabla v|+\|c\|_{\infty}\int_{\Omega}^{} |uv|   \le a \|\nabla u\|_2 \|\nabla u\|_2+\|c\|_\infty\|u\|_2\|v\|_2\]
\[\le a \|\nabla u\|_2\|\nabla v\|_2+\|c\|_\infty C_p^2(\Omega)\|\nabla u\|_2\|\nabla v\|_2\]
\[=(a+\|c\|_\infty C_p(\Omega))\|\nabla u\|_2\]
finire\\
\begin{itemize}
	\item $b(u,v)$ coerciva

\end{itemize}
$b(u,v)=\int_{\Omega}^{} a|\nabla u|^2+cu^2\ge \int_{\Omega}^{} a|\nabla u|^2=a\|\nabla u\|  $ finire

\subsubsection{Esistenza delle soluzioni per Neumann}
\begin{tcolorbox}
	\textbf{Teorema: }Nelle ip. sopra definite, supponiamo anche $c(x)>0$ il problema $(D)_v$: trovare $u\in H^1(\Omega)$ tale che
	\[\int_{\Omega}^{} a\nabla u\cdot \nabla v+cuv=\int_{\Omega}^{} fv\ \forall v\in H^1(\Omega)\]
	Ammette una e una sola soluzione. Inoltre $u$ è caratterizzata nel modo seguente:
	\[\min_{v\in H_0^1(\Omega)}E(v):=\frac{1}{2}\int_{\Omega}^{} (a|\nabla v|^2+cv^2)-\int_{\Omega}^{} fv\]

\end{tcolorbox}
\textbf{Dimostrazione} 
\\Analoga al caso di Dirichlet lavorando su $H=H^1(\Omega)$ munito di $\|u\|_{H^1}=\|u\|_{L^{2}(\Omega)}+\|\nabla u\|_{L^{2}(\Omega)}$
\\Tranne che per la coercività di $b$:
\[b(u,u)=\int_{\Omega}^{} a|\nabla u|^2+cu^2\ge \alpha\|u\|_{H^1}^2?\]
\[\ge \int_{\Omega}^{} a|\nabla u|^2+c_0u^2\ge \min \{a,c_0\} \int_{\Omega}^{} |\nabla u|^2+|u|^2\]
\subsubsection{}
\[\int_{\Omega}^{} \text{div}X=\int_{\partial \Omega}^{} X\cdot \nu \ \forall X\in C^1(\Omega)\]
$X=v\nabla u$, $v\in C^1$, $u\in C^2$
\[\text{div}(v\nabla u)=\]
completare
\[\int_{\Omega}^{} \nabla u\cdot \nabla v+v\nabla u=\int_{\partial \Omega}^{} v \frac{\partial u}{\partial \nu} \ v\in C^1,u\in C^2\]
(Formula di Gauss-Green)

\begin{tcolorbox}
\textbf{Lemma di DuBois-Raymond}
\\Se $u\in C(\overline{\Omega})$ è tale che:
\[\int_{\Omega}^{} u\varphi=0\ \forall \varphi\in C_0^\infty(\Omega)\implies u\equiv0\text{ in }\Omega\] 
\end{tcolorbox}
Per dimostrarlo si procede per assurdo
\subsubsection{Dimostrazione proposizione di Dirichlet}
\begin{enumerate}
	\item Sia $u$ sol. di $(D)_c$

\end{enumerate}
Allora $u\in H_0^1(\Omega)$ $(u,v\nabla u\in C\overline{\Omega})\subseteq  L^{2}(\Omega),u=0\text{ su }\partial\Omega)$
Moltiplico l'equazione per $v\in C_0^\infty(\Omega)$ 
\[-a\Delta u\cdot v+cuv=fv\text{ in }\Omega\]
Integrando
\[\int_{\Omega}^{} -a\Delta u\cdot v+cuv=\int_{\Omega}^{} fv\]
Per Gauss Green
\[\int_{\Omega}^{} av \frac{\partial u}{\partial \nu}=0\]
\[\int_{\Omega}^{} a\nabla u\cdot \nabla v_n+cuv_n=\int_{\Omega}^{} fv_n\ \forall v\in C_0^\infty(\Omega) \]
Data $v\in H_0^1(\Omega),\exists \{v_n\} \subseteq  C_0^\infty(\Omega):v_n\to ^{H^1}v$ (per definizione di $H^1_0(\Omega)$ 
\\Passando al limite
\[\int_{\Omega}^{} a\nabla u\cdot \nabla v+cuv=\int_{\Omega}^{} fv\ \forall v\in H_0^1(\Omega) \]
Tale limite si dimostra
\[\bigg|\int_{\Omega}^{} fv_n-fv\bigg|\le \int_{\Omega}^{} |f(v_n-v)|\le_H \|f\|_2 \|v_n-v\|_2\to 0\]
In modo analogo si verificano le altre convergenze
\begin{enumerate}
	\item 2) Sia $u\in H_0^1(\Omega)$ sol. variazionale, supponendo $u\in C^2(\overline{\Omega})$, ($c,f$ continue)
\end{enumerate}
$u=0$ su $\partial\Omega$ 
\\Sappiamo che
\[\int_{\Omega}^{} a\nabla v\cdot \nabla u+cuv=\int_{\Omega}^{} fv\ \forall v\in H^1_0(\Omega)\text{ in particolare }\ \forall v\in C_0^\infty(\Omega)\]
Tramite Gauss Green
\[\int_{\Omega}^{} -a\Delta u\cdot v+cuv-fv=0\ \forall v\in C_0^\infty(\Omega)\]
\[\implies \int_{\Omega}^{} (-a\Delta u+cu -f)v=0\ \forall v\in C_0^\infty(\Omega)\]
La funzione nelle parentesi è continua su $\overline{\Omega}$
\\Per il lemma di DBR
\[\implies -a\Delta u+cu-f=0\text{ in }\Omega\]
\section{Serie di Fourier in spazi di Hilbert}
\begin{tcolorbox}
	\textbf{Definizione: }Sia $H$ di Hilbert. Una famiglia di vettori $\{u_n\} \subseteq  H$ si dice \emph{sistema ortogonale} se $(u_n,u_m)=0\ \forall n\neq m$. 
	\\Si dice poi \emph{sistema ortonormale} se è ortogonale e $(u_n,u_n)=1\ \forall n$
\end{tcolorbox}
\textbf{Esempi:} 
\begin{itemize}
	\item $H=\R^3$: $e_1=(1,0,0),\ e_2=(0,1,0),\ e_3=(0,0,1)$ 
	\item $H=l^2=\{(x_n)_{n\in \N}:x_n\in \R\text{ tali che }\sum_{n\ge 0}^{} x_n^2<+\infty\} $ è uno spazio vettoriale
		\[\|x\|_{l^2}=\bigg(\sum_{n\ge 0}^{} x_n^2 \bigg)^{\frac{1}{2}}\]
		è di Hilbert poiché $((x_n),(y_n))=\sum_{}^{} x_ny_n$ 
		\\$e_n=(0,\ldots,1,\ldots,0)$
\end{itemize}
\begin{tcolorbox}
	\textbf{Definizione: }Sia $H$ di Hilbert e sia $(u_n)$ sistema ortonormale.
	\\Dato $u\in H$ 
	\begin{itemize}
		\item $(u,u_n)\in \R$ \emph{coefficienti di Fourier} di $u$ (rispetto a $(u_n)$ )
		\item $\sum_{n}^{} (u,u_n)u_n$ \emph{serie di Fourier} di $u$ (rispetto a $(u_n)$ )
	\end{itemize}
\end{tcolorbox}
\textbf{Esempi} 
\begin{itemize}
	\item $H=\R^3$, $\{e_1\} $, $(u,e_1)e_1=P(u)\text{ su }\left< e_1 \right> $ 
	\item "", $\{e_1,e_2\} $, $(u,e_1)e_1+(u,e_2)e_2=P(u)\text{ su }\left< e_1,e_2 \right> $ 
	\item "", $\{e_1,e_2,e_3\} $, $(u,e_1)e_1+\ldots=P(u)\text{ su }\left< e_1,e_2,e_3 \right> $ 
	\item $H=l^2$, $\{e_1\} =\{(1,0,\ldots,0)\} $, $(u,e_1)e_1=P_{\left< e_1 \right> }(u)$
		\\""$\{e_i\} $ pari $\sum_{k}^{} (u,e_{2k})e_{2k}$
		\\""$\{e_n\} $ con $n$ qualsiasi $\sum_{k}^{} (u,e_k)e_k=u$ 
\end{itemize}
\begin{tcolorbox}
\textbf{Teorema di convergenza per serie di Fourier}
\\Sia $H$ Hilbert, sia $\{u_n\} $ sistema ortonormale fissato.
\\Dato $u\in H$, la serie di Fourier di $u$ converge in $H$ e 
\[\sum_{n}^{} (u,u_n)u_n=u'\]
Dove $u'$ è la proiezione ortogonale di $u$ su $M$, dove $M$ è la chiusura del sottospazio generato dal sistema.
\end{tcolorbox}
\subsubsection{Convergenza in H}
$\sum_{n}^{} (u,u_n)u_n $ corrisponde a $S_N(u)=\sum_{n=0}^{N} (u,u_n)u_n$, converge a $u'$ se 
\[\exists \lim_{N \to +\infty} S_N(u)=u' \iff \lim_{N \to +\infty} \|S_N(u)-u'\|=0\]
\subsubsection{Sottospazio generato}
Il sottospazio generato, indicato con $\left< u_n \right> $ è definito come
\[\left< u_n \right> :=\{\text{combinazioni lineari degli }u_n\} \]
\[M=\overline{\left< u_n \right> }:=\{\text{limiti di comb. lineari degli }u_n\} \]
$M$ è un sottospazio chiuso.
\subsubsection{Disuguaglianza di Bessel}
\begin{tcolorbox}
	\textbf{Teorema: }Sia $H$ Hilbert, e sia $(u_n)$ sistema ortonormale, dato $u\in H$, vale 
	\[\sum_{n}^{} (u,u_n)^2\le \|u\|^2\]
\end{tcolorbox}
\textbf{Dimostrazione} 
\\Fisso $N\in \N$ e mostriamo
\[\sum_{n=0}^{N} (u,u_n)^2\le \|u\|^2\]
la tesi è dimostrata passando al limite, dunque:
\[0\le \|u-\sum_{n\le N}^{} (u,u_n)u_n\|^2=(u-\sum_{n\le N}^{} (u,u_n)u_n,u-\sum_{n\le N}^{} (u,u_n)u_n)=\]
\[=\|u\|^2 -2 \sum_{n\le N}^{} (u,u_n)^2+\sum_{n\le N}^{} (u,u_n)^2\]
\[=\|u\|^2-\sum_{n\le N}^{} (u,u_n)^2\]
L'ultima somma vale poiché siamo in un sistema ortonormale:
\[((u,u_1)u_1+(u,u_2)u_2,((u,u_1)u_1+(u,u_2)u_2)=(u,u_1)^2+(u,u_2)^2\]
\subsubsection{Dimostrazione teorema di convergenza delle serie di Fourier}
Per dimostrare la convergenza della serie, basta mostrare che $S_N(u)$ è di Cauchy:
\[\ \forall \varepsilon\exists \nu:\|S_N-S_M(u)\|^2<\varepsilon\ \forall N,M\ge \nu\]
ovvero: (supposto $N>M$)
\[\|S_n(u)-S_M(u)\|^2=(S_N(u)-S_M(u),S_N(u)-S_M(u))=\]
\[\bigg(\sum_{n=M+1}^{N} (u,u_n)u_n,\sum_{n=M+1}^{N} (u,u_n)u_n\bigg)\]
\[=\sum_{n=M+1}^{N} (u,u_n)^2=|T_N(u)-T_M(u)|\]
dove $t_N=\ldots$ 
Bessel $\implies \{T_N(u)\} $ è di Cauchy
\\Essendo in un Hilbert $S_N(u)$ converge
\\Sia ora $u':=\sum_{n}^{} (u,u_n)u_n$ 
\\Per dimostrare $u'=P_M(u)$ basta mostrare che
\begin{enumerate}
	\item $u'\in M$ 
	\item $u-u'\in M^\perp$
\end{enumerate}
Per l'unicità nel teorema delle proiezioni, $u'=P_M(u),\ u-u'=P_{M^\perp}(u)$
\\Infatti
\begin{enumerate}
	\item $u'\in M$ vale per costruzione: completare
	\item Per mostrare che $u-u'\in M^\perp$ basta far vedere che $(u-u',u_n)=0\ \forall n$, questo assicura che $u-u'$ sarà ortogonale a tutti i limiti delle combinazioni lineari degli $u_n$, ovvero a tutti gli elementi di $M$.\[(u-u',u_m)=(u-\sum_{n}^{} (u,u_n)u_m,u_m)=(u,u_m)-(u,u_m)(u_m,u_m)=0\]
\end{enumerate}
\divider
\begin{tcolorbox}
	\textbf{Definizione: }Sia $H$ di Hilbert e sia $(u_n)$ sistema ortonormale. 
	Si dice che $(u_n)$ è \emph{sistema completo} se è massimale rispetto all'inclusione.
	\\Ovvero: $\not\exists (v_n)$ sistema ortonormale che che contenga propriamente $(u_n)$
\end{tcolorbox}
\begin{tcolorbox}
	\textbf{Proposizione di caratterizzazione di sistemi ortonormali completi}
	\\Sia $(u_n)$ ortonormale in un Hilbert.
	\\Sono equivalenti:
	\begin{enumerate}
		\item $(u_n)$ è completo
		\item $u\in H:(u,u_n=0\ \forall n\implies u=0$ 
		\item Posto $M:=\overline{\left< u_n \right> }$, si ha $M\equiv H$
		\item $\sum_{n}^{} (u,u_n)u_n=u\ \forall u\in H$
		\item $\sum_{n}^{} (u,u_n)(v,u_n)=(u,v)\ \forall u,v\in H$ (identità di Parseval)
		\item $\sum_{n}^{} (u,u_n)^2=\|u\|^2\ \forall u\in H$ (identtà di Bessel)
	\end{enumerate}
\end{tcolorbox}
\textbf{Dimostrazione} 
\\$(1)\iff (2)\implies (3)\implies (4)\implies (5)\implies (6)\implies (2)$
\\\textbf{Dimostrazione 1}
\\Se è falsa la 2 implica che è falsa la 1
\[\exists u\in H:(u,u_n)=0\ \forall n\text{ MA }u\neq 0\]
Allora $(u_n)$ non è massimale.
Se è falsa la 1 implica che è falsa la 2
\\se $(u_n)$ non è massimale, posso aggiungere almeno un elemento $\implies \exists u\in H$ per cui la 2 è falsa.
\\\textbf{Dimostrazione 2 implica 3}
\\Per mostrare $M\equiv H$, basta mostrare $M^\perp=\{0\} $, vero per la 2
\[\text{ se }u\in M^\perp\text{ allora }u=0\]
\textbf{Dimostrazione 3 implica 4} 
\\Se $M$ coincide con $H$, allora $u'=P_M(u)=u$ 
\\\textbf{Dimostrazione 4 implica 5}
Per la quattro ogni elemento è la sua serie di Fourier, dunque
\[(u,v)=(\sum(u,u_n)u_n,\sum(v,u_n)u_n)=\sum(u,u_n)(v,u_n)\]
\\\textbf{Dimostrazione 5 implica 6}
\\Prendere $u=v$ in Parseval
\\\textbf{Dimostrazione 6 implica 2}
\\Se $(u,u_n)=0\ \forall n, \sum0=0$

\section{Funzioni olomorfe}
\begin{tcolorbox}
	\textbf{Definizione} \\
	$f$ si dice olomorfa su $\Omega$ se è derivabile in $z_0\forall z_0\in\Omega$
\end{tcolorbox}
\subsection{Invertibilità locale}
\begin{tcolorbox}
\textbf{Teorema}
\\Sia $f:\Omega \subseteq\C\to \C$ olomorfa in $\Omega$, e sia $z_0\in\Omega$ tale che $f'(z_0)\ne0$ allora f è "localmente invertibile in $z_0$"\\
($\exists u(z_0)$ tale che $f|_{u(z_0)}$ invertibile)
\\E la funzione inversa $f^{-1}$ è derivabile in senso complesso in $f(z_0)$ e 
\[(f^{-1})'|_{z_0}=\frac{1}{f'(z_0)}\]	
\end{tcolorbox}
\textbf{Dimostrazione}\\
$\Phi(u,v)$ definito su $\Omega \subseteq \R^2\to \R^2, (x_0,y_0)\in\Omega$, se $detJ\Phi(x_0,y_0)\ne 0\implies\Phi$ "localmente invertibile" e 
\[J\Phi^{-1}(\Phi(x_0,y_0))=(J\Phi(x_0,y_0))^{-1}\]
Dunque se $f=u+iv$ si riformula il teorema con $\Phi=(u,v)$
\[J\Phi(x_0,y_0)=\begin{pmatrix}
	u_x&u_y\\v_x & v_y
\end{pmatrix}
=\begin{pmatrix}
	\alpha & -\beta \\ \beta & \alpha
\end{pmatrix}
\implies detJ\Phi(x_0,y_0)=\alpha^2+\beta^2=|f'(z_0)|^{2}
\]
Poiché $f'=\alpha +i\beta$ e l'ipotesi del teorema è che $|f'(z_0)|^2\ne 0$
\[J\Phi^{-1}(\Phi(x_0,y_0))=\frac{1}{\alpha^2+\beta^2}\begin{pmatrix}
	\alpha&\beta\\-\beta &\alpha
\end{pmatrix}
\]
\[\implies(f^{-1})'(f(z_0))=\frac{\alpha}{\alpha^2+\beta^2}-i\frac{\beta}{\alpha^2+\beta^2}= \frac{\overline{f'(z_0)}}{|f'(z_0)|^2}=\frac{1}{f'(z_0)}\]
\\
\subsection{Ricerca delle primitive - antiderivazione}
\textbf{Problema: }Data $f:\Omega \subseteq \C\to \C$ esiste? unica? $F:\Omega\subseteq\C$ olomorfa in $\Omega$ tale che \[F'(z)=f(z)\]
Una tale $F$ si dice \textbf{primitiva} di $f$.\\
\\\textbf{Richiamo - Teorema fondamentale del calcolo: }Data $f:(a,b)\in\R\to \R$ continua, allora una primitiva di $f$ è data da \[F(x)=\int_{a}^{x} f\]
E tutte le altre primitive si ottengono aggiungendo una costante reale

\noindent\rule{\textwidth}{0.5pt}
\textbf{Unicità: }una primitiva, se esiste, è univocamente determinata a meno di costante additiva.
\begin{itemize}
	\item $F$ primitiva di $f$, $\lambda\in\C\implies F+\lambda \text{ primitiva di }f$ poiché $(F+\lambda)'=F'+\lambda'=f$
	\item $F_1,F_2$ primitive di f $\implies\exists \lambda\in\C:F_1-F_2=\lambda$
\end{itemize}
$G:=F_1-F_2$, Tesi: $G$ è costante, Dim: \[G'=(F_1-F_2)'=f-f=0\]
$G=u+iv$ $G'=u_x-iu_y=v_y+iv_x$ $G'=0\implies \nabla u(x_0,y_0)=\nabla v(x_0,y_0)=\underline 0$
\\$\implies u \text{ costante},\ v \text{ costante}$
\\N.B vale se $\Omega$ è connesso 

\noindent\rule{\textwidth}{0.5pt}
\textbf{Esistenza}
\\$f=u+iv$, $F=U+iV$ ( $f$ data, $F$ incognita )
\\$F'=U_x-iU_y=V_y+iV_x=f=u+iv$
\[
	\implies \begin{cases}
	U_x=u\\U_y=-v
\end{cases}
	\begin{cases}
	V_x=v\\V_y=u
\end{cases}
\]
ovvero $U$ potenziale per $w_1:=udx-vdy$
\\e $V$ potenziale per $w_2:=vdx+udy$
\\Concludiamo che dire $f$ ammette primitive $\iff\omega_1,\omega_2$ esatte $\implies\omega_1,\omega_2$ chiuse
\\Ovvero se la funzione $f$ soddisfa le condizioni di Cauchy-Riemann dunque se $f$ olomorfa
\[f\text{ ammette primitive }\iff\omega_i \text{ esatte }\implies f \text{ olomorfa }\iff w_i\text{ chiuse }\]
L'implicazione inversa è vera se $\Omega$ è semplicemente connesso
\\Dunque $F=U+iV$ dove $U$ potenziale per $\omega_1$, $V$ potenziale per $\omega_2$
\\\textbf{Nota: } $\omega$ chiusa $\implies \oint_\gamma \omega$ non cambia se sostituisco $\gamma$ con un circuito omotopo. 
\begin{tcolorbox}
	\textbf{Definizione} 
	\\Data $f:\Omega\subseteq\C\to \C$, dato $\gamma$ cammino in $\Omega$ parametrizzata da una funzione $r:[a,b]\to \Omega$, $r(t)=r_1(t)+ir_2(t)$
	\[\int_{\gamma}^{} f(z)dz:=\int_{a}^{b} f(r(t))r'(t)dt\]

\end{tcolorbox}
	\[=\int_{a}^{b} (u+iv)(r_1'+ir_2')dt=\int_{a}^{b} (ur_1'-vr_2')+i \int_{a}^{b} vr_1'+ur_2')\]
	\[=\int_{\gamma}^{} \omega_1+i \int_{\gamma}^{}\omega_2\]
Riformulazione del calcolo di $F$
\[F(z)=\int_{\gamma:z_0\to z}^{} f \]
Questo implica che \[\oint_\gamma f=0\]
\textbf{Teorema di Morera}
\\ $\oint_\gamma f=0 \ \forall \gamma $ circuito $\subseteq \Omega \implies f$ olomorfa
\\\\\textbf{Teorema di Cauchy}
\\$f$ olomorfa su $\Omega\implies\oint f$ non cambia se sostituisco un circuito $\gamma \subseteq\Omega$ con uno ad omotopo (In particolare, se $\gamma$ omotopa ad un punto $\oint_\gamma f=0$)

\section{Funzioni analitiche in campo complesso}
\begin{tcolorbox}	
\textbf{Definizione} 
\\ $f:\Omega$ aperto $\subset \C\to \C$ si dice analitica su $\Omega$ se $\forall z_0\in\Omega,\ \exists\ u(z_0)$ tale che
\[f(z)=\sum_{k\ge 0}^{} c_k(z-z_0)^k\ \ \forall z\in u(z_0)\]
\end{tcolorbox}
\subsection{Serie di potenze in $\C$}
\[\sum_{k\ge 0}^{} c_k(z-z_0)^k\]
\[S_N(z):=\sum_{k=0}^{N} c_k(z-z_0)^k\]
\textbf{Tipi di convergenza}\\ 
La serie conv. puntualmente in $z\in\C$ se \[\lim_{N \to + \infty} S_N(z)\in\C\]
La serie conv. uniformemente in $\Omega$ a $S(z)$ se \[\exists\lim_{N \to +\infty} \text{sup}_{z\in\Omega}|S_N-S(z)|=0\]
La serie conv. assolutamente in $z\in\C$ se converge \[\sum_{k\ge 0}^{} |c_k| |z-z_0|^k\]
\textbf{Dominio di convergenza della serie} 
\[\mathcal D:=\{z\in\C:\text{ la serie converge puntualmente in }z\}\] 
\textbf{Proprietà} 
\\1. $\text{int}(\mathcal D)=\{z\in\C: |z-z_0|<R\}$ dove $R:=$raggio di convergenza
\\$\implies$La serie converge assolutamente in $\text{int}(\mathcal D)$
\\$\implies$La serie converge uniformemente su $\{|z-z_0|\le \rho,\forall\rho<R\} $
\\
\\2. $R=\frac{1}{L}$ dove \[L=\lim_{k \to +\infty} (\text{sup})\sqrt{|c_k|} \]
Con la convenzione $\frac{1}{0}=+\infty$, $\frac{1}{+\infty}=0$
\\\\3. La serie delle derivate n-esime
\[\sum_{k\ge 0}^{} D^n(c_k(z-z_0)^k)\]
ha lo stesso raggio di convergenza della serie di partenza
\\\divider
\\\textbf{Calcolo dei coefficienti $c_k$} 
\[f(z)=\sum_{k\ge 0}^{} c_k(z-z_0)^k=c_0+c_1(z-z_0)+c_2(z-z_0)^2+\ldots\]
\[f'(z)=\sum_{k\ge 1}^{} kc_k(z-z_0)^{k-1}=c_1+2c_2(z-z_0)+\ldots\]
\[f^{(n)}(z)=\sum_{k\ge n}^{} k(k-1)\ldots(k-n+1)c_k(z-z_0)^{k-n}\]
Si ottiene infine
\[f(z_0)=c_0,\ f'(z_0)=c_1,\ f''(z_0)=2c_2\]
\[f^{(n)}(z_0)=n!c_n\]
\[\implies c_n=\frac{f^{(n)}(z_0)}{n!}
\]
\subsection{Un altro modo di calcolare i coefficienti $c_k$} 
Sia $f$ analitica in $\Omega$, sia $z_0\in\Omega$, $R:=$ raggio di conv.
\\Fissato $r\in(0,R)$, e fissato $k\ge 0$, calcoliamo 
\[I_k:=\int_{C_r(z_0)}^{} \frac{f(z)}{(z-z_0)^{k+1}}dz
\]
Dove $C_r(z_0)$ è una circonferenza centrata in $z_0$ di raggio $r$ percorso una volta in senso antiorario parametrizzato $r(t)=z_0+re^{i t}\ \ t\in[0,2\pi]$, scrivibile anche come $(x_0+r\cos t)+i(y_0+r\sin t)$
\[I_k=\int_{C_r(z_0)}^{} \frac{\sum_{n\ge 0}^{} c_n(z-z_0)^n}{(z-z_0)^{k+1}}dz=\sum_{n\ge 0}^{} c_n\int_{C_r(z_0)}^{} (z-z_0)^{n-k-1}dz\]
È permesso per la convergenza uniforme della serie.
\[\int_{C_r(z_0)}^{} (z-z_0)^ndz=\begin{cases}
	0\ \ m\neq -1\\2\pi i\ \ m=-1 
\end{cases}  \]
Dunque tutti gli integrali nella somma si annullano tranne per $n-k-1=-1\implies n=k$ 
\[=c_k\cdot 2\pi i\implies c_k= \frac{I_k}{2\pi i}= \frac{1}{2\pi i}\int_{C_r(z_0)}^{} \frac{f(z)}{(z-z_0)^{k+1}}dz\]
\begin{tcolorbox}
	\textbf{Formula di Cauchy per la derivata k-esima:} 
	\[f^{(k)}(z)=\frac{k!}{2\pi i}\int_{C_r(z_0)}^{} \frac{f(z)}{(z-z_0)^{k+1}}dz\]
\end{tcolorbox}
In particolare con $k=0$
\[f(z_0)=\frac{1}{2\pi i}\int_{C_r(z_0)}^{} \frac{f(z)}{z-z_0}dz\]
Dove $r$ è un qualsiasi raggio appartenente all'intervallo $(0,R)$
\\\divider
\\\textbf{Osservazione: } \[z\mapsto   \frac{f(z)}{(z-z_0)^{k+1}}\text{ è olomorfa su }D\setminus \{z_0\} \]
\[\implies \int_{C_r(z_0)}^{}  \frac{f(z)}{(z-z_0)^{k+1}}dz \text{ è indipendente dalla scelta di }r\in(0,R)\]
Per $k=0$ vale in realtà una proprietà più forte
\begin{tcolorbox}
	\textbf{Formula di Cauchy} \\
	$f$ olomorfa su $\Omega$ contenente $\overline{B_r(z_0)}$, allora $\forall z\in B_r(z_0)$
	\[f(z)=\frac{1}{2\pi i}\int_{C_r(z_0)}^{}  \frac{f(\xi)}{\xi-z}d\xi\]
\end{tcolorbox}
\textbf{Precisazione: }$B_r(z_0):=\{z\in\C: |z-z_0|<r\}$ 
\\Questa formula è estremamente forte e generica poiché vale per tutte le funzioni olomorfe, non è necessaria l'ipotesi di funzione analitica.
\\\textbf{Osservazione: }\[z\mapsto \frac{f_(\xi)}{\xi-z}\] è somma della serie di potenze generica 
\[\frac{1}{1-z}=\sum_{k}^{} z^k\]
È dunque una funzione analitica.
\subsection{Analiticità e olomorfia}
\begin{tcolorbox}
	\textbf{Teorema di analiticità delle funzioni olomorfe}\\ 
	Sia $f$ olomorfa su $\Omega\implies f$ analitica su $\Omega$
\end{tcolorbox}
\textbf{Osservazioni}
\begin{itemize}
	\item $\impliedby $ (implicazione inversa) è ovvia
	\item Differenza rispetto al caso reale
\end{itemize}
Valgono gli sviluppi già noti dall'analisi reale.
\\

\section{Singolarità isolate e loro classificazione}	
\begin{tcolorbox}
	Sia $f:\Omega \setminus \{z_0\} \subseteq\C\to \C$, si dice che $z_0$ è una \textbf{singolarità isolata} per $f$ se $\exists u(z_0)\subseteq\Omega$ tale che $f$ sia olomorfa se $u(z_0)\setminus \{z_0\}$  
\end{tcolorbox}
Sia $z_0$ una singolarità isolata per $f$.
\subsubsection{Singolarità eliminabile}
\begin{tcolorbox}
Si dice che $z_0$ è una \textbf{singolarità eliminabile} se 
\[\exists u(z_0),\exists\tilde f:u(z_0)\to \C \text{ tale che }\tilde f|_{u(z_0)\setminus \{z_0\} }=f\]
e $\tilde f$ sia olomorfa in $u(z_0)$.
\end{tcolorbox}
Esempio: $f(z)= \frac{\sin z}{z}$
\\\textbf{Osservazione:} Se $\exists \tilde f$, $\tilde f$ è unica.
\\Se una $g$ è olomorfa è anche continua:
\[\lim_{z \to z_0} [g(z)-g(z_0)]=\lim_{z \to z_0} \frac{g(z)-g(z_0)}{z-z_0}(z-z_0)=g'(z_0)\cdot 0=0\]
Ne consegue che il valore che assumera $\tilde f$ in $z_0$ è 
\[\tilde f(z_0)=\lim_{z \to z_0} f(z)\]\\
\textbf{Osservazione 2:} $z_0$ singolarià eliminabile per $f\implies f$ limitata (in modulo) vicino a $z_0$.
\[\exists u(z_0), \exists M>0 \text{ tale che } \|f(z)\|\le M\forall u(z_0)-\{z_0\} \]
Infatti, se $z_0$ singolarità eliminabile per $f\implies \exists \lim_{z \to z_0} f(z)\in\C$
\begin{tcolorbox}
	\textbf{Teorema di rimozione della singolarità} 
	\\Se $f$ olomorfa e limitata in $u(z_0)\setminus \{z_0\} \implies z_0$ è singolarità eliminabile
\end{tcolorbox}
Quindi, in conclusione, se $f$ è olomorfa su $u(z_0)\setminus \{z_0\} $, $z_0$ singolarità eliminabile di $f \iff f$ limitata in $u(z_0)\setminus \{z_0\} $ 
\subsubsection{Polo}
\begin{tcolorbox}
	Si dice che $z_0$ è un \textbf{polo} per $f$ se 
	\[\lim_{z \to z_0} f(z)=\infty\]
\end{tcolorbox}
Esempio: $f(z)=\frac{1}{sz^mm}$ con $m\in \N\setminus \{0\} $
\subsubsection{Singolarità essenziale}
\begin{tcolorbox}
	Si dice che $z_0$ è una \textbf{singolarità essenziale} per $f$ se è una singolarità isolata e non è né eliminabile né polo.
\end{tcolorbox}
Esempio: $f(z)=e^{\frac{1}{z}}$
\\\textbf{Teorema di Picard: }$z_0$ singolarità essenziale per $f\implies \forall u(z_0),f(u(z_0))$ (ovvero l'immagine di $f$) è data da $\C$ oppure $\C\setminus \{1 \text{ punto}\}$.
\subsection{Sviluppabilità in serie di Laurent}
\begin{tcolorbox}
	\textbf{Teorema} $f$ olomorfa su $\Omega\setminus \{z_0\} $ aperto di $\C$, allora $f$ è "sviluppabile in serie di Laurent di centro $z_0$", ovvero
	\[\exists u(z_0)\subseteq  \Omega \text{ tale che }\forall x\in u(z_0)\setminus \{z_0\} \]
	\[f(z)=\sum_{k=-\infty}^{+\infty} c_k(z-z_0)^k\]
	\[=\sum_{k\ge 0}^{} c_k(z-z_0)^k+\sum_{k<0}^{} c_k(z-z_0)^k\]
\end{tcolorbox}
Ovvero parte regolare dello sviluppo + parte singolare dello sviluppo
\\\textbf{Inoltre, il calcolo dei coefficienti:} 
\[c_k=\frac{1}{2\pi i}\int_{C_r(z_0)}^{} \frac{f(z)}{(z-z_0)^{k+1}}dz\] 
In particolare:
\[c_{-1}=\frac{1}{2\pi i}\int_{C_r(z_0)}^{} f(z)dz\]
C'è una relazione tra $c_{-1}$ e gli integrali sui circoli.
Esempio: $f(z)=\frac{1}{z}$
\\Tramite serie di Laurent è possibile riconoscere le singolarità
\\$z_0$ è singolarità eliminabile $\iff$ parte singolare dello sviluppo $=0$.

\section{Riconoscere le singolarità}
\begin{itemize}
	\item $z_0$ eliminabile $\iff$ parte singolare dello sviluppo $=0$
	\item $z_0$ polo 
	\item $z_0$ sing. essenziale 
\end{itemize}
Idea: $z_0$ è un polo per $f\iff z_0 $ è zero per la funzione $1/f$
\[\lim_{z \to z_0} \frac{1}{f}=0\]
\subsubsection{Principio di identità}
Sia $f:\Omega \subseteq  \C\to \C$ olomorfa e supposto $\Omega$ connesso
\\Sia $Z(f)=\{z\in\Omega : f(z)=0\} $, sono equivalenti i seguenti fatti
\begin{enumerate}
	
	\item $z_0\in \text{acc}(Z(f))$
	\item $f^{(n)}(z_0)=0\ \forall n\in\N$
	\item $Z(f)$ contiene un intorno di $z_0$
	\item $Z(f)\equiv\Omega$
\end{enumerate}
\\In conclusione: $Z(f)$
\begin{itemize}
	\item È fatto da punti isolati, oppure 
	\item Coincide con tutto $\Omega$ 
\end{itemize}
\subsubsection{Ordine di zeri}
Sia $f$ olomorfa su $\Omega$ connesso, $f\neq 0$ su $\Omega$, sia $z_0\in Z(f)$, Per il principio di identità, $z_0$ è uno zero isolato.
\\La (2) è quindi falsa $\implies \{n\in\N: f^{(n)}(z_0)\neq 0\} \neq 0$. Per il principio di buon ordinamento $\nu := \text{min}\{n\in\N: f^{(n)}(z_0)\neq 0\} $: \textbf{Ordine dello zero}.\\
\\\textbf{Osservazione}: $\nu$ è anche caratterizzato da:
\[f(z)=\sum_{n\ge \nu}^{} c_n(z-z_0)^n=c_\nu(z-z_0)^\nu+o(z-z_0)^\nu\]
Inoltre
\[\exists \lim_{z \to z_0} \frac{f(z)}{(z-z_0)^\nu}\in\C\setminus \{0\} \]
\subsection{Ordine dei poli}
$z_0$ polo per $f\iff z_0$ zero per $1 / f$
\begin{tcolorbox}	
\textbf{Definizione:} Sia $z_0$ polo per $f$. Chiamiamo ordine del polo $z_0$ l'ordine di $z_0$ come $z_0$ per $1 / f$
\end{tcolorbox}
\textbf{Definizione: }In particolare si dice polo semplice un polo di ordine 1.
\textbf{Osservazione:} L'ordine di un polo è caratterizzato anche:
\begin{itemize}
	\item $z_0$ polo di ordine $\nu$ per $f \iff z_0$ zero di ordine $\nu$ per $1 / f\iff$
\end{itemize}
		\[\exists \lim_{z \to z_0} \frac{1}{f(z)}\cdot \frac{1}{(z-z_0)^\nu}\in\C\setminus \{0\}\]
		\[\implies \exists \lim_{z \to z_0} (z-z_0)^\nu f(z)\in\C\setminus \{0\} \ \text{ (*)}\]

\begin{itemize}
	\item $z_0$ polo di ordine $\nu$ per $f \iff$
\end{itemize}
\[f(z)=\sum_{n=-\nu}^{+\infty} c_n(z-z_0)^n,\text{ con }c_{-\nu}\neq 0\]
Infatti 
\[f(z)=\sum_{n=-\infty}^{+\infty} c_n(z-z_0)^n\implies(z-z_0)^\nu f(z)=\sum_{n=-\infty}^{+\infty} c_n(z-z_0)^{n+\nu}\]
(*)$\iff$ tutti i coefficienti $c_n$ con $n+\nu<0$ e il coefficiente $c_{-\nu}\neq 0$.
\\Dunque, lo sviluppo di Laurent di una funzione che ha un polo ha parte singolare composta da un numero finito di termini.
\\È quindi possibile classificare le singolarità guardando lo sviluppo in serie di Laurent, guardando la parte singolare
\begin{itemize}
	\item p. singolare nulla: ELIMINABILE
	\item p. singolare con numero finito di termini: POLO
	\item p. singolare con infiniti termini: ESSENZIALE
\end{itemize}
\textbf{Osservazioni su zeri e poli} 
\begin{enumerate}
	\item Se $f,g$ hanno entrambe uno zero in $z_0$ o entrambe un polo, allora
\end{enumerate}
\[\exists  \lim_{z \to z_0} \frac{f}{g}=\lim_{z \to z_0} \frac{f'}{g'}\]
Cenno di dim (Primo caso):
\[f(z)=c_\nu(z-z_0)^\nu+o(z-z_0)^\nu, g(z)=c_\eta(z-z_0)^\eta+o(x-x_0)^\eta\]
Sono possibili solo tre casi: $\eta=\nu\implies$ limite finito, $\nu>\eta$ limite 0, $\nu < \eta$ limite infinito.
\begin{enumerate}
	\setcounter{enumi}{1}
	\item $z_0$ zero di ordine $\nu$ per $f \iff \lim_{z \to z_0} \frac{(z-z_0)}{f(z)}=\nu$
		\\
\end{enumerate}
Questa è una modalità per calcolare l'ordine.
Dim: $f(z)=c_\nu(z-z_0)^\nu+\ldots$, $f'(z)=\nu c_\nu(z-z_0)^{(\nu-1)}+\ldots$
\[\implies \frac{(z-z_0)f'(z)}{f(z)}=\frac{\nu c_nu(z-z_0)^\nu+o(z-z_0)^\nu}{c_\nu(z-z_0)^\nu+o(z-z_0)^\nu}\to_{z-z_0}\nu\]
\begin{enumerate}
	\setcounter{enumi}{2}
	\item $z_0$ zero di ordine $\nu$ per $f$, con $\nu \ge 1\implies z_0$ zero di ordine $\nu-1$ per $f'$.
$z_0$ polo di ordine $\nu$ per $f$, con $\nu \ge -1\implies z_0$ polo di ordine $\nu+1\text{ per }f$.
Controllare su libro.
\end{enumerate} 
\subsubsection{Unicità del prolungamento analitico}
\begin{tcolorbox}
	Sia $\Omega \subseteq  \C$ connesso, sia $S \subseteq  \Omega$ tale che $\text{acc}(S)\cap \Omega \neq 0$.
	\\Data $f:S\to \C$, esiste al più una $\tilde f:\Omega\to \C$ olomorfa tale che $\tilde f|_S=f$
\end{tcolorbox}
\textbf{Dimostrazione: }Supponiamo $\tilde f_1,\tilde f_2:\Omega\to \C$ siano prolungamenti di $f$. Tesi: $\tilde f_1\equiv \tilde f_2$
\\Considerando $g:=\tilde f_1-\tilde f_2$. Tesi: $g\equiv 0$.
\\$g$ è olomorfa, $S \subseteq  Z(g)\implies Z(g)$ ha punti di accumulazione in $\Omega$, quindi $Z(g)\equiv \Omega$ 

\section{Teorema dei residui}
\textbf{Motivazione dello studio del teorema: }è il calcolo di integrali in campo complesso e anche in campo reale.
\\Se $f$ è olomorfa su $\Omega \subseteq  \C \implies \int_{\gamma}^{} f(z)dz=0$ dove $\gamma$ è un circuito omotopo a un punto.
Se $f$ è olomorfa su $\Omega$ tranne che in un numero finito di punti, come si calcola $\int_{\gamma}^{} f(z)dz$?
\begin{tcolorbox}
	\textbf{Definizione}\\ 
	Se $z_0$ è una singolarità isolata per $f$ si dice residuo di $f$ in $z_0$ il coefficiente $c_{-1}$ dello sviluppo in serie di Laurent di $f$ di centro $z_0$.
\end{tcolorbox}
\subsection{Calcolo dei residui}
\begin{itemize}
	\item Se $z_0$ è una singolarità eliminabile: $\text{Res}(f,z_0)=0$ poiché la parte singolare dello sviluppo $\equiv 0$
	\item $z_0$ singolarità essenziale: non c'è modo diretto di calcolare il residuo (serve calcolare lo sviluppo)
	\item Se $z_0$ è un polo di ordine $\nu$
\end{itemize}
\[\text{Res}(f,z_0)=\lim_{z \to z_0} \frac{1}{(\nu-1)!}D^{(\nu-1)}[(z-z_0)^\nu f(z)]\]
In particolare se $z_0$ è un polo semplice
\[\text{Res}(f,z_0)=\lim_{z \to z_0} [(z-z_0)f(z)]\]
\subsubsection*{Dimostrazione polo semplice}
$z_0$ polo semplice $\implies f(z)=\sum_{n\ge -1}^{} c_n(z-z_0)^n$, con $c_{-1}\neq  0$
\[(z-z_0)f(z)=\sum_{n\ge -1}^{} c_n(z-z_0)^{n+1}=c_{-1}+c_0(z-z_0)+c_1(z-z_0)^2+o(z-z_0)^2\]
\[\lim_{z \to z_0} [(z-z_0)f(z)]=c_{-1}\]
\textbf{Osservazione:} $\text{Res}(\frac{g}{h},z_0)=\frac{g(z_0)}{h'(z_0)}$ con $g$ olomorfa, $h$ con uno zero di ordine 1 in $z_0$.\\
\textbf{Dimostrazione}\\
Caso $g(z_0)\neq 0\implies z_0$ polo semplice 
\[(z-z_0) \frac{g(z)}{h(z)}= \frac{g(z_0)(z-z_0)+o(z-z_0)}{h'(z_0)(z-z_0)+o(z-z_0)}\to  \frac{g(z_0)}{h'(z_0)}\]
Tramite la formula per il residuo del polo semplice
\[\text{Res}( \frac{g}{h},z_0)= \frac{g(z_0)}{h'(z_0)}\]
Caso G(z0)=0
Dico che $z_0$ è una singolarità eliminabile
\[\frac{g}{h}= \frac{g'(z_0)(z-z_0)+o(z-z_0)}{h'(z_0(z-z_0)+o(z-z_0)}\to \frac{g'(z_0)}{h'(z_0)}\in\C\]
\subsection{Definizione e calcolo dell'indice di avvolgimento}
\begin{tcolorbox}
	\textbf{Definizione (intuitivia)}
	\\Sia $\gamma$ circuito $\subseteq \C$ e sia $z_0 \not\in \gamma$.
	\\Si dice indice di avvolgimento di $\gamma$ rispetto a $z_0$ è il numero di volte che $\gamma$ gira attorno a $z_0$, contate con segno + nel caso di verso antiorario
\end{tcolorbox}
\begin{tcolorbox}
	\textbf{Definizione (formale)} \\
	Sia $r(t):[a,b]\to \C$ parametrizzazione di $\gamma$ ($\gamma$) circuito $\subseteq \C,\ z_0 \not\in \C $.
	\\Sia $\rho (t):=|r(t)-z_0|$. Allora $\exists \theta:[a,b]\to \C$ tale che $r(t)=z_0+\rho(t)e^{i\theta(t)}$.
	\[\text{Ind}(\gamma,z_0):= \frac{\theta(b)-\theta(a)}{2\pi}\in\Z\]
\end{tcolorbox}
L'indice è un numero $\in\Z$ poiché $r(a)=r(b)\implies\rho(a)=|r(a)-z_0|=|r(b)-z_0|=\rho(b)$
\[r(a)=\rho (a)+e^{i\theta(a)}\]
\[r(b)=\rho(b)+e^{i\theta(b)}\]
\[\implies e^{i\theta(a)}=e^{i\theta(b)}\]
\[\implies i\theta(a)-i\theta(b)=2k\pi i = \theta(a)-\theta(b)=2k \pi\]
\textbf{Osservazioni}
\begin{enumerate}
	\item L'indice non cambia per parametrizzazioni equivalenti (dello stesso circuito)
	\item L'indice di avvolgimento non cambia sostituendo $\gamma$ con un circuito omotopo a $\gamma$ in $\C\setminus \{z_0\} $
\end{enumerate}
\subsubsection{Modalità analitica per calcolare l'indice}
\[\text{Ind}(\gamma, z_0)=\frac{1}{2\pi i} \int_{\gamma}^{} \frac{1}{z-z_0}dz\]
\textbf{Dimostrazione} 
\\$r(t)=z_0+\rho(t)e^{i\theta(t)},\ t\in[a,b]$.
\[\int_{\gamma}^{} \frac{1}{z-z_0}dz=\int_{a}^{b} \frac{\rho'(t)e^{i\theta(t)}+\rho(t)i\theta'(t)e^{i\theta(t)}}{z_0+\rho(t)e^{i\theta(t)}-z_0}dt\]
\[=\int_{a}^{b} \frac{\rho'(t)etsi\theta(t)}{\rho(t)e^{i\theta(t)}}dt+i \int_{a}^{ b} \frac{\rho(t)\theta'(t)e^{i\theta(t)}}{\rho(t) e^{i\theta(t)}}dt\]
\[=\log\rho(t)|_a^b+i[\theta(b)-\theta(a)]=i[\theta(b)-\theta(a)]\]
Dunque dividendo
\[\frac{1}{2\pi i}\int_{\gamma}^{} \frac{1}{z-z_0}dz=\frac{\theta(b)-\theta(a)}{2\pi}\]
\subsection{Teorema dei residui}
\begin{tcolorbox}
	Sia $\Omega$ aperto $\subseteq\C$ e sia  $\gamma \subseteq\C$ circuito omotopo a un punto (in $\Omega$).
	\\Sia $f:\Omega\setminus S\to \C$ olomorfa, dove $S$ "insieme singolare" soddisfa
	\begin{itemize}
		\item $\gamma \subseteq\Omega \setminus S$
		\item $\text{acc} (S)\cap\Omega = \emptyset$
	\end{itemize}
	Allora:
	\\$\text{Ind}(\gamma,z_0)\neq 0$ per al più un numero finito di punti e vale
	\[\int_{\gamma}^{} f(z)dz=2\pi i \sum_{z_0\in S}^{} \text{Res}(f,z_0)\text{Ind}(\gamma,z_0)\] 
\end{tcolorbox}


\section{Applicazioni del teorema dei residui in campo reale}
\subsection{Primo tipo}
\begin{tcolorbox}
\[\int_{0}^{2\pi} f(\cos t, \sin t)dt\]
\end{tcolorbox}
\[\cos t = \frac{e^{it}+e^{-it}}{2}\]
\[\sin t = \frac{e^{it}-e^{-it}}{2}\]
Dunque 
\[\int_{0}^{2\pi} f(\cos t, \sin t)dt=\int_{0}^{2\pi} g(e^{it})ie^{it}dt \]
\[\int_{C_1(0)}^{} g(z)dz\]
Se $g$ soddisfa le ipotesi del teorema dei residui su $C_1(0) \subseteq  \Omega$, con $\gamma=C_1(0)$
\[=2\pi i \sum_{|z_0|<1}^{} \text{Res}(g,z_0)\]
Esempio: $\int_{0}^{2\pi} \frac{1}{2+\sin t}dt$
\subsection{Secondo tipo}
\begin{tcolorbox}	
	\[\text{V.P.  }\int_{\R}^{} f(x)dx:= \lim_{R \to +\infty} \int_{-R}^{R} f(x)dx \] 
\end{tcolorbox}
La definizione cambia leggermente nel caso sia presente una singolarità su $\R$.
Se $f$ è integrabile (secondo Riemann) allora 
\[\int_{\R}^{} f(x)dx=\lim_{R \to +\infty} \int_{-R}^{R} f(x)dx\]
In generale può accadere che il V.P. $\int_{\R}^{} f(x)dx \in\R$ ma $f$ non è integrabile
\textbf{Esempio:} 
\[f(x)=\begin{cases}
	\frac{1}{x}\ \ \ x\ge 1\\
	1\ \ \ x\in[0,1]\\
	-1\ \ \ x\in[-1,0]\\
	\frac{1}{x}\ \ \ x\le -1
\end{cases}
\]
$f$ non è integrabile secondo Riemann, ma il V.P. è uguale a 0.\\\divider
\\\textbf{Ipotesi: }$f=f(z)$ abbia un numero finito di singolarità nel semipiano $\{\text{Im}z>0\} $ (e nessuna singolarità sull'asse reale)+ (*) ipotesi di decadimento.
\[I=\lim_{R \to +\infty} \int_{-R}^{-R} f(x)dx+\int_{C_R^+(0)}^{} f(z)dz-\int_{C_R^+(0)}^{} f(z)dz \]
\begin{figure}[ht]
    \centering
    \incfig{semicirconferenza}
    \caption{Semicirconferenza}
    \label{fig:semicirconferenza}
\end{figure}
\[I=\lim_{R \to +\infty} \int_{\gamma_R}{f(z)dz} -\lim_{R \to +\infty} \int_{C_R(0)^+}^{} f(z)dz\]
Dove $\gamma_R=[-R,R]+C_R^+(0)$
\\Per il teorema dei residui
\[I=2\pi i \sum_{\substack{z_0\in S,\\  \text{Im}z_0>0}}^{} \text{Res}(f,z_0)\]
L'indice di avvolgimento è uguale a 1.
\subsubsection{Lemma tecnico di decadimento}
Se $\exists \alpha >1$ tale che $|f(z)|\le \frac{c}{|z|^\alpha}$ (per $|z|$ abbastanza grande) (*), allora
\[\lim_{R \to +\infty} \int_{C_R^+(0)}^{} f(z)dz=0\]
Aggiungendo l'ipotesi di decadimento all'integrale precedente si avrà il risultato scritto.
\\Si ha un calcolo analogo per il semipiano $\{\text{Im}<0\} $
\subsection{Terzo tipo}
\begin{tcolorbox}	
	\[I=\text{V.P.  }\int_{\R}^{} f(x)e^{i\omega x}dx=2 \pi i \sum_{\substack{z_0\in S\\\text{Im}z_0>0}}^{\infty} \text{Res}(f(z)e^{i\omega z},z_0)\] 
\end{tcolorbox}
Dove $\omega \in \R^+$
\\\textbf{Ipotesi: }$f(z)e^{i\omega x}$ abbia un numero finito di singolarità nel semipiano $\{\text{Im}z>0\} $ (e nessuna singolarità sull'asse reale) + (**) lemma di Jordan.
\[I=\lim_{R \to +\infty} \int_{-R}^{-R} f(z)e^{i\omega z}dz+\int_{C_R(0)^+}^{} f(z)e^{i\omega z}dz-\int_{C_R(0)^+}^{} f(z)e^{i\omega z}dz \]
\begin{figure}[ht]
    \centering
    \incfig{semicirconferenza}
    \caption{Semicirconferenza}
    \label{fig:semicirconferenza}
\end{figure}
\[I=\lim_{R \to +\infty} \int_{\gamma_R}{f(z)e^{i\omega z}dz} -\lim_{R \to +\infty} \int_{C_R(0)^+}^{} f(z)e^{i\omega z}dz\]
Dove $\gamma_R=[-R,R]+C_R^+(0)$
\\Per il teorema dei residui
\[I=2\pi i \sum_{\substack{z_0\in S,\\  \text{Im}(z_0)>0}}^{} \text{Res}(f(z)e^{i\omega x},z_0)\]
L'indice di avvolgimento è uguale a 1.\\
Il secondo termine dell'integrale si elide grazie a il 
\subsubsection{Lemma di Jordan}
Sotto l'ipotesi 
\[\lim_{R \to +\infty}\text{sup}_{z\in c_R^+(0)}|f(z)|=0  \ \ (**)\]
\[\lim_{R \to +\infty} \int_{C_R^+(0)}^{} f(z)e^{i\omega x}dz=0\] 
\textbf{Osservazione:} Variante analoga nel semipiano $\{\text{Im}z<0\} $ quando $\omega \in \R^-$
\\Jordan vale anche per $\omega \in \R^-$ in $C_R^-(0)$
\\Esempio: $I= \text{V.P.}\int_{\R}^{} \frac{\cos x}{1+x^2}dx$
\subsection{Quarto tipo}
\begin{tcolorbox}
\[I=\text{V.P.}\int_{\R}^{} f(x)dx\]	
\end{tcolorbox}
\textbf{Ipotesi:} $f(z)$ abbia un numero finito di singolarità su $\{\text{Im}>0\} $, \\$\lim_{R \to +\infty} \int_{C_R^+(0)}^{} f(z)dz=0$ (***) e abbia un numero finito di poli semplici su $\R$.
\[\gamma_{R,\varepsilon}=[-R,x_0-\varepsilon]-C_\varepsilon^+(x_0)+[x_0+\varepsilon,R]+C_R^+(0)\]
\[I=\lim_{R \to +\infty} \int_{\gamma_{R,\varepsilon}}f(z)dz + \lim_{\varepsilon \to 0} \int_{C_\varepsilon^+(x_0)}^{} f(z)dz-\lim_{R \to +\infty} \int_{C_R^+(0)}{f(z)dz}\]
\subsubsection{Lemma del polo semplice}
Se $x_0$ è un polo semplice
\[\lim_{\varepsilon \to 0} \int_{C_\varepsilon^+(x_0)}^{} f(z)dz=\pi i \text{Res}(f,x_0)\] 
\begin{figure}[ht]
    \centering
    \incfig{quarto-tipo}
    \caption{Quarto tipo}
    \label{fig:quarto-tipo}
\end{figure}
Esempio: $I=\text{(V.P.)}\int_{\R}^{} \frac{1-\cos 2x}{x^2}dx$
\section{Cenni aggiuntivi sull'analisi complessa}
\subsection{Residuo all'infinito}
\begin{tcolorbox}
	\textbf{Definizione:} Diciamo che $\infty$ è una singolarità isolata per $f$ se $f$ olomorfa nel complementare di una palla
\end{tcolorbox}
In modo equivalente: $g(z)=f(\frac{1}{z})$ ha una singolarità isolata nell'origine.
\[\text{Olomorfa su }\bigg|\frac{1}{z}\bigg|>R\iff |z|<R\]
\[\text{Res}(f,\infty):=\text{Res}\bigg(-\frac{1}{z^2}f\bigg(\frac{1}{z}\bigg),0\bigg)\]
\begin{tcolorbox}
	\textbf{Teorema:} La somma di tutti i residui di una funzione olomorfa su $\C\setminus \{\text{n. finito di punti}\} $ è zero. (compreso il punto all'infinito).
\end{tcolorbox}
Da utilizzare quando si deve calcolare la somma di tanti residui al finito. (stesso indice di avvolgimento)
\subsection{Funzioni polidrome}
$z=|z|e^{i\text{Arg}z}$, $\text{Arg}z:=\{\text{arg}+2k\pi:k\in\Z\} $, $\text{arg}z$ argomento principale $\in[0,2\pi]$.
\[\sqrt[n]{z}=\{\sqrt[n]{|z|}e^{i \frac{\text{Arg}z}{n}} \}=\{\sqrt[n]{|z|}e^{i( \frac{\text{arg}z}{n}+ \frac{2k\pi}{n})}:k=0,\ldots,n-1 \} \]
\[\log z=\{\log |z|+i\text{Arg}z\}\]
Alla radice sono associati n valori, al logaritmo $\infty$ valori.
\\$z \mapsto \sqrt[n]{z},\log z$ non sono funzioni!
\\Per definire una radice n-esima funzione si può specificare l'intervallo di variabilità di $\text{Arg}z$.
$z\in\C\mapsto \sqrt[n]{|z|} e^{i \frac{\text{Arg}z}{n}}$ con $\text{Arg}z\in[\overline \theta,\overline\theta+2\pi]$: Branca della radice n-esima.
\\\textbf{Osservazione:} Una branca della radice n-esima non è continua su $\C$. (è continua su $\C-\{\theta=\overline\theta\}$
\\Non è possibile incollare due branche diverse ottenendo una funzione olomrofa su $\C$.
\subsection{Funzioni armoniche}
\begin{tcolorbox}
	\textbf{Definizione:} $u:\R^2\to \R$ si dice armonica se 
	\[\nabla^2 u=0=u_{x x}+u_{y y}\]
\end{tcolorbox}
\textbf{Osservazione:} $f=f(z)$ olomorfa, $f=u+iv\implies u,v$ armoniche.
\[\begin{cases}
	u_x=v_y
	\\u_y=-v_x
\end{cases}
\begin{cases}
	u_{x x}=v_{y x}\\
	u_{y y}=-v_{xy}
\end{cases}
\]
Sommando le due equazioni
\[u_{x x}+u_{y y}=0\]
(Analogamente per $\nabla ^2 v=0$
\\\textbf{Osservazione 2:} $u$ armonica su $\Omega$, con $\Omega$ semplicemente connesso $\implies\exists v$ armonica coniugata di $u$ t.c. $f=u+iv$ olomorfa.

\section{Analisi funzionale}
Uno \textbf{spazio vettoriale} (su $\R$) è un insieme (V) su cui sono definite due operazioni:
\\\textbf{Somma} $+:V \times V\to V$
\\\textbf{Prodotto per scalare} $\cdot :\R \times \to V$
\\Tali operazioni godono delle seguenti proprietà
\\Per la somma
\begin{itemize}
	\item $u+v=v+u$ 
	\item $u+(v+w)=(u+v)+w$
	\item $u+\underline 0=u$
	\item $u+(-u)=\underline 0$
\end{itemize}
Per il prodotto per scalare
\begin{itemize}
	\item $(ts)u=t(su)$
	\item $t(u+v)=tu+tv$
	\item $(t+s)u=tu+su$
	\item $1\cdot u=u$
\end{itemize}
\subsubsection{Norma}
\begin{tcolorbox}
	Sia $V$ uno spazio vettoriale su $\R$.
	\\Una norma su $V$ è una funzione $\|.\|:V\to \R^+$ tale che:
	\begin{itemize}
		\item $\|v\|>0\forall v\in V-\{\underline 0\} $ (positività)
		\item $\|tv\|=|t|\|v\|\forall t\in\R,\forall v\in V$ (omogeneità)
		\item $\|u+v\|\le \|u\|+\|v\|\forall u,v\in V$ (dis. triangolare)
	\end{itemize}
\end{tcolorbox}
\begin{tcolorbox}
	$(V,\|.\|)$ si dice \textbf{spazio vettoriale normato}.
\end{tcolorbox}
Seguono le seguenti proprietà (resettare counter)
\begin{enumerate}
	\item $\|\underline 0\|=0$
	\item $| \|u\|-\|v\| |\le \|u-v\|\forall u,v\in V (dim)$
\end{enumerate}
Norma euclidea:
\[\|x\|_2=\sqrt{x_1^2+x_2^2+\ldots+x_n^2} =\sum_{i=1}^{n} (x_i^2)^\frac{1}{2}\]
\[\|x\|_\infty=\max_{i=1,\ldots,n}|x_i|\]
\[\|x\|_p=(\sum_{i=1}^{n}(|x|^p)^{\frac{1}{p}}\]
La disuguaglianza triangolare per la norma $p$, ovvero disuguaglianza di Minkowski, richiede l'ipotesi $p\ge 1$.
\subsubsection{Norma su uno spazio funzionale di dimensione infinita}
$V=C^0([a,b])$
\[\|f\|_\infty:=\text{max}_{x\in[a,b]}|f(x)|\]
\[\|f\|_1:=\int_{a}^{b} |f(x)|dx\]
\begin{tcolorbox}	
\textbf{Definizione:} Sia $(V,\|.\|)$ sp. vettoriale normato. Allora $d(u,v):=\|\underline{u}-\underline{v}\|$ definisce una distanza su $V$, ovvero
\[d:V\times V\mapsto \R\]
tale che
\begin{enumerate}
	\item $d(u,v)\ge 0$ con $=0\iff u=v$ positività
	\item $d(u,v)=d(v,u)$ simmetria
	\item $d(u,v)\le d(u,w)+d(w,v)$ disuguaglianza triangolare
\end{enumerate}
\end{tcolorbox}
\[d_p(x,y)=\|x-y\|_p=(\sum_{i=1}^{n} \|x_i-y_i\|^p)^{\frac{1}{p}}\]
\[d_\infty(f,g)=\max_{x\in[a,b]}|f(x)-g(x)|\]
\[d_1(f,g)=\int_{a}^{b} |f(x)-g(x)|\]
\begin{tcolorbox}
	\textbf{Definizione:} $(V,d)$ si dice spazio metrico
\end{tcolorbox}
\textbf{Osservazione:} In uno spazio metrico possiamo definire le \textbf{sfere}: dato $r\ge 0,v_0\in V$
\[B_r(v_0):=\{v\in V:d(v,v_0)<r\} \]
Si dice sfera chiusa se la disuguaglianza non è stretta.\\
\begin{tcolorbox}	
\textbf{Definizione:} Se $\{v_n\} \subseteq  (V,\|.\|)$, si dice che $v_n\to v$ in $V$ se
\[d(v_n,v)\to 0\text{, oppure }\|v_n-v\|\to 0\]
\[(\forall \varepsilon>0\exists \nu\in\N:d(v_n,v)=\|v_n-v\|<\varepsilon\forall n\ge \nu)\]
\end{tcolorbox}
Alcuni fatti veri in dimensione finita ma falsi in dimensione infinita:
\begin{enumerate}
	\item Tutte le norme sono "equivalenti" fra loro
	\item Tutte le successioni di Cauchy convergono
	\item Tutti i sottospazi vettoriali sono chiusi
\end{enumerate}
1) Norme equivalenti\\
\textbf{Definizione:} Sia $V$ uno spazio vettoriale (su $\R$), consideriamo su $V$ due possibili norme $ \|.\|,|\|.\||$.
\\Queste due norme si dicono \textbf{equivalenti} se:
\begin{enumerate}
	\item $\exists c>0:\forall v\in V,\|v\|\le c|\|v\||$
	\item $\exists c'>0:\forall v\in V,|\|v\||\le c\|v\|$
\end{enumerate}
Le successioni convergenti nelle due norme sono le stesse.
\\Interpretazione geometrica:
\[\|.\|_\infty\le \|.\|_1\iff B_1^1(0)\subseteq  B_1^\infty(0)\]
\begin{tcolorbox}
	\textbf{Teorema:} Se $\text{dim}V<+\infty\implies$ tutte le norme su $V$ sono tra loro equivalenti
\end{tcolorbox}
In uno spazio a dimensione infinita non è generalmente vero, ad esempio nello spazio delle funzioni continue su $[a,b]$
\begin{tcolorbox}
	\textbf{Definizione:} Sia $(V,\|.\|)$ uno spazio vettoriale normato. Una successione $\{v_n\} \subseteq V$ si dice successione di cauchy se 
	\[\forall \varepsilon>0 \exists \nu\in\N:d(v_n,v_m)<\varepsilon\ \forall n,m\ge \nu\]
\end{tcolorbox}
\textbf{Osservazione:} Vale sempre che se $\{v_n\} $converge allora è di Cauchy.
\[(\|v_n-v_m\|=\|v_n-v+v-v_m\|\le \|v_n-v\|+\|v_m-v\|<\frac{\varepsilon}{2}+\frac{\varepsilon}{2}=\varepsilon\]
\begin{tcolorbox}
	\textbf{Teorema:} Se $\text{dim}V<+\infty$ vale anche il viceversa, ovvero
	\[\{v_n\} \text{ converge }\iff \{v_n\} \text{ di Cauchy}\]
\end{tcolorbox}
Questo teorema è falso se $\text{dim}V=+\infty$
\\Prendendo lo spazio delle funzioni $V=C^0([a,b])$ con norma 1, si può costruire una successione di Cauchy che non converge.
\[f_n(x)=\begin{cases}
	-1\ \ \ x\le -\frac{1}{n}
	\\\frac{1}{n}\ \ \ -\frac{1}{n}<x<\frac{1}{n}
	\\1\ \ \ x\ge \frac{1}{n}
\end{cases}
\]
\[\|f_n-f_m\|_1<\varepsilon,\ \int_{a}^{b} |f_n-f_m|=\int_{a}^{b} f_m-f_n\]
\subsection{Spazio di Banach}
\begin{tcolorbox}
	\textbf{Definizione:} Uno spazio vettoriale normato $(V,\|.\|)$ si dice completo o di Banach se tutte le successioni di Cauchy convergono.
\end{tcolorbox}
\textbf{Teorema/osservazione: }$V=(C^0([a,b],\|.\|_\infty)$ è di Banach
\\Generalizzazione:
\[V=C^k([a,b])\ \ \|f\|_{C^k}:=\sum_{|\alpha|\le k}^{} \|D^\alpha f\|_\infty\]
\begin{tcolorbox}
	\textbf{Teorema} Sia $(V,\|.\|)$ spazio vettoriale normato. Se $\text{dim}V<+\infty\implies$ tutti i sottospazi vettoriali $W$ sono \textbf{chiusi}.
	\[\{v_n\} \subseteq  W,v_n\to v\text{ in }V\implies v\in W\]
\end{tcolorbox}
Il teorema diventa falso se $\text{dim}V=+\infty$, ad esempio $V=(C^0([a,b]), \|.\|_\infty)$.

\section{Integrazione secondo Lebesgue}
\begin{enumerate}
	\item Misure e funzioni misurabili
	\item Definizione di integrale di Lebesgue
	\item Confronto con Riemann
	\item Teoremi principali
\end{enumerate}
\begin{figure}[ht]
    \centering
    \incfig{lebesgue}
    \caption{Integrale secondo Lebesgue, intuizione geometrica}
    \label{fig:lebesgue}
\end{figure}
\[\int_{a}^{b} f=\lim_{N \to +\infty} \sum_{k=1}^{N}l(f^{-1}(j_k))\cdot y_k\]
\subsubsection{Misure e funzioni misurabili}
\begin{tcolorbox}
	\textbf{Definizione:} Sia $X$ insieme, e sia $F\subseteq P(X)$ una famiglia di sottoinsiemi di $X$.
	\\$F$ si dice una $\sigma$-algebra se:
	\\(i) $\varnothing\in F$
	\\(ii) $A\in F\implies X\setminus A\in F$
	\\(iii) $\{A_n\} _{n \in \N}\subseteq F\implies \bigcup_{n}A_n\in F$
\end{tcolorbox}
\textbf{Osservazione:} $\{A_n\} _{n\in\N}\subseteq F\implies \bigcap_n A_n \in F$
\\Esempi
\begin{itemize}
	\item $X$ qualsiasi, $F=P(X)=$ parti di $X$
	\item $X=\R^n$, $F=$ la pià piccola $\sigma$-algebra contenente gli aperti ($\sigma$ di Borell) 
\end{itemize}
\begin{tcolorbox}
	\textbf{Definizione: } la coppia $(X,F)$ si dice spazio misurabile

\end{tcolorbox}
\begin{tcolorbox}
	\textbf{Definizione:} Sia $(X,F)$ spazio misurabile, una misura positiva su $(X,F) $ è una funzione
\[\mu:F\to \R\cup \{+\infty\} \] 
tale che
\begin{enumerate}
	\item $\mu(A)\ge 0\forall A\in F$ (positività)
	\item Se $\{A_n\} $ è una famiglia al più numerabile di insiemi di $F$ 2 a 2 disgiunti allora 
\end{enumerate}
\[\mu(\cup _n A_n)=\sum_{n\ge 1}^{} \mu(A_n)\]
(additività, eventualmente $+\infty=+\infty$)
\end{tcolorbox}
Esempi:
\begin{itemize}
\item $(X,P(X)),\ \mu(A)=\text{card}A$
\item $(X,P(X))$ fissato $x_0\in X$
	\[\mu(A)=\begin{cases}
		1\text{  se }x_0\in A\\
		0\text{  se }x_0\not\in A
	\end{cases}\]
\end{itemize}
\textbf{Osservazione:} Seguono da 1), 2)
\begin{enumerate}
\setcounter{enumi}{2}
	\item $A_1\subseteq A_2\subseteq \ldots,\ A\in F$\[\implies \mu(\cup_n A_n)=\lim_{n \to \infty} \mu(A_n)\]
	\item $A_1 \supseteq A_2\supseteq A_3\supseteq \ldots, \ A_i\in F,\mu (A_1)<+\infty$\[\implies\mu( \cap_n A_n)=\lim_{n \to \infty}\mu(A_n)\] 
\end{enumerate}
\begin{tcolorbox}
	\textbf{Teorema:} Esistono su $\R^n$ una $\sigma$-algebra $M$ (misurabile secondo lebesgue) e una misura positiva m (misura di Lebesgue in $\R^n$) tali che:
	\begin{itemize}
		\item Tutti gli insiemi aperti appartengono a $M$
		\item $A\in M$ e $m(A)=0\implies\forall B\subseteq A,\ B\in M \text{ e } m(B)=0$ (completezza)
		\item $A=\{x\in \R^n: a_i<x_i<b_i\ i=1,\ldots,n\}$\[\implies m(A)=\prod_{i=1}^n(b_i-a_i)=(b_1-a_1)(b_2-a_2)\ldots(b_n-a_n)\]
	\end{itemize}
	[\ldots] 
\end{tcolorbox}
\textbf{Osservazione: }Non tutti i sottoinsiemi di $\R^n$ sono misurabili secondo Lebesgue.
\\\textbf{Osservazione: }La misura di Lebesgue in $\R^n$ estende il concetto di volume n-dimensionale
\\\textbf{Osservazione:} Gli insiemi di misura nulla sono importanti 
\begin{tcolorbox}
	\textbf{Definizione:} Una funzione $f:\R^n\to \R$ si dice misurabile secondo Lebesgue se
	\[\forall A\subseteq \R \text{ aperto},\ f^{-1}(A) \text{ misurabile secondo Lebesgue}\]
	\[\forall C\subseteq\R\text{ chiuso }, \ f^{-1}(C)\text{ misurabile secondo Lebesgue}\]
\end{tcolorbox}
\textbf{Osservazione:} $f$ continua $\implies$f misurabile secondo Lebesgue ($f$ continua $\implies\forall A$ aperto $f^{-1}(A)$ aperto$\implies \forall  A$ aperto $f^{-1}(A)$ misurabili
\\\textbf{Osservazione 2:} Sono misurabili anche limiti, inferiore, superiore di funzinoni continue (di funzioni misurabili)\bigbreak
Più in generale se 
\[f: E\to \R\] con $E$ misurabile, $f$ si dice misurabile secondo Lebesgue se $\forall  A\subseteq\R$ aperto $E\cap f^{-1}(A)$ misraubile secondo Lebesgue
\subsubsection{Definzione di integrale secondo Lebesgue}
Sia $f:E$ misurabile $\subseteq\R^n\to \R$ misurabile.
\\\textbf{Funzioni semplici}
\\$S$ funzione semplice è una funzione (misurabile) che assume un numero finito di valori (ciascuno su un insieme misurabile).
\[S=\sum_{k=1}^{N} \alpha_k\chi_{E_i},\ \chi_E=\begin{cases}
	1\ \ x\in E
	\\0\ \ x\not\in E
\end{cases}\]
Dove gli $E_i$ sono insiemi misurabili 2 a 2 disgiunti 
\[\int_{E}^{} S:=\sum_{k=1}^{N} \alpha_k m(E_k)\]
\textbf{Precisazione:} con la convenzione $0\cdot \infty=0$
\\\textbf{Funzioni misurabili} $f\ge 0$
\[\int_{E}^{} f:=\sup_{\substack{S \text{ semplici}\\S\ge  f}}\int_{E}^{}S\ \ \ \bigg(=\inf_{\substack{S \text{ semplici}\\S\ge f}} \int_{E}^{S}\bigg)\]
\textbf{Funzioni misurabili di segno qualsiasi}
\\Data $f$ misurabile su $E$ misurabile, scriviamo:
\\$f=f^+-f^-$ con $f^+,f^-\ge 0$, $f^+:=\max \{f,0\} $, $f^-:=-\min \{f,0\} $
\begin{figure}[ht]
    \centering
    \incfig{funzioni-di-segno-qualsiasi}
    \caption{Funzioni di segno qualsiasi}
    \label{fig:funzioni-di-segno-qualsiasi}
\end{figure}
\[\int_{E}^{} f:=\int_{E}^{} f^+-\int_{E}^{} f^-\]
A patto che almeno uno tra i due integrali sia finito, (eventualmente l'integrale vale $\pm \infty$
\begin{tcolorbox}
	\textbf{Definizione:} $f:E\to \R$ misurabile si dice \emph{integrabile secondo Lebesgue} se
	\[\int_{E}^{} f\in\R\] 
\end{tcolorbox}
\textbf{Osservazione:} $f$ è integrabile secondo Lebesgue $\iff \int_{E}^{} f^{\pm}\in\R $
\\Quindi $f$ integrabile seconodo Lebesgue $\iff|f|$ integrabile secondo Lebesgue, infatti
\[|f|=f^++f^-\]
\subsubsection{Proprietà principali dell'integrale di Lebesgue}
1) \textbf{Linearità: }$f,g$ Lebesgue integrabili, $\alpha, \beta\in\R\implies \alpha f+\beta g$ Lebesgue integrabile e 
\[\int_{E}^{} (\alpha f +\beta g)=\alpha \int_{E}^{} f+\beta \int_{E}^{} g\]
2) \textbf{Monotonia: }$f,g$ Lebesgue integrabili, $f\le g$ q.o. su $E$
\[\implies \int_{E}^{f} \le \int_{E}^{} g\]
3) \textbf{Maggiorazione del modulo: }$f$ Lebesgue integrabile 
\[\implies \bigg|\int_{E}^{} f\bigg|\le \int_{E}^{} |f|\]
Segue da 2), $-|f|\le f\le |f|\implies-\int_{E}^{} |f|\le \int_{E}^{} f\le \int_{E}^{} |f|$
\bigbreak4) L'integrale di Lebesgue "non vede" gli insiemi di misura nulla.
\\Sia $S$ semplice, $E\to \R$
\[S(x)=\begin{cases}
	0\text{ su }E\setminus N
	\\1\text{ su }N
\end{cases}\ \ m(N)=0\]
\[\int_{E}^{} S=m(E\setminus N)\cdot 0+m(N)\cdot 1=0\]
Più in generale, se $f$ misurabile: $E\to \R$ se $f$ si annulla su $E$ tranne che su un insieme di misura nulla
\[\int_{E}^{} f=0\] 
Conseguenza: se $f,g$ misurabili: $E\to \R$ se $f=g$ su $E$ tranne che su un insieme di misura nulla
\[\int_{E}^{} f=\int_{E}^{} g\]  
\begin{tcolorbox}
	\textbf{Definizione:} Si dice che una proprietà $P(x)$ vale per q.o. $x\in E$ se $P(x)\text{ vale }\forall x\in E\setminus N,\text{ con }m(N)=0$
\end{tcolorbox}
Quindi
\begin{itemize}
	\item $f=0$ q.o. su $E\implies \int_{E}^{} =0 $
	\item $f=g$ q.o. su $E\implies \int_{E}^{} f=\int_{E}^{} g $
\end{itemize}
\subsection{Confronto Riemann-Lebesgue}
\subsubsection{Integrali propri}
$f$ R-integrabile $\implies f$ L-integrabile, in caso affermativo, i valori degli integrali coincidono, in generale non vale il viceversa.
\\($\implies$) Le funzioni semplici seconodo Lebesgue $S_L$ sono una classe più ampia delle funzinoi semplici secondo Riemann $S_R$
\[\sup_{\substack{s\in S_R\\s\le f}}\int_{E}^{}s\le \sup_{\substack{s\in S_L\\s\le f}}\int_{E}^{} s \le \inf_{\substack{s \in S_L\\s\ge f}}\int_{E}^{}  s\le   \inf_{\substack{s\in S_R\\s\ge f}}\int_{E}^{}s\]
Controesempio: $\exists f$ L-integrabile ma non R-integrabile.
\[f(x)=\begin{cases}
	1\text{ se }x\in\Q\\
	0\text{ se }x\in \R-\Q
\end{cases}\]
Non R-integrabile poiché approssimando da sotto e da sopra non si trova lo stesso valore
\[s\in S_R,s\ge f\implies s\ge 1\text{ su }(0,1)\implies \int_{0}^{1} s\ge 1\]
\[s\in S_R, s\le f\implies s\le 0\text{ su }(0,1)\implies \int_{0}^{1} s\le 0\]
$f$ è però L-integrabile, $\int_{0}^{1} f=0$
\[\int_{0}^{1} f=1\cdot m(0,1)\cap\Q+0\cdot m((0,1)\cap(\R\setminus\Q))=0\]
\subsubsection{Integrali impropri}
In $\R$, supponiamo che $f$ limitata, sia R-integrabile su $[-L,L]\forall L>0$.
\\Allora: $f$ L-integrabile su $\R \iff|f|$ R-integrabile (in senso improprio su $\R$).
\\E in tal caso l'integrale di Lebesgue di $f$ coincide con l'integrale improprio di $f$.
\\Analogamente se $f$ non è limitata.
\\Controesempio: una funazione R-integrabile ma non R-integrabile in modulo e quindi non L-integrabile.
\[f(x)= \frac{\sin x}{x}\text{  su }(0,+\infty)\]
Riemann integrabile su $(0,+\infty)$ (tramite analisi complessa)
\\Ma non è Riemann integrabile in modulo
\[\int_{0}^{+\infty} \frac{|\sin x|}{x}=+\infty\]
(Tramite serie)
